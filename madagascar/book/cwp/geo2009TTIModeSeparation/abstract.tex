\begin{abstract}
Seismic waves propagate through the earth as a superposition of
different wave-modes.
% WE imaging - complex geology
Seismic imaging in areas characterized by complex geology requires
techniques based on accurate reconstruction of the seismic wavefields.
A crucial component of the methods in this category, collectively
known as wave-equation migration, is the imaging condition
which extracts information about the discontinuities of physical
properties from the reconstructed wavefields at every location in
space.
% distinction between acoustic (scalar) and elastic (vector)
Conventional acoustic migration techniques image a scalar wavefield
representing the P wave-mode, in contrast with elastic migration
techniques which image a vector wavefield representing both the P and
S wave-modes.
% IC - what is the physical meaning? - must separate wave modes
For elastic imaging, it is desirable that the reconstructed vector
fields are decomposed in pure wave-modes, such that the imaging
condition produces interpretable images, characterizing for example PP
or PS reflectivity.
% How are wave modes separated in anisotropic media? - projection on
%polarization (k-domain) - or filtering (x-domain)
In anisotropic media, wave-mode separation can be achieved by
projection of the reconstructed vector fields on the polarization
vectors characterizing various wave modes. For heterogeneous media,
the polarization directions change with position, therefore wave-mode
separation needs to be implemented using space-domain filters.
% TTI vs. VTI
For transversely isotropic media with a tilted symmetry axis (TTI),
the polarization vectors depend on the elastic material parameters,
including the tilt angles. Using these parameters, I separate
the wave-modes by constructing nine filters corresponding to the nine
Cartesian components of the three polarization directions at every
grid point.
% 3D vs. 2D - polarization vectors by orthogonality
Since the S polarization vectors in TI media are not defined in the
singular directions, e.g. along the symmetry axes, I construct these
vectors by exploiting the orthogonality between the SV and SH
polarization vectors, as well as their orthogonality with the P
polarization vector. This procedure allows one to separate S wave-modes
which are only kinematically correct.
% what we show in this chapter? - discuss the main examples
Realistic synthetic examples show that this wave-mode separation is
effective for both 2D and 3D models with high heterogeneity and strong
anisotropy.
\end{abstract}
