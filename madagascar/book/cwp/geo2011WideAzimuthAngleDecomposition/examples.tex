\section{Examples}

% ------------------------------------------------------------
\inputdir{seam357}
% ------------------------------------------------------------

We illustrate the method discussed in the preceding section with
common-image-point-gathers constructed using the wide-azimuth SEAM
data \cite[]{}. \rFg{velo} shows the velocity model in the area used
for imaging. For demonstration, we consider $16$ shots located at the
locations of the thick dots in \rfg{vpwin}. The thin dots represent all
the $357$ shots available in one of the SEAM data subsets. The
solid lines in \rfgs{vpmig-037449}-\rfn{vpmig-041729} depict the
decimated receiver lines for each of the $3$ shots shown. In all
panels \rfn{vpmig-037449}-\rfn{vpwin}, the large dot indicates the
surface projection of the CIP used for illustration, located at
coordinates $\{x,y,z\}=\{23.450,11.425,2.38\}$~km. For this example we
consider the azimuth reference vector oriented in the $x$ direction,
i.e. $\vv=\{1,0,0\}$.

% ------------------------------------------------------------
\sideplot{velo}{width=\textwidth}{A subset of the SEAM velocity model
 used for the imaging example in \rfgs{cstk}-\rfn{cang}.}
% ------------------------------------------------------------
\sideplot{cstk}{width=\textwidth}{Conventional image obtained using
 wavefield extrapolation with the $16$ shots shown in \rfg{vpwin}.}
% ------------------------------------------------------------

% ------------------------------------------------------------
\multiplot{2}{vpmig-037449,vpmig-041729,vpmig-043873,vpwin}{width=0.45\textwidth}
{Geometry of SEAM imaging experiment. Panels (a)-(c) show the position
of one shot and the associated receiver lines (decimated by a factor
of $30$ in the $y$ direction. Panel (d) shows the locations of the
$16$ shots used for creating the image shown in \rfg{cstk}.}
% ------------------------------------------------------------

\rFgs{eic-037449}-\rfn{eic-043873} show the extended image obtained at
the CIP location indicated earlier using migration by downward
continuation. The extended image cubes use $41$ grid points in the
$\hx$ and $\hy$ directions sampled on the image grid, i.e. at every
$30$~m, and $31$ grid points in the $\tt$ direction sampled on the
data grid, i.e. at every $8$~ms. The vertical lag $\hz$ is not
computed in this example, since the analyzed reflector is
nearly-horizontal. This lag is computed in the decomposition process
from the horizontal lag and from the known information about the normal to
the reflector at the given position. \rFg{estk} shows the extended
image obtained for all $16$ shots used for imaging. Although here we
show the extended image cubes for independent shots, in practice these
cubes need not be computed separately -- the decomposition separates
the information corresponds to different angles of incidence, as shown
in this simple example.

Finally, \rfgs{cang-037449}-\rfn{cang} show the angle-domain
decomposition of the extended image cubes shown in
\rfgs{eic-037449}-\rfn{estk}, respectively. In these plots, the
circles indicating the reflection angles are drawn at every $5^\circ$
and the radial lines indicating the azimuth directions are drawn at
every $15^\circ$. Given the sparse shot sampling, the CIP is sparsely
illuminated, but at the correct reflection and azimuth angles.

% ------------------------------------------------------------
\multiplot{2}{eic-037449,eic-041729,eic-043873,estk}{width=0.45\textwidth}
{Extended image cubes for the SEAM imaging experiment. Panels (a)-(c)
  show extended image cubes at the same location for $3$ different
  shots, and panel (d) shows the extended image obtained for all $16$
  shots considered in this experiment.}
% ------------------------------------------------------------

% ------------------------------------------------------------
\multiplot{2}{cang-037449,cang-041729,cang-043873,cang}{width=0.35\textwidth}
{Reflectivity as a function of reflection and azimuth angles for the
  SEAM imaging experiment. Panels (a)-(c) show the angle-domain CIPs
  at the same location for $3$ different shots, and panel (d) shows
  the angle-domain CIP obtained for all $16$ shots considered in this
  experiment.  The angles $\phi$ and $\theta$ are indexed along the
  contours using the trigonometric convention and along the radial
  lines increasing from the center.}
% ------------------------------------------------------------
