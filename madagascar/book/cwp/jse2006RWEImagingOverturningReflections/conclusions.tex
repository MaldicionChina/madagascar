\section{Conclusions}
This paper demonstrates the applicability of Riemannian wavefield extrapolation to the problem of imaging overhanging salt flanks. Imaging such reflectors using one-way wavefield extrapolation in Cartesian coordinates is impractical since waves propagate partially down, partially up. A possible solution to this problem consists of using tilted Cartesian coordinate systems. This procedure partially reduces the angle between the direction of wave propagation and the direction of extrapolation. However, even in this coordinate framework, waves need to be extrapolated at high angles up to $90^\circ$ which degrades the imaging accuracy.
\par
In contrast, wavefield extrapolation in Riemannian coordinates has the flexibility to follow closely the paths of wave propagation. Therefore, the relative angle between the direction of extrapolation and the direction of wave propagation is much smaller than in the case of extrapolation in tilted Cartesian coordinates, thus improving imaging accuracy.
\par
Overturning reflections can, in principle, be imaged using Kirchhoff migration. However, this imaging procedure has difficulty producing accurate images in complex geology characterized by wave multipathing and sharp velocity variation. In contrast, imaging overturning reflections using Riemannian wavefield extrapolation benefits from all characteristics of one-way wavefield extrapolation, i.e. stability accross boundaries between media with large velocity variation, multipathing, etc. 
