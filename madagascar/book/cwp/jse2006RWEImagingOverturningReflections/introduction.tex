\section{Introduction}
Imaging of steeply-dipping reflectors, e.g. faults or salt flanks, is a crucial step in seismic imaging of complex geology. In particular, accurate positioning of overhanging salt-flanks influences the quality of migrated images in subsalt regions which are increasingly regarded as the most important targets for seismic exploration. 
\par
This challenge for seismic imaging lead to development of many techniques addressing this problem. Among the developned techniques, we can identify:
\begin{itemize}
\item {\bf Kirchhoff migration} techniques based on traveltimes computed from overturning rays \cite[]{SEG-1991-1178,GEO66-05-16221640}. Such techniques could be used for imaging of reflections at arbitrary dip angles. However, traveltime computation in complex velocity media requires model approximations, e.g. smoothing of sharp velocity boundaries. Furthermore, Kirchhoff migration using multiple arrivals is possible, but technically challenging.
\item {\bf Reverse-time migration}, based on solutions of the acoustic wave-equation, also has the potential to image reflectors at arbitrary dip angles. Furthermore, such techniques allow for imaging of multiply-reflected waves. However, reverse-time migration is computationally expensive, which limits its usability in practical imaging problems. Nevertheless,  despite its large computational cost, reverse-time migration is gaining popularity.
\item {\bf Wavefield extrapolation migration} is also employed in imaging steeply dipping reflectors, despite the intrinsic dip limitation of typical downward continuation operators. However, these techniques have been modified in various ways to allow for imaging of overturning energy. For example, \cite{GEO57-11-14531462} and \cite{Zhang.eageab.2006} employ a succession of downward/upward continuation; \cite{GEO59-05-08010809} and \cite{SEG-2004-09690972} use tilted coordinates to bring the direction of extrapolation closer to the direction of wave propagation; \cite{SEG-2003-09770980} use extrapolation along beams to achieve even tighter proximity of the directions of extrapolation and wave propagation; \cite{SavaFomel.geo.rwe} use one-way extrapolation in general (Riemannian) coordinate systems.
\end{itemize}
\par
This paper concentrates on using Riemannian wavefield extrapolation (RWE) for imaging reflectors with high dip angles. The basic characteristics of RWE recommend it as a good candidate for imaging of steeply-dipping reflectors: 
like a Kirchhoff technique, the (overturning) Riemannian coordinate system allows extrapolation of waves along their natural direction of propagation;
like a wavefield extrapolation technique, RWE allows for extrapolation of all branches of the wavefield, thus making used of all multiple-paths of extrapolated wavefields. 
Extrapolation in Cartesian coordinates, including tilted coordinates, and extrapolation along beams, represent special cases of RWE for particular choices of the coordinate system. Coordinate systems for RWE can be constructed by ray tracing or by other approaches based on alternative criteria, e.g. conformal maps with topography \cite[]{ShraggeSava.segab.2005}.
\par
This paper demonstrates the applicability of RWE to the problem of imaging steeply-dipping reflectors, in particular (overhanging) salt flanks. In addition to accurate implementation of extrapolation, a challenge for RWE is represented by the construction of the coordinate system that is appropriate for imaging of particular reflectors. Thus, a large fraction of this paper is dedicated to coordinate-system construction methods.
\par
This paper is organized as follows: we begin with a brief review of Riemannian wavefield extrpaolation, then describe alternatives for the construction of coordinate systems supporting RWE, and demonstrate the technique with applications to imaging of overhanging salt flank for synthetic salt modeled data.