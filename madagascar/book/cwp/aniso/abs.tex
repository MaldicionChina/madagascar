\begin{abstract}
The familiar hyperbolic approximation of $P$-wave reflection
moveout is exact for homogeneous isotropic or elliptically anisotropic
media above a planar reflector.
%only if the reflector is a plane, and the
%medium is homogeneous and isotropic (or elliptically
%anisotropic). 
% -- I still disagree with your correction. There is infinite number of
%    inhomogeneous media (isotropic or elliptically anisotropic), where
%    the moveout is exactly hyperbolic.
Any realistic combination of heterogeneity, reflector curvature, and nonelliptic 
anisotropy will cause departures from hyperbolic moveout at large
offsets. Here, we analyze the similarities and differences in
the influence of those three factors on $P$-wave reflection traveltimes.
%the situations where anisotropy is
%coupled with one of the other two effects. 
Using the weak-anisotropy approximation
%assumption 
for velocities in transversely isotropic media with a vertical symmetry axis
(VTI model), we show that although the nonhyperbolic moveout due to both 
vertical 
heterogeneity and reflector curvature
%Both the case of vertical heterogeneity
%and the case of a curved reflector 
can be interpreted in terms of effective anisotropy, such anisotropy
is inherently different from that of a generic homogeneous VTI model.
%though their anisotropic effects are inherently
%different from the effect of a homogeneous transversely isotropic
%model.
\end{abstract}

%%% Local Variables: 
%%% mode: latex
%%% TeX-master: "CORabs"
%%% End: 
