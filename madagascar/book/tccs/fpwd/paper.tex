\published{Geophysics, 78, no. 1, V1-V9, (2013)}

\title{Accelerated plane-wave destruction}
\righthead{Accelerated plane-wave destruction}
\lefthead{Chen, Fomel \& Lu}

\address{
\footnotemark[1] Department of Automation,\\
State Key Laboratory of Intelligent Technology and Systems\\
Tsinghua National Laboratory for Information Science and Technology \\
Tsinghua University \\
Beijing, China. 100084 \\
Email: zhonghuanchen@gmail.com. \\
\footnotemark[2]Bureau of Economic Geology, \\
Jackson School of Geosciences \\
The University of Texas at Austin \\
University Station, Box X \\
Austin, TX 78713-8924\\
Email: sergey.fomel@beg.utexas.edu. 
}

\author{Zhonghuan Chen\footnotemark[1],
 Sergey Fomel\footnotemark[2], and Wenkai Lu\footnotemark[1]}

\maketitle

\begin{abstract}
When plane-wave destruction (PWD) is implemented by implicit finite differences,
the local slope is estimated by an iterative algorithm.
We propose an analytical estimator of the local slope 
that is based on convergence analysis of the iterative algorithm. 
Using the analytical estimator, 
we design a noniterative method 
to estimate slopes by a three-point PWD filter.
Compared with the iterative estimation, 
the proposed method needs only one regularization step,
which reduces computation time significantly.
With directional decoupling of the plane-wave filter,
the proposed algorithm is also applicable to 3D slope estimation.
We present both synthetic and field
experiments to demonstrate that the proposed algorithm can yield
a correct estimation result with shorter computational time.
\end{abstract}

\section{Introduction}

Local slope fields have been widely used
in geophysical applications,
such as wave-field separation and denoising \cite[]{harlan:1869,fomel:U89},
antialiased seismic interpolation \cite[]{GPR:GPR343},
seislet transform \cite[]{fomel:V25},
velocity-independent NMO correction and imaging \cite[]{fomel:S139,cooke:WCA65},
predictive painting \cite[]{fomel:A25},
seismic attribute analysis \cite[]{marfurt:104}, etc.

Several tools exist for local slope estimation:
local slant stack \cite[]{ottolini1983signal,harlan:1869},
complex trace analysis \cite[]{barnes:264},
multiwindow dip search \cite[]{marfurt:P29},
local structure tensor \cite[]{fehmers:1286,halelocal},
and plane-wave destruction \cite[]{claerbout1992earth,fomel:1946}.
Plane-wave destruction (PWD) approximates 
the local wave-field by a local plane wave,
and models it using a linear differential equation \cite[]{claerbout1992earth}.

When plane-wave destruction is applied on discrete sampled seismic signals, 
the corresponding differential equation needs to be discretized by finite differences.
\cite{claerbout1992earth} used explicit finite differences.
In this method, plane-wave approximation of the wavefield can be seen as 
applying a linear finite impulse reponse (FIR) filter to the wavefield.
Slope estimation is equivalent to estimating a parameter of the FIR filter.
A least-squares estimator of the local slope can be obtained 
by minimizing the prediction error of the filter.
To improve estimation performance of the explicit finite difference scheme,
\cite{schleicher:P25} proposed total least-squares estimation.

The implicit finite difference scheme was applied to 
the differential equation by \cite{fomel:1946}.
%With the rational polynomial approximation of the phase-shift operator,
%the plane-wave destructor becomes an infinite impulse response (IIR) filter,
Using an infinite impulse response (IIR) filter,
known as the \textit{Thiran allpass filter} 
\cite[]{thiran1971recursive},
to approximate the phase-shift operator,
the plane-wave destruction equation becomes
a nonlinear equation of the slope.
An iterative algorithm was designed to estimate the slope.
In order to improve stability in the iterative algorithm,
a smoothing regularization \cite[]{fomel:R29} of the increment 
can be applied at each iteration.
Iterations of regularization can be time consuming, 
however, particularly in the 3D case.

In this paper, we prove the fact that
the plane-wave destruction equation is
a polynomial equation of an unkown slope.
In the case of a three-point approximation of \textit{Thiran's filter},
the convergence results
of the iterative algorithm can be analytically analyzed.
In this case, we obtain an analytical estimator of the local slope
and show that the smoothing regularization can be applied on 
the final estimator only once.
This approach reduces the computational time significantly.
We present both 2D and 3D examples, 
which demonstrate that the proposed algorithm can 
obtain a slope-estimation result faster than the iterative algorithm,
with a similar or even better accuracy.

\section{Theory}

\inputdir{Gnuplot}

\multiplot{3}{iter0,iter1,iter2}{width=0.3\textwidth}
{
The iterative results of Newton's algorithm.
The initial point is $p_0=0$ (solid circle), 
and the convergence result is marked by a blank circle:
(a) when $\discriminant\leq 0$,
(b) when $\discriminant>0$ and $a_1a_2>0$,
(c) when $\discriminant>0$ and $a_1a_2<0$.
}


\subsection{Review of PWD}
The local plane wave can be represented by the following differential equation
\cite[]{claerbout1992earth}:
\begin{equation}
\frac{\partial u}{\partial x}+\sigma\frac{\partial u}{\partial t} =0\;,
\end{equation}
where $\sigma$ is the local slope in continuous space, 
with dimension time/length.
The wavefields observed at the two positions $x_1,x_2$ have a time delay
which is proportional to their distance, $\sigma|x_1-x_2|$.
In the sampled system with space and time intervals $\Delta x$ and $\Delta t$,
we define the discrete space slope in the unit of $\Delta t/\Delta x$,
as $p=\sigma\Delta x/\Delta t$.
As $p$ is independent of the sampling interval, 
it can be directly used in irregular dataset
(in this case, the unit of the slopes becomes space variant).
The time delay between two adjacent positions is then the slope $p\Delta t$:
\begin{equation}
u(x,t)=u(x+\Delta x, t+p\Delta t)\;.
\end{equation}

With the $Z$ transform applied along both time and space directions, 
the above equation becomes
\begin{equation}\label{eq:pwd0}
(1-Z_xZ_t^p)U(Z_x,Z_t)=0\;,
\end{equation}
where $Z_t$ is the unit time-shift operator,
$Z_x$ denotes the unit space-shift operator
and $U(Z_x,Z_t)$ is the $Z$ transform of $u(x,t)$.
The operator $1-Z_xZ_t^p$ is the plane-wave destructor.
Using \textit{Thiran's fractional delay filter} 
$H(Z_t)=\displaystyle{\frac{B(1/Z_t)}{B(Z_t)}}$
\cite[]{thiran1971recursive}
to approximate the time-shift operator $Z_t^p=e^{j\omega p}$,
where $\omega$ is the circular frequency,
the plane-wave destructor can be expressed as 
\cite[]{fomel:1946},
\begin{equation}\label{eq:pwd}
C(p)=B(Z_t)-Z_xB(\frac{1}{Z_t}),
\end{equation}
where
\begin{equation}
B(Z_t)=\sum_{k=-N}^N b_k(p) Z_t^{-k},
\end{equation}
$N$ is the order of the noncausal temporal filter
and $b_k(p)$ are functions of the local slope $p$.

Equation \ref{eq:pwd} is a 2D filter.
Applying the filter at an arbitrary point in the wavefield,
the plane-wave destruction equation \ref{eq:pwd0} becomes 
a nonlinear equation for the local slope $p$:
\begin{equation}\label{eq:nonlinear}
C(p, Z_x, Z_t)U(Z_x,Z_t) \approx 0\;.
\end{equation}

An iterative method, such as Newton's method,
can be applied to find the slope.
In practice, wavefields are polluted by noise
and the plane wave assumption may not hold true 
where faults and conflicting boundaries exist.
To obtain a stable slope estimation,
an additional smoothing regularization process \cite[]{fomel:R29} 
is needed at each step.
The total computational cost of slope estimation by 
plane-wave destruction becomes $O(N_dN_fN_lN_n)$,
where $N_d$ is the size of the data, 
$N_f=2N+1$ is the size of the filter,
$N_l$ is the number of linear iterations for regularization,
and $N_n$ is the number of nonlinear iterations 
for solving equation \ref{eq:nonlinear}.
Typical values are $N_f=3,5$, $N_l=10$-50, and $N_n=5$-10.



\subsection{Accelerated PWD}

Gauss-Newton's iteration searches the solutions for nonlinear equation 
\ref{eq:nonlinear} as follows:
Let $p_k$ be the estimated slope at step $k$,
with estimating error (or destructive error) $e_k=C(p_k)U(Z_x,Z_t)$.
In order to find the correct solution $p_{k+1}$ that minimizes $e_{k+1} $,
we need to find the increment $\Delta p_k$ from the local linearization:
\begin{equation}\label{eq:error}
e_{k+1} = C(p_k)U(Z_x,Z_t)+C'(p_k)U(Z_x,Z_t)\Delta p_k \approx 0\;,
\end{equation} 
where $C'(p_k)$ is the derivative of $C(p)$ at $p_k$ 
with respect to $p$.

The iterative algorithm stops when a stationary point
or a root of $C(p)U$ is reached. They are:
\begin{enumerate}
\item 
Points where $C'(p_k)U=0$:
When $p_k$ satisfies $C'(p)U=0$, then $e_{k+1}=e_k$,
and the $\Delta p_k$ dependency in equation \ref{eq:error} is removed,
stopping further iterations on $p_k$.
\item 
Points where $C(p_k)U=0$ and $C'(p)U\neq 0$:
In this case, $\Delta p_k=0$, thus $p_{k+1}=p_k$,
eliminating the need for further improvements on $p_k$.
\end{enumerate}


The iterative algorithm for equation \ref{eq:nonlinear}
may converge at different points,
depending on the initial point that we chose;
$p_0=0$ is a common practical choice for the initial solution.
In this case, 
the iterative algorithm may converge to the least absolute root, 
which denotes the event with smallest dip angle.

In order to analyze the convergence results,
the maximally flat fractional delay filter 
\cite[]{thiran1971recursive,zhang2009maxflat}
is designed with polynomial coefficients:
\begin{equation}\label{eq:coef}
b_k(p)=
\frac{(2N)!(2N)!}{(4N)!(N+k)!(N-k)!}
\prod_{m=0}^{N-1-k}(m-2N+p)
\prod_{m=0}^{N-1+k}(m-2N-p).
\end{equation}
Details on how to design the filter can be found in the Appendix.


Since $b_k(p)$ is a polynomial of $p$,
expanding it, we get $b_k(p)=\displaystyle{\sum_{i=0}^{2N}c_{ki}p^i}$ and 
\begin{equation}
B(Z_t,p)=\sum_{k=-N}^N\sum_{i=0}^{2N}c_{ki}Z_t^{-k}p^i\;.
\end{equation}
From equation \ref{eq:coef}, it is obvious that $b_k(p)=b_{-k}(-p)$, 
therefore
$c_{ki}=(-1)^ic_{-k,i}$ and
\begin{equation}
B(Z_t,p)=B(\frac{1}{Z_t},-p).
\end{equation}

Substituting the above two equations,
the nonlinear equation \ref{eq:nonlinear} 
becomes a $2N$-th degree polynomial equation for $p$:
\begin{equation}\label{eq:pwd:poly}
\sum_{i=0}^{2N}a_ip^i=0,
\end{equation}
and the coefficients of the polynomial plane-wave destruction
can be expressed as
\begin{equation}
a_i=[1-(-1)^iZ_x]\sum_{k=-N}^Nc_{ki}Z_t^{-k}U,
\end{equation}
which says that the coefficients of the polynomial PWD
can be obtained by applying a 2D filter on the wavefield $u$.
Moreover, the 2D filter can be decoupled into the cascade of
two 1D directional filters:
the temporal filter $\displaystyle{\sum_{k=-N}^Nc_{ki}Z_t^{-k}}$ 
and the spatial filter $1-(-1)^iZ_x$.

In the special case of $N=1$,
we get a three-point approximation of $B(Z_t)$.
It takes the following form \cite[]{fomel:1946}: 
\begin{equation}
B(Z_t) = \frac{(1+p)(2+p)}{12}Z_t^{-1}+\frac{(2+p)(2-p)}{6}
+\frac{(1-p)(2-p)}{12}Z_t\;.
\end{equation}
The plane-wave destruction equation \ref{eq:pwd:poly} is a quadratic equation. 
The coefficients $a_i(i=0,1,2)$ can be solved for and expressed as
\begin{equation}
a_i=\frac{1}{12}[1-(-1)^iZ_x]v_i\;,
\end{equation}
where $v_i$ are outputs of the following three-point temporal filters:
\begin{eqnarray}
\label{eq:temp30}
v_0 &=& 2(Z_t^{-1}+4+Z_t)U\;, \\
v_1 &=& 3(Z_t-Z_t^{-1})U\;, \\
\label{eq:temp32}
v_2 &=& (Z_t^{-1}-2+Z_t)U\;.
\end{eqnarray}


In this case, 
the quadratic plane-wave destruction equation \ref{eq:pwd:poly} 
has one stationary point and two roots, 
which can be analytically expressed as:
$\left\{
 \frac{-a_1}{2a_2},\quad
 \frac{-a_1\pm\sqrt{\discriminant}}{2a_2}
\right\}
$, where $\discriminant=a_1^2-4a_0a_2$.




The plots in Figure \ref{fig:iter0,iter1,iter2} show the convergence process
of the iterative algorithm when we choose $p_0=0$ as the starting value.
Geometrically when $\discriminant\leq 0$, 
the iteration converges to the stationary point $\displaystyle{\frac{-a_1}{2a_2}}$,
as shown in Figure \ref{fig:iter0,iter1,iter2}a.
When $\discriminant >0$, 
it converges to the least absolute solution of equation \ref{eq:pwd:poly}.
Figure \ref{fig:iter0,iter1,iter2}b and \ref{fig:iter0,iter1,iter2}c 
shows the convergence process 
to the least absolute solution in different cases.
We can summarize the convergence result of the iterative algorithm as follows:
\begin{equation}\label{eq:slope}
p=\left\{\begin{array}{ll}
\displaystyle{\frac{-a_1}{2a_2}}  & \discriminant\leq 0\\
\displaystyle{\frac{-2a_0}{a_1-\sqrt{\discriminant}}} \qquad 
& \discriminant >0, ~a_1<0\;.\\
\displaystyle{\frac{-2a_0}{a_1+\sqrt{\discriminant}}}  
& \discriminant >0, ~a_1>0\\
\end{array}\right.
\end{equation}

As $\displaystyle{\frac{-2a_0}{a_1\pm\sqrt{\discriminant}}=
\frac{-a_1\pm\sqrt{\discriminant}}{2a_2}}$
in the above equation, we use 
$\displaystyle{\frac{-2a_0}{a_1\pm\sqrt{\discriminant}}}$
instead of 
$\displaystyle{\frac{-a_1\pm\sqrt{\discriminant}}{2a_2}}$
to obtain better numerical stability.



When the data is polluted by noise,
in order to obtain a robust slope estimation,
we can combine the equations in a local window into the following equation set:
\begin{equation}
\mtx F \vct p \approx \vct g,
\end{equation}
where $\mtx F$ is a normalized diagonal matrix and $\vct g$ is a vector.
Their elements are denuminators and numerators of 
equation \ref{eq:slope} respectively.
When we are solving the above equation set by least squares,
we can use Tikhonov's regularization \cite[]{fomel:1946} or
the shaping regularization \cite[~equation 13]{fomel:R29}
to obtain a smooth solution as follows
\begin{equation}
\vct p = \mtx H
[\mtx I+\mtx H^T (\mtx F^T\mtx F-\mtx I)\mtx H]^{-1}
\mtx H^T\mtx F^T \vct g\;,
\end{equation}
where $\mtx H$ is an appropriate smoothing operator.
In this case, the regularization runs only once,
therefore the computational cost is reduced to $O(N_dN_fN_l)$.

In 3D applications, there are two polynomial PWD equations 
for inline and crossline slopes separately.
Note that, using the decoupling,
inline and crossline slope estimations can share
the temporal filtering results in equations \ref{eq:temp30}$-$\ref{eq:temp32}.
We can obtain the coefficients of the crossline 
plane-wave destruction equation as
\begin{equation}
a_i=\frac{1}{12}[1-(-1)^iZ_y]v_i
\end{equation}

The five-point or longer approximations of \mbox{$B(Z_t)$} 
can achieve higher accuracy.
Equation \ref{eq:nonlinear} in this case becomes 
a higher-order polynomial equation (see details in the Appendix),
which can be solved numerically.
However, there are multiple stationary points, 
and it is difficult to determine the right one analytically.
For applications that need five-point or higher accuracy,
we suggest obtaining an initial slope estimation by
the proposed three-point method and 
using it to make the iterative algorithm converge faster 
(to decrease \mbox{$N_n$}).

\section{Examples}

\subsection{Synthetic examples}
\inputdir{const}
\plot{modl}{width=0.45\textwidth}
{Harmonic waves with constant slope $p_0=0.3$.}

To test the performance of the proposed slope-estimation method,
we generated a harmonic wave field with constant slope $p_0=0.3$
shown in Figure \ref{fig:modl}.
We added different scales of additive white Gaussian noise (AWGN) 
to the wave field and estimate the slope by the proposed method.
To compare with the iterative algorithm by \cite{fomel:1946},
the mean square error (MSE) is used as the criterion:
\begin{equation}
\textrm{MSE}(p)=\textrm E\{(p-p_0)^2\}.
\end{equation}

\multiplot{2}{mse-fdip,mse-dip}{width=0.46\textwidth}
{Mean square errors of of the slope estimations by 
the proposed method (a) and the three-point ($N=1$) iterative method (b).
}
\plot{runtime}{width=0.46\textwidth}
{
Run time of the proposed method (solid line)
and the three-point ($N=1$) iterative method (dash line).
}

We use five iterations in the iterative method, $N_n=5$.
For constant slope model, using a large smoothing window,
the smoothing regularization can converge faster (with less $N_l$)
and we can obtain a better estimation accuracy.
For each methods, we try the smoothing windows from 2 to 200
and show the mean square errors in Figure \ref{fig:mse-fdip,mse-dip}.
Compared with the iterative method,
the proposed method has better accuracy 
at the left upper (low SNR and small smoothing window)
and worse accuracy at right bottom corner (high SNR and large smoothing window).

We show the total runtime of all the noise scale data in Figure \ref{fig:runtime}.
For all smoothing windows in the regularization,
the proposed method (solid line) only uses about one fifth run time
of the three-point ($N=1$) iterative method (dash line).


A more complicated model 
from \cite{claerbout1999geophysical} and \cite{fomel:1946}
is shown in Figure \ref{fig:sigmoid,fdip,dip,pwd-fdip}a.
It has variable slopes in synclines, anticlines, and faults.
The slope estimated by the proposed method is shown in 
Figure \ref{fig:sigmoid,fdip,dip,pwd-fdip}b.
The estimation takes about 20 ms.
The three-point ($N=1$) iterative method 
can obtain a similar estimation (shown in Figure \ref{fig:sigmoid,fdip,dip,pwd-fdip}c)
by five iterations, which takes about 130ms.
Both methods use a 4-point smoothing window in the regularization,
but the proposed method obtains a slightly smoother estimation.
In Figure \ref{fig:sigmoid,fdip,dip,pwd-fdip}d we show the faults 
detected by the residuals of the proposed plane-wave destruction.

\inputdir{sigmoid}
\multiplot{2}{sigmoid,fdip,dip,pwd-fdip}{height=0.23\textheight}{
2D slope estimation example:
(a) synthetic data,
(b) slope field estimated by the proposed algorithm,
(c) slope field estimated by the 
iterative algorithm after five iterations,
(d) faults detection by plane-wave destruction with the estimated slope.
}



\subsection{Application}


\inputdir{teapot}

The 2D seislet transform \cite[]{fomel:V25}
uses local slopes to predict and update even and odd traces 
in the wavelet lifting scheme.
The seislet transform itself is fast, 
but the slope estimation step is comparatively slow.
The 3D seislet transform can be constructed in the same way by using 3D slopes 
and cascading 2D transforms in inline and crossline directions.
So that an efficient transform can be built, 
the proposed accelerated plane-wave destruction is applied to slope estimation.

Figure \ref{fig:cuber} shows a part of the Teapot Dome image from Wyoming.
The 3D slope estimation yields two slope fields along two space directions.
Figure \ref{fig:fdip1,fdip2} shows the local inline 
and crossline slope fields estimated by the proposed algorithm.

The two slopes are used by the lifting scheme 
in the seislet transform to obtain the seislet transform coefficients.
The coefficients of the 2D seislet transform 
are concentrated at the planes near the zero inline plane,
as shown in Figure \ref{fig:fseis1,fseis2}a,
whereas in Figure \ref{fig:fseis1,fseis2}b, 
the 3D transform coefficients are concentrated 
in the corner region near the origin.

\plot{cuber}{width=0.6\textwidth}
{A portion of the 3D data from the Teapot dataset.}

To illustrate the compressive performance of the 3D seislet transform,
Figure \ref{fig:fapprox2,fapprox1} shows the reconstruction results 
at different percentages of the compression.
Most of the data can be reconstructed using only $1\%$ of 
the seislet coefficients (1:100 compression ratio).
In order to compare with the iterative methods quantitatively,
we define the following normalized cross-correlation (NCC)
\begin{equation}
NCC(\vct x, \vct y) = \frac{\vct x^T\vct y}{\|\vct x\|_2\|\vct y\|_2}.
\end{equation}


\multiplot{2}{fdip1,fdip2}{width=0.46\textwidth}{
3D slopes estimated by the proposed algorithm:
(a) inline slope,
(b) crossline slope.
}

\multiplot{2}{fseis1,fseis2}{width=0.46\textwidth}{
Coefficients of:
(a) the 2D seislet transform along inline direction only,
(b) the 3D seislet transform.
}

\multiplot{2}{fapprox2,fapprox1}{width=0.46\textwidth}{
Reconstruction of 3D-seislet-compressed data using:
(a) $5\%$ of the seislet coefficients,
(b) $1\%$ of the seislet coefficients.
}



In the seislet transform, 
the more accurate dip we use, the better compression we can obtain.
The NCC between the reconstructed data and the original data
can be used to quantify the compressive performance.
In Table \ref{tbl:seislet}, the proposed method 
is compared with both three-point ($N=1$) and five-point ($N=2$) iterative methods.

In all the three methods, we use ten-point smoothing windows 
in both inline and crossline dimensions,
and use five-point smoothing window in time direction.
In order to obtain a similar slope estimation as the proposed method,
the three-point ($N=1$) and five-point ($N=2$) iterative mathods 
need about six and five iterations respectively. 
The five-point method can obtain a better compression ratio than the three-point method, 
because of its better accuracy in slope estimation.
However, although the iterative methods have smaller PWD residuals,
neighter of them achieves a better NCC than the proposed method.
In this case, using the noniterative estimation as the initial 
in five-point estimation can save at least 250s.
That is to say,
the computational time cost is reduced by a factor of about 6.

\tabl{seislet}{Performance of different dip estimation methods 
in 3D seislet transform:
Runtime is the run time of the slope estimation process;
RES-inline is the inline residual;
RES-xline is the crossline residual;
NCC-1 is the normalized cross-correlation between the original data
and the reconstruction using one percent of the seislet coefficients;
NCC-5 uses five percents of the coefficients.
The run time for 2D and 3D seislet transform are about 22.45~s and 45.0~s respectively.
}{
\begin{center}
\begin{tabular}{c|c|c|c|c|c|c|c}
\hline
Method & $N$ & Iterations & Runtime (s) & RES-inline & RES-xline & NCC-1 & NCC-5
\\ \hline
noniterative & 1 & 0 & 
   54.63 & 0.2314 & 0.2242 & 0.8858 & 0.9618 \\
iterative & 1 & 6 & 
   334.5 & 0.2267 & 0.2172 & 0.8814 &  0.961 \\
iterative & 2 & 5 & 
   301.6 & 0.2194 & 0.2075 & 0.8838 & 0.9615
\\ \hline
\end{tabular}
\end{center}
}



\section{Conclusions}

In this paper, we derived an analytical estimator of the local slope
in three-point implicit plane-wave destruction.
On the basis of this result, we built an accelerated slope estimation algorithm.
Examples show that the proposed method can produce  a result that is 
similar to that of the iterative algorithm,
at a reduced computation time.


Two or more conflicting slopes can be estimated simultaneously by the iterative algorithm.
In this case, the PWD equation is a multi-variable polynomial,
and the convergence analysis becomes complicated.
The proposed noniterative method is not yet suitable for multi-slope estimation.
However, we believe that it can find many applications in situations 
where one dominant slope is sufficient.


\section{Acknowledgments}

We thank Alexander Klokov for his helpful technical suggestions.
We also thank the reviewers and the editors for 
their careful review of the paper.

The Teapot Dome data is provided by the Rocky Mountain Oilfield Testing Center,
sponsored by the U.S. Department of Energy.
Zhonghuan Chen's joint research is sponsored by China Scholarship Council
and China State Key Science and Technology Projects (2011ZX05023-005-007).

\appendix
\section{Appendix A: Polynomial form of PWD}

If all the coefficients of $B(Z_t)$ are polynomials of $p$,
equation \ref{eq:pwd} is also a polynomial of $p$,
and the plane-wave destruction equation becomes 
in turn a polynomial equation of $p$.
The problem is to design a $2N+1$ points filter $B(Z_t)$ 
with polynomial coefficients
such that the allpass system $H(Z_t)=\frac{B(1/Z_t)}{B(Z_t)}$ can approximate 
the phase-shift operator $Z_t^p=e^{j\omega p}$.
Denoting the phase response of the system as $\theta(\omega)$,
that is $H(e^{j\omega})=e^{j\theta(\omega)}$,
the group delay of the system is
\begin{equation}
\tau(\omega)=\frac{\partial \theta(\omega)}{\partial \omega}.
\end{equation}
The maximally flat criteria designs a filter 
with a smoothest phase response. 
There are $2N$ unknown coefficients in $H(Z_t)$,
so we can add $2N$ flat constraints for the first $2N-$th order deviratives
of the phase response. 
It becomes
\cite[~equation 7]{zhang2009maxflat}
\begin{equation}
\left\{\begin{array}{l}
\tau(\omega)=p \\
\displaystyle{\frac{\partial^n\tau(\omega)}{\partial \omega^n}}=0
\qquad n=1,2,\dots,2N
\end{array}\right.,
\end{equation}
which is equivalent to the following linear maximally flat conditions
\cite[]{thiran1971recursive}:
\begin{equation}\label{eq:mf:cond}
\sum_{k=-N}^N (d-k)^{2n+1}b_k =0,
\end{equation}
where $n=0,1,\dots,2N-1$ and 
$d=p/2$ is the fractional delay of $B(1/Z_t)$ or $1/B(Z_t)$.

In order to solve $b_k$ from the above equations,
\cite{thiran1971recursive} used an additional condition $b_0=1$,
which leads $b_k(k\neq 0)$ to be a fractional function of $p$.
Differently from that, we use the following condition,
\begin{equation}\label{eq:pc:cond}
\sum_{k=-N}^N b_k =1,
\end{equation}
where $b_k$ can be proved to be polynomials of $p$.

Let vector $\vct b=[b_0,b_N,\dots,b_1,b_{-1},\dots,b_{-N}]^\textrm T$.
Combining equations \ref{eq:mf:cond} and \ref{eq:pc:cond},
we rewrite them into the following matrix form:
\[
\left[\begin{array}{c|cccccc}
1 & 1 & \dots & 1 & 1 & \dots & 1 \\ \hline
d & d-N &\dots &d-1 & d+1 &\dots & d+N \\
d^3 & (d-N)^3 & \dots & (d-1)^3 & (d+1)^3 &\dots & (d+N)^3 \\
\vdots & \vdots &\dots &\dots &\dots &\dots &\vdots \\
d^{4N-1} & (d-N)^{4N-1} & \dots & (d-1)^{4N-1} 
& (d+1)^{4N-1} &\dots & (d+N)^{4N-1}
\end{array}\right]\vct b=
\left[\begin{array}{c}
1 \\ 0 \\ 0 \\ \vdots \\ 0
\end{array}\right].
\]

The matrix on the left side, denoted as $\mtx V$, 
can be split into four blocks
$\left[\begin{array}{c|c}
\mtx A & \mtx B \\ \hline  \mtx C & \mtx D
\end{array}\right]$ 
as shown above.
Following the \textit{lemma of matrix inversion},
\begin{equation}
\mtx V^{-1}=\left[\begin{array}{cc}
(\mtx A-\mtx B\mtx D^{-1}\mtx C)^{-1} &
-(\mtx A-\mtx B\mtx D^{-1}\mtx C)^{-1}\mtx B\mtx D^{-1} \\
-\mtx D^{-1}\mtx C(\mtx A-\mtx B\mtx D^{-1}\mtx C)^{-1} &
\mtx D^{-1}+\mtx D^{-1}(\mtx A-\mtx B\mtx D^{-1}\mtx C)^{-1}\mtx B\mtx D^{-1} \\
\end{array}\right],
\end{equation}
therefore the coefficients
\begin{equation}\label{eq:sltn}
\vct b=\mtx V^{-1}[1,0,\dots,0]^\textrm T=
\left[\begin{array}{c}
(\mtx A-\mtx B\mtx D^{-1}\mtx C)^{-1} \\
-\mtx D^{-1}\mtx C(\mtx A-\mtx B\mtx D^{-1}\mtx C)^{-1} 
\end{array}\right].
\end{equation}

Let subindex $i=-N,-N+1,\dots,-1,1,2,\dots,N$ and $x_i=d+i$.
Submatrix $\mtx D$ can be expressed as
\begin{eqnarray*}
\mtx D &=& \mtx E\mtx X \\
&=&
\left[\begin{array}{cccccc}
1 & \dots &1 &1 & \dots & 1 \\ 
(d-N)^2 & \dots & (d-1)^2 &(d+1)^2 & \dots & (d+N)^2 \\
(d-N)^4 & \dots & (d-1)^4 &(d+1)^4 & \dots & (d+N)^4 \\
\vdots & \dots & \vdots  & \vdots &\dots &\vdots \\
(d-N)^{4N-2} & \dots & (d-1)^{4N-2} & (d+1)^{4N-2} & \dots & (d+N)^{4N-2}
\end{array}\right]\textrm{diag}
\left[\begin{array}{c}
x_{-N}\\ \vdots \\ x_{-1} \\ x_1 \\ \vdots \\ x_N
\end{array}\right],
\end{eqnarray*}
so $\mtx D^{-1}=\mtx X^{-1}\mtx E^{-1}$.
Denoting $\mtx U=\mtx E^{-1}$ with elements $u_{ij},~j=1,2,\dots,2N$,
as $\mtx E$ is a Vandermonde matrix,
$u_{ij}$ and Lagrange intepolating polynomials have the following relationship:
\begin{equation}\label{eq:vand:inv}
\sum_{j=1}^{2N}u_{ij}x^{2j-2}=\ell_i(x),
\end{equation}\
where $i=-N,\dots,-1,1,\dots,N$, 
and $\ell_i(x)$ is the Lagrange polynomial related to the basis $d+i$,
\begin{equation}\label{eq:lag}
\ell_i(x)=
\prod_{-N\leq m\leq N}^{m\neq i,m\neq 0}
\frac{x^2-(d+m)^2}{(d+i)^2-(d+m)^2}.
\end{equation}

Substituting the above equation, $u_{ij}$ and $x$ 
into equation \ref{eq:vand:inv},
we can prove equation \ref{eq:vand:inv}.
It follows that
\begin{equation}
[\mtx E^{-1}\mtx C]_i=d\ell_i(d),
\end{equation}
\begin{equation}
[\mtx D^{-1}\mtx C]_i=
[\mtx X^{-1}\mtx E^{-1}\mtx C]_i=
\frac{d}{d+i}\ell_i(d),
\end{equation}
with
\begin{equation}
\ell_i(d)=\frac{(-1)^{i+1}N!N!}{(N+i)!(N-i)!}
\frac{d+i}{d}
\prod_{m=-N}^N \frac{2d+m}{2d+m+i}.
\end{equation}

Thus hence
\begin{eqnarray}
\mtx A-\mtx B\mtx D^{-1}\mtx C &=&
\sum_{i=-N}^N
(-1)^{i}\frac{N!N!}{(N+i)!(N-i)!}
\prod_{m=-N}^N \frac{p+m}{p+m+i}
\nonumber \\ &=&
\frac{(4N)!N!N!}
{(2N)!(2N)!}
\frac{1}{
\displaystyle{\prod_{m=N+1}^{2N}(m^2-p^2)}}
\end{eqnarray}
and
\begin{equation}
[\mtx A-\mtx B\mtx D^{-1}\mtx C]^{-1} =
\frac{(2N)!(2N)!}{(4N)!N!N!}
\prod_{m=N+1}^{2N}(m^2-p^2).
\end{equation}

It is the coefficient $b_0$, a $2N$-th degree polynomial of $p$.
Substituting it into equation \ref{eq:sltn},
the coefficients at $k=\pm 1,\pm 2,\dots \pm N$ are expressed as
\begin{eqnarray}
b_k &= &
-[\mtx D^{-1}\mtx C]_k
[\mtx A-\mtx B\mtx D^{-1}\mtx C]^{-1}
\nonumber \\ &=&
\frac{(2N)!(2N)!}{(4N)!(N+k)!(N-k)!}
\prod_{m=0}^{N-1-k}(m-2N+p)
\prod_{m=0}^{N-1+k}(m-2N-p).
\end{eqnarray}

With the additional condition \ref{eq:pc:cond} in $2N+1$ points approximation,
all the coefficients are polynomials of $p$ of $2N$-th degree.
Thus the plane-wave destruction equation \ref{eq:nonlinear} 
therefore is proved to be a polynomial equation of $2N$-th degree.

\bibliographystyle{seg}
\bibliography{fpwd}



