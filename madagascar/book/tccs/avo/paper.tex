\published{Geophysical Prospecting, v. 57, 311-321 (2009)}
\title{Velocity analysis using $AB$ semblance}
\author{Sergey Fomel}

\address{Bureau of Economic Geology, \\
John A. and Katherine G. Jackson School of Geosciences \\
The University of Texas at Austin \\
University Station, Box X \\
Austin, TX 78713-8972}

\lefthead{Fomel}
\righthead{Velocity analysis using $AB$ semblance}

%\ms{GP-2008-0571}

\maketitle

\begin{abstract}
  I derive and analyze an explicit formula for a generalized semblance
  attribute, which is suitable for velocity analysis of prestack
  seismic gathers with distinct amplitude trends. While the
  conventional semblance can be interpreted as squared correlation
  with a constant, the $AB$ semblance is defined as a correlation with a
  trend.  This measure is particularly attractive for analyzing class
  II AVO anomalies and converted waves. Analytical derivations and
  numerical experiments show that the resolution of the $AB$ semblance
  is approximately twice lower than that of the conventional
  semblance. However, this does not prevent it from being an effective
  attribute. I use synthetic and field data examples to demonstrate
  the improvements in velocity analysis from $AB$ semblance.
\end{abstract}

\section{Introduction}

Since its introduction by \cite{GEO34-06-08590881}, the semblance
measure has been an indispensable tool for velocity analysis of
seismic records. Conventional velocity analysis of seismic gathers
scans different values of effective moveout velocity, computes
semblance of flattened gathers and generates velocity spectra for
later velocity picking \cite[]{yilmaz} .

While effective in most practical situations, semblance becomes
troublesome in the case of strong variation of amplitudes along
seismic events \cite[]{GEO66-04-12841293}. A particular example is
class II AVO anomalies \cite[]{GEO54-06-06800688} that cause seismic
amplitudes to go through a polarity reversal. To address this
problem, \cite{SEG-2000-02320235} and
\cite{GEO66-04-12841293,GEO67-05-16641672} developed 
algorithms for correcting the semblance measurement for amplitude
variations.

In this paper, I interpret the semblance attribute as a correlation
with a constant and derive an explicit mathematical expression for the
measure which corresponds to correlation with an amplitude trend. This
measure is equivalent to \emph{$AB$ semblance} \old{defined}
\new{proposed} by
\cite{GEO66-04-12841293,GEO67-05-16641672}. It reduces, in the case of
constant amplitudes, to the conventional semblance. I analyze the
statistics of the $AB$ semblance attribute and quantify the loss of
resolution associated with it. Numerical experiments with synthetic
and field data demonstrate the effectiveness of the $AB$ semblance as a
robust velocity analysis attribute, which is applicable even in the
presence of strong amplitude variations and polarity
reversals. Moreover, the ratio of the $AB$ and conventional semblances
serves as a useful AVO indicator attribute.

\section{Theory}

I start by interpreting the meaning of the conventional semblance
attribute as a correlation with a constant. Next, I define $AB$
semblance as a correlation with a trend and analyze its statistical
properties.

\subsection{Semblance as correlation}

The correlation coefficient $\gamma$ between two sequences of numbers
$\mathbf{a}=a_1,a_2,\ldots,a_N$ and $\mathbf{b}=b_1,b_2,\ldots,b_N$
is defined as
\begin{equation}
\label{eq:alpha}
\gamma(\mathbf{a},\mathbf{b}) = 
\frac{\mathbf{a} \cdot \mathbf{b}}{|\mathbf{a}|\,|\mathbf{b}|} =
\frac{\displaystyle \sum_{i=1}^N a_i\,b_i}{
\displaystyle \sqrt{\sum_{i=1}^{N} a_i^2}\,\sqrt{\sum_{i=1}^{N} b_i^2}}
\end{equation}
The correlation coefficient is analogous to the cosine
\old{measurement} \new{of the angle} between two vectors \new{$\mathbf{a}$ and $\mathbf{b}$}. It takes
values in the range from $-1$ to $1$.  Taking a correlation of a
sequence \new{$\mathbf{a}$} with a constant sequence
$\mathbf{c}=C,C,\ldots,C$ produces a measure \new{$\beta$, defined as}
\begin{equation}
\label{eq:beta}
\beta(\mathbf{a}) = \gamma(\mathbf{a},\mathbf{c}) =
\frac{\displaystyle \sum_{i=1}^N a_i\,C}{
\displaystyle \sqrt{\sum_{i=1}^{N} a_i^2}\,\sqrt{\sum_{i=1}^{N} C^2}}
= \frac{\displaystyle \sum_{i=1}^N a_i}{
\displaystyle \sqrt{N\,\sum_{i=1}^{N} a_i^2}}
\end{equation}
Squaring the correlation with a constant yields the measure equivalent
to semblance 
\begin{equation}
\label{eq:semb}
\beta^2(\mathbf{a}) = 
\frac{\displaystyle \left(\sum_{i=1}^N a_i\right)^2}{
\displaystyle N\,\sum_{i=1}^{N} a_i^2}\;.
\end{equation}
Semblance is maximized when the sequence $\mathbf{a}$ has a uniform
distribution. When seismic amplitude is uniformly distributed along a
moveout curve, the semblance of a horizontal slice through the gather
will be maximized when the event is flattened. This fact is the basis
of the conventional velocity analysis originally developed by
\cite{GEO34-06-08590881}. The approach fails, however, when the 
amplitude variation is distinctly non-uniform.

\subsection{$AB$ semblance: correlation with a trend}

Suppose that the reference sequence has a trend $b_i = A + B\,\phi_i$,
where $\phi_i$ is a known function. The trend can be, for example, an
expression of the $PP$ reflection coefficient in Shuey's approximation
\cite[]{GEO50-04-06090614}, where $A$ and $B$ are the AVO intercept
and gradient, $\phi_i=\sin^2{\theta_i}$, and $\theta_i$ corresponds to
the reflection angle at trace $i$. \new{In examples of this
paper, I use offset instead of angle. Relating offset and reflection
angle can be done either by using approximate equations of by ray
tracing once the velocity model is established.}


Estimating $A$ and $B$ from least-square fitting of the trend amounts
to the minimization of
\begin{equation}
\label{eq:ls}
F(A,B) = \sum_{i=1}^N \left(a_i - A - B\,\phi_i\right)^2
\end{equation}
Differentiating equation~(\ref{eq:ls}) with respect to $A$ and $B$,
setting the derivatives to zero, and solving the system of two linear
equations produces the well-known linear fit equations
\begin{eqnarray}
\label{eq:a} 
A & = & \frac{\displaystyle \sum_{i=1}^N \phi_i\,\sum_{i=1}^N a_i\,\phi_i -
\sum_{i=1}^N \phi_i^2\,\sum_{i=1}^N a_i}{\displaystyle 
\left(\sum_{i=1}^N \phi_i\right)^2 - N\,\sum_{i=1}^N \phi_i^2}\;, \\
\label{eq:b}
B & = & \frac{\displaystyle \sum_{i=1}^N \phi_i\,\sum_{i=1}^N a_i -
N\,\sum_{i=1}^N a_i\,\phi_i}{\displaystyle 
\left(\sum_{i=1}^N \phi_i\right)^2 - N\,\sum_{i=1}^N \phi_i^2}\;.
\end{eqnarray}
Substituting the trend $b_i = A + B\,\phi_i$ with $A$ and $B$ defined
from the least-squares equations~(\ref{eq:a}) and~(\ref{eq:b}) into the
correlation coefficient equation~(\ref{eq:alpha}) and squaring the
result leads to	equation 
\begin{equation}
\label{eq:semb2}
\boxed{
\alpha^2(\mathbf{a}) = 
\frac{\displaystyle 2\,\sum_{i=1}^{N} a_i\,\sum_{i=1}^N \phi_i\,
\sum_{i=1}^N a_i\,\phi_i - \left(\sum_{i=1}^{N} a_i\right)^2\,
\sum_{i=1}^N \phi_i^2 -	 N\,\left(\sum_{i=1}^N a_i\,\phi_i\right)^2}
{\displaystyle \sum_{i=1}^{N} a_i^2\,
\left[\left(\sum_{i=1}^N \phi_i\right)^2 - N\,\sum_{i=1}^N \phi_i^2\right]}\;.}
\end{equation}
Equation~(\ref{eq:semb2}) generalizes the semblance measure
\new{$\beta$} defined in equation~(\ref{eq:semb}) \new{to a new measure $\alpha$}. In the absence of
a trend (when the numerator in equation~(\ref{eq:b}) is zero), \old{it
simply reduces to equation~(\ref{eq:semb})} \new{$\alpha$ is
equivalent to $\beta$}.

\cite{GEO66-04-12841293} defined semblance using a 
normalized least-squares objective
\begin{equation}
\label{eq:sarkar}
\alpha^2(\mathbf{a}) = 1 - \frac{F(A,B)}{\displaystyle \sum_{i=1}^{N} a_i^2}\;.
\end{equation}
Substituting equations~(\ref{eq:a}) and~(\ref{eq:b}) into~(\ref{eq:sarkar})
is an alternative way of deriving equation~(\ref{eq:semb2}). This is \new{the}
\emph{$AB$ semblance} in terminology of
\cite{GEO66-04-12841293,GEO67-05-16641672}.

\subsection{Sensitivity analysis of $AB$ semblance}

\cite{GEO66-04-12841293,GEO67-05-16641672} observed a decrease of resolution 
in $AB$ semblance in comparison with the conventional
semblance when applying it on synthetic examples. The explicit
equation~(\ref{eq:semb2}) allows us to access the resolution limits of
both methods by measuring their effect on random noise.

As shown in Appendix A, the semblance of a sequence of uncorrelated
normally-distributed zero-mean noise samples has the expectation value
of $1/N$, where $N$ is the number of samples. The corresponding
$AB$ semblance has the expectation $2/N$ or twice
higher. Moreover, while the standard deviation of the conventional
semblance decreases as $1/N$, the deviation of the $AB$
semblance decreases as $2/N$. This analysis shows that the
$AB$ measurement is generally twice as sensitive to noise
and has a twice lower resolution. This is the price one has to pay for
the ability to handle amplitude trends.

\section{Examples}

In this section, I demonstrate the behavior of $AB$ semblance
with a synthetic and a field data example.

\subsection{Synthetic example} 

\inputdir{avo}

\plot{cmp}{width=\textwidth}{Synthetic CMP gathers. 
a: no AVO, b: AVO trend with polarity reversal in the middle section.} 

Figure~\ref{fig:cmp} shows two synthetic CMP gathers generated by
applying inverse normal moveout with a variable moveout velocity and
adding a modest amount of random noise. The first gather contains no
amplitude variations, while the second gather contains a region of
polarity reversal. Figure~\ref{fig:scn} shows velocity analysis panels
using conventional semblance scans and the corresponding automatic
velocity picks. The picking algorithm is explained in Appendix~B. We
can observe that the AVO anomaly causes a signal loss in the semblance
measure, which in turn leads to inaccurate velocity picking. The $AB$
semblance, on the other hand, is not affected much by the amplitude
variations and allows for accurate velocity picking in both cases, as
shown in Figure~\ref{fig:avoscn}.

\plot{scn}{width=\textwidth}{Conventional semblance scans for CMP gathers from
 Figure~\ref{fig:cmp}. Black curves indicate automatic velocity picks.
 The loss of signal in the right plot is caused by the AVO anomaly. a: no AVO,
b: AVO trend with polarity reversal in the middle section.}
 
\plot{avoscn}{width=\textwidth}{$AB$ semblance scans for CMP 
gathers from Figure~\ref{fig:cmp}. Black curves indicate automatic
velocity picks. Semblance remains strong despite the AVO anomaly. a:
no AVO, b: AVO trend with polarity reversal in the middle
section. Compare with Figure~\ref{fig:scn}. }

Figures~\ref{fig:nmo} and~\ref{fig:avonmo} compare NMO-corrected
gathers using velocities picked from the conventional and $AB$
semblance scans respectively. While Figures~\ref{fig:nmo}a
and~\ref{fig:avonmo}a are virtually identical, the residual curvature
evident in the anomalous region in Figure~\ref{fig:nmo}b disappears in
Figure~\ref{fig:avonmo}b, which clearly demonstrates the advantage of
the $AB$ approach. Figure~\ref{fig:attr} shows NMO-corrected
gathers colored according to the AVO-indicator attribute, which I
define as a ratio between the conventional and the $AB$
semblances.  The anomalous region is clearly visible in the attribute
display.

\plot{nmo}{width=\textwidth}{NMO-corrected CMP gathers from
 Figure~\ref{fig:cmp} using velocity functions picked from
 conventional semblance in Figure~\ref{fig:scn}.}
\plot{avonmo}{width=\textwidth}{NMO-corrected CMP gathers from
 Figure~\ref{fig:cmp} using velocity functions picked from
 $AB$ semblance in Figure~\ref{fig:avoscn}.}
\plot{attr}{width=\textwidth}{NMO-corrected CMP gathers from
 Figure~\ref{fig:avonmo} colored according to the AVO indication attribute.}

\subsection{Field data example}

For a field data example, I select a gather already processed by a
seismic contractor. The gather, shown in Figure~\ref{fig:gath4},
exhibits a clear polarity reversal around 3.8~s. The polarity
reversal is the apparent cause of a visible residual moveout
artifact. In order to correct the residual curvature, I apply
semblance-based analysis. The comparison between the conventional and
the $AB$ semblance is shown in
Figure~\ref{fig:scan4}. Similarly to the synthetic example, the
$AB$ semblance provides a better indicator of the residual
velocity for the curved event with anomalous
amplitude. Figure~\ref{fig:nmo4} shows NMO-corrected gather using a
velocity trend picked automatically from the $AB$
analysis. The curved reflection event is successfully flattened.

\inputdir{avo2}

\plot{gath4}{width=\textwidth}{Input CMP gather after preprocessing.
  a: trace display, b: wiggle display.}
\plot{scan4}{width=\textwidth}{Residual moveout curvature scans using
  conventional semblance (a) and $AB$ semblance (b).  Black
  curves indicate automatically picked trends.}
\plot{nmo4}{width=\textwidth}{CMP gather from Figure~\ref{fig:gath4}
  after residual normal moveout correction using trends picked from
  the $AB$ semblance in Figure~\ref{fig:scan4}. a: trace
  display colored according to the AVO indicator attribute, b: wiggle
  display.}

\section{Conclusions}

If one interprets the conventional semblance as a correlation with a
constant, the $AB$ semblance is a correlation with an amplitude
trend. I have derived an explicit expression for the $AB$ semblance
and analyzed it to quantify the observable loss of resolution. I have
demonstrated the advantages of the $AB$ semblance attribute using
synthetic and field data examples. The ratio of the $AB$ and
conventional semblances serves as a useful AVO indicator
attribute. \new{The sensitivity of this attribute to non-linear
variations in the amplitude trend can be a subject of further
research.}  Further applications may \new{also} include velocity
analysis of converted waves and seismic diffractions.

\section{Acknowledgments}

I would like to thank Tury Taner and Richard Uden for inspiring
discussions and for providing the field data example.

This publication is authorized by the Director, Bureau of Economic
Geology, The University of Texas at Austin.

\appendix
\section{Apendix A: Statistical analysis of semblance measures}

In this appendix, I study the influence of noise on semblance
measures. Let us assume that the signal
$\mathbf{a}=a_1,a_2,\ldots,a_N$ is composed of random independent
\old{components} \new{samples} normally distributed with zero mean and $\sigma^2$
variance. In this case, the mathematical expectation for the semblance
measure~(\ref{eq:semb}) is
\begin{equation}
\label{eq:mean}
E\left[\beta^2(\mathbf{a})\right] = 
\frac{\displaystyle E\left[\left(\sum_{i=1}^N a_i\right)^2\right]}
{\displaystyle E\left[N\,\sum_{i=1}^{N} a_i^2\right]} =
\frac{\displaystyle \sum_{i=1}^{N} E\left[a_i^2\right]}
{\displaystyle N\,\sum_{i=1}^{N} E\left[a_i^2\right]}
 = \frac{1}{N}\;.
\end{equation}
Correspondingly, the variance of the noise semblance is
\begin{eqnarray}
V\left[\beta^2(\mathbf{a})\right] & = & 
\frac{\displaystyle E\left[\left(\sum_{i=1}^N a_i\right)^4\right]}
{\displaystyle E\left[\left(N\,\sum_{i=1}^{N} a_i^2\right)^2\right]} -
\frac{1}{N^2} =
\frac{\displaystyle \sum_{i=1}^{N} E\left[a_i^4\right] +
3\,\sum_{i=1}^{N} \sum_{j \ne i} E\left[a_i^2\,a_j^2\right]}
{\displaystyle N^2\,\left(\sum_{i=1}^{N} E\left[a_i^4\right] +
\sum_{i=1}^{N} \sum_{j \ne i} E\left[a_i^2\,a_j^2\right]\right)} -
\frac{1}{N^2} \nonumber \\ 
& = &
\frac{3\,N\,\sigma^4 + 3\,(N^2-N)\,\sigma^4}
{N^2\,\left[3\,N\,\sigma^4 + (N^2-N)\,\sigma^4\right]} - 
\frac{1}{N^2} = \frac{2\,(N-1)}{N^2\,(N+2)}\;.
\label{eq:vari}
\end{eqnarray}
Equations~(\ref{eq:mean}) and~(\ref{eq:vari}) show that both the
mathematical expectation and the standard deviation (the square root
of variance) of the random noise semblance decrease at the rate of
$1/N$ with the increase in the number of traces. To derive these
equations, I make an assumption that the terms in the numerator and
denominator are statistically independent. Rather than proving this
assumption mathematically, I test it by numerical experiments with
multiple random number realizations. Figure~\ref{fig:mean,vari}
compares the theoretical prediction with experimental measurements
from 10,000 random realizations.

Applying similar analysis to the $AB$
semblance~(\ref{eq:semb2}), we deduce that
\begin{eqnarray}
\nonumber
E\left[\alpha^2(\mathbf{a})\right] & = & \frac{\displaystyle E\left[2\,\sum_{i=1}^{N} a_i\,\sum_{i=1}^N \phi_i\,
\sum_{i=1}^N a_i\,\phi_i - \left(\sum_{i=1}^{N} a_i\right)^2\,
\sum_{i=1}^N \phi_i^2 -	 N\,\left(\sum_{i=1}^N a_i\,\phi_i\right)^2\right]}
{\displaystyle E\left[\sum_{i=1}^{N} a_i^2\right]\,
\left[\left(\sum_{i=1}^N \phi_i\right)^2 - N\,\sum_{i=1}^N \phi_i^2\right]} \\ & = &
\frac{\displaystyle 2\,E\left[a_i^2\right]\,\left[\left(\sum_{i=1}^N \phi_i\right)^2 - N\,\sum_{i=1}^N \phi_i^2\right]}
{\displaystyle N\,\,E\left[a_i^2\right]\,\left[\left(\sum_{i=1}^N \phi_i\right)^2 - N\,\sum_{i=1}^N \phi_i^2\right]}
 = \frac{2}{N}\;.
\label{eq:amean}
\end{eqnarray}
and
\begin{eqnarray}
\nonumber
V\left[\alpha^2(\mathbf{a})\right] & = & 
\frac{\displaystyle E\left[\left\{2\,\sum_{i=1}^{N} a_i\,\sum_{i=1}^N \phi_i\,
\sum_{i=1}^N a_i\,\phi_i - \left(\sum_{i=1}^{N} a_i\right)^2\,
\sum_{i=1}^N \phi_i^2 -	 N\,\left(\sum_{i=1}^N a_i\,\phi_i\right)^2\right\}^2\right]}
{\displaystyle E\left[\left(\sum_{i=1}^{N} a_i^2\right)^2\right]\,
\left[\left(\sum_{i=1}^N \phi_i\right)^2 - N\,\sum_{i=1}^N \phi_i^2\right]^2} - \frac{4}{N^2} \\ 
\nonumber
& = &
\frac{\displaystyle 2\,\left[2\,\left(\sum_{i=1}^N \phi_i\right)^2 + N\,\sum_{i=1}^N \phi_i^2\right]^2 +
6\,N^2\,\left(\sum_{i=1}^N \phi_i^2\right)^2 - 24\,N\,\left(\sum_{i=1}^N \phi_i\right)^2\,\sum_{i=1}^N \phi_i^2}
{\displaystyle \left[3\,N + (N^2-N)\right]\,\left[\left(\sum_{i=1}^N \phi_i\right)^2 - N\,\sum_{i=1}^N \phi_i^2\right]^2} - \frac{4}{N^2} \\
& = & \frac{4\,(N-2)}{N^2\,(N+2)}\;.
\label{eq:avari}
\end{eqnarray}
One can see that, in the case of the $AB$ semblance, the
mathematical expectation and the standard deviation of the random
noise semblance decrease at the rate of $2/N$, twice higher than that
for the conventional semblance. Figure~\ref{fig:amean,avari} compares
the theoretical prediction with experimental measurements.

\inputdir{stat}

\multiplot{2}{mean,vari}{width=0.45\textwidth}{Mathematical expectation (a) 
and standard deviation (b) of random-noise semblance as functions of
the number of traces $N$. Solid lines are theoretical curves, circles
are measurements from a numerical experiment.}

\multiplot{2}{amean,avari}{width=0.45\textwidth}{Mathematical expectation (a) and standard deviation (b) of 
random-noise $AB$ semblance as a function of the number of
traces $N$. Solid lines are theoretical curves, circles are
measurements from a numerical experiment.}

\appendix
\section{Appendix B: Automatic velocity picking from semblance scans}

The problem of automatic picking of velocities from semblance scans
has been considered by many authors
\cite[]{SEG-1999-11621165,SEG-2003-20882091,SEG-2004-16271629}. The
approach taken in this paper is inspired by the suggestion of
\cite{harlan} to look at velocity picking as a variational
problem. According to \cite{harlan}, an optimally picked velocity trend
$v(t)$ in the semblance field $\alpha(t,v)$ corresponds to the maximum
of the variational integral
\begin{equation}
P_1[v(t)] = \int\limits_{t_{min}}^{t_{max}} \alpha\left(t,v(t)\right)\,d t\;.
\end{equation}

I take the variational formulation further by considering its
analogy to the ray tracing problem. The first-arrival seismic
ray is a trajectory corresponding to the minimum traveltime. The
trajectory corresponding to an optimal velocity trend should minimize
an analogous measure defined in the space of the velocity
scan~$\{t,v\}$. I use the variational measure
\begin{equation}
P_2[v(t)] = \int\limits_{t_{min}}^{t_{max}}
\exp[-\alpha\left(t,v(t)\right)]\,\sqrt{\lambda^2+\left[v'(t)\right]^2}\,d t\;.
\end{equation}
where $\lambda$ is a scaling parameter. According to \old{the} variational theory
\cite[]{lanc}, an optimal trajectory can be determined by solving the eikonal equation
\begin{equation}
\label{eq:eik}
\left(\frac{\partial T}{\partial v}\right)^2 + 
\frac{1}{\lambda^2}\,\left(\frac{\partial T}{\partial t}\right)^2 = 
\exp[-2\,\alpha\left(t,v\right)]
\end{equation}
with a finite-difference algorithm. The quantity in the right hand
side of equation~(\ref{eq:eik}) plays the role of squared slowness.
Small slowness corresponds to high semblance and attracts ray
trajectories in a ``wave guide''. After obtaining a finite-difference
solution, the picking trajectory \old{is} \new{can be} extracted by
tracking \new{backward} along the traveltime gradient direction. An
analogous approach has been used in medical imaging in the method of
virtual endoscopy \cite[]{thomas}.  To remove random oscillations, I
smooth the picked trajectory using the method of shaping
regularization
\cite[]{shape}.

\bibliographystyle{seg}
\bibliography{SEG,avo}
