\published{Geophysical Prospecting,  61, 526-536 (2013)}
\renewcommand{\thefootnote}{\fnsymbol{footnote}}

\title{Seismic wave extrapolation using lowrank symbol approximation}
\author{Sergey Fomel\footnotemark[1], Lexing Ying\footnotemark[2], and Xiaolei Song\footnotemark[1]}

\ms{GP-2011-1032.R1}

\lefthead{Fomel, Ying, \& Song}
\righthead{Lowrank wave extrapolation}

\address{
\footnotemark[1]Bureau of Economic Geology, \\
John A. and Katherine G. Jackson School of Geosciences \\
The University of Texas at Austin \\
University Station, Box X \\
Austin, TX 78713-8972 \\
USA \\
sergey.fomel@beg.utexas.edu \\
\footnotemark[2]Department of Mathematics \\
The University of Texas at Austin \\
1 University Station \\
Austin, TX 78712 \\
USA \\
lexing@math.utexas.edu
}

\maketitle

\newpage

\begin{abstract}
We consider the problem of constructing a wave extrapolation operator
in a variable and possibly anisotropic medium. Our construction
involves Fourier transforms in space combined with the help of a
lowrank approximation of the space-wavenumber wave-propagator
matrix. A lowrank approximation implies selecting a small set of
representative spatial locations and a small set of representative
wavenumbers. We present a mathematical derivation of this method, a
description of the lowrank approximation algorithm, and numerical
examples which confirm the validity of the proposed approach. Wave
extrapolation using lowrank approximation can be applied to seismic
imaging by reverse-time migration in 3D heterogeneous isotropic or
anisotropic media.
\end{abstract}

\section{INTRODUCTION}

Wave extrapolation in time plays an important role in seismic imaging
(reverse-time migration), modeling, and full waveform
inversion. Conventionally, extrapolation in time is performed by
finite-difference methods \cite[]{Etgen.sep.48.133}. Spectral methods
\cite[]{GPR35-05-04790490,GEO53-09-11751183} have started to 
gain attention recently and to become feasible in large-scale 3-D
applications thanks to the increase in computing power. The
attraction of spectral methods is in their superb accuracy and, in
particular, in their ability to suppress dispersion artifacts
\cite[]{chunlei,etgen}.

Theoretically, the problem of wave extrapolation in time can be
reduced to analyzing numerical approximations to the mixed-domain
space-wavenumber operator \cite[]{wards}. In this paper, we propose a
systematic approach to designing wave extrapolation operators by
approximating the space-wavenumb\-er matrix symbol with a lowrank
decomposition. A lowrank approximation implies selecting a small set
of representative spatial locations and a small set of representative
wavenumbers. The optimized separable approximation or OSA
\cite[]{song} was previously employed for wave extrapolation 
\cite[]{zhang,du} and can be considered as another form of lowrank decomposition. However, the
decomposition algorithm in OSA is significantly more expensive,
especially for anisotropic wave propagation, because it involves
eigenfunctions rather than rows and columns of the original
extrapolation matrix. Our algorithm can also be regarded as an
extension of the wavefield interpolation algorithm of \cite{etgen},
with optimally selected reference velocities and weights. Another
related method is the Fourier finite-difference (FFD) method proposed
recently by \cite{song2010b}. FFD may have an advantage in efficiency, because
it uses only one pair of multidimensional forward and inverse
FFTs (fast Fourier transforms) per time step. However, it does not
offer flexible controls on the approximation accuracy.

Our approach to wave extrapolation is general and can apply to
different types of waves, including both acoustic and elastic seismic
waves, as well as velocity continuation \cite[]{GEO68-05-16501661},
offset continuation \cite[]{GEO68-02-07180732}, prestack exploding
reflector extrapolation \cite[]{perm}, etc.

The paper is organized as follows. We first present the theory behind
the proposed algorithm, then describe the algorithm and test its
accuracy on a number of synthetic benchmark examples of increasing
complexity.

\section{WAVE EXTRAPOLATION}

\newcommand{\x}{\mathbf{x}}
\renewcommand{\k}{\mathbf{k}}
\newcommand{\eps}{\varepsilon}

Let $P(\x,t)$ be the seismic wavefield at
location~$\x$ and time~$t$. The wavefield at the next time
step $t+\Delta t$ can be approximated by the following mixed-domain
operator \cite[]{wards}
\begin{equation}
  \label{eq:exp}
  P(\x,t+\Delta t) = \int
  \widehat{P}(\k,t)\,e^{i\,\phi(\x,\k,\Delta t)}\,d\k\;,
\end{equation}
where $\widehat{P}(\k,t)$ is the spatial Fourier transform of $P(\x,t)$
\begin{equation}
  \label{eq:ft}
  \widehat{P}(\k,t) = \frac{1}{(2\,\pi)^3}\,\int P(\x,t) e^{-i\,\k \cdot \x}\,
  d\x\;,
\end{equation}
where $\k$ is the spatial wavenumber.
To define the phase function~$\phi(\x,\k,t)$, which appears in
equation~(\ref{eq:exp}), one can substitute
approximation~(\ref{eq:exp}) into the wave equation and extract the
geometrical (high-frequency) asymptotic of it. In case of seismic
wave propagation, this leads to the eikonal-like equation
\begin{equation}
  \label{eq:eiko}
  \frac{\partial \phi}{\partial t} = \pm V(\x,\k)\,\left|\nabla \phi\right|\;,
\end{equation}
where $V(\x,\k)$ is the phase velocity, and the choice of the sign
corresponds, in the case of a point source, to expanding or
contracting waves. In the isotropic case, $V$ does not depend on
$\k$. The initial condition for equation~(\ref{eq:eiko}) is
\begin{equation}
  \label{eq:ic}
  \phi(\x,\k,0) = \k \cdot \x\;,
\end{equation}
which turns equation~(\ref{eq:exp}) into the simple inverse Fourier transform operation.

Assuming small steps $\Delta t$ in equation~(\ref{eq:exp}), one can
build successive approximations for the phase function~$\phi$ by
expanding it into a Taylor series. In particular, let us represent the
phase function as
\begin{equation}
  \label{eq:taylor}
  \phi(\x,\k,t) \approx  \k \cdot \x + \phi_1(\x,\k)\,t + \phi_2(\x,\k)\,\frac{t^2}{2} + \cdots
\end{equation}
Correspondingly,
\begin{equation}
  \label{eq:grad}
  \left|\nabla \phi\right| \approx |\k| + \frac{\nabla \phi_1 \cdot
    \k}{|\k|}\,t + O(t^2)\;.
\end{equation}
Substituting expansions~(\ref{eq:taylor}) and~(\ref{eq:grad}) into equation~(\ref{eq:eiko}) and
separating terms with different powers of $t$, we find that
\begin{eqnarray}
  \label{eq:phi1}
  \phi_1(\x,\k) & = & V(\x,\k)\,|\k|\;, \\
  \label{eq:phi2}
  \phi_2(\x,\k) & = & V(\x,\k)\,\nabla V \cdot \k\;.
\end{eqnarray}

When either the velocity gradient $\nabla V$ or the time step $\Delta
t$ are small, the Taylor expansion~(\ref{eq:taylor}) can be reduced to
only two terms, which in turn reduces equation~(\ref{eq:exp}) to the
familiar expression \cite[]{etgen}
\begin{equation}
  \label{eq:expf}
  P(\x,t+\Delta t) \approx \int
  \widehat{P}(\k,t)\,e^{i\,\left[\k \cdot \x+V(\x,\k)\,|\k|\,\Delta t\right]}\,d\k\;,
\end{equation}
or
\begin{equation}
P(\x,t+\Delta t) + P(\x,t-\Delta t) \approx 
    2\,\int \widehat{P}(\k,t)\,e^{i\,\k \cdot \x}\cos\left[V(\x,\k)\,|\k|\,\Delta t\right]\,d\k\;.
\label{eq:cosf}
\end{equation}

In rough velocity models, where the gradient $\nabla V$ does not
exist, one can attempt to solve the eikonal
equation~\ref{eq:eiko} numerically or to apply approximations other than
the Taylor expansion~(\ref{eq:taylor}). In the examples of this
paper, we used only the $\phi_1$ term.

Note that the approximations that we use, starting from
equation~(\ref{eq:exp}), are focused primarily on the phase of wave
propagation. As such, they are appropriate for seismic migration but
not necessarily for accurate seismic modeling, which may require
taking account of amplitude effects caused by variable density and
other elastic phenomena.

The computational cost for a straightforward application of
equation~(\ref{eq:exp}) is $O(N_x^2)$, where $N_x$ is the total size
of the three-dimensional $\x$ grid. Even for modest-size problems,
this cost is prohibitively expensive. In the next section, we describe
an algorithm that reduces the cost to $O(M\,N_x\,\log{N_x})$, where
$M$ is a small number.

\section{LOWRANK APPROXIMATION}

The key idea of the lowrank decomposition is decomposing the
wave extrapolation matrix
\begin{equation}
\label{eq:mat}
W(\x,\k) = e^{i\,\left[\phi(\x,\k,\Delta t)-\k \cdot \x\right]}
\end{equation}
for a fixed $\Delta t$ into a separated representation
\begin{equation}
  \label{eq:lra}
  W(\x,\k) \approx \sum\limits_{m=1}^M \sum\limits_{n=1}^N W(\x,\k_m) a_{mn} W(\x_n,\k).
\end{equation}
Representation~(\ref{eq:lra}) speeds up the computation of
$P(\x,t+\Delta t)$ since
\begin{eqnarray}
\nonumber
  P(\x,t+\Delta t) & = & \int e^{i \x \k} W(\x,\k) \widehat{P}(\k,t) d
  \k \\
 & \approx & \sum\limits_{m=1}^M W(\x,\k_m) \left( \sum\limits_{n=1}^N a_{mn} \left(\int e^{i \x\k} W(\x_n,\k) \widehat{P}(\k,t) d\k \right) \right)\;.
\label{eq:pxt}
\end{eqnarray}
The evaluation of the last formula is effectively equivalent to
applying $N$ inverse Fast Fourier Transforms. Physically, a separable
lowrank approximation amounts to selecting a set of $N$ representative
spatial locations and $M$ representative wavenumbers.

In order to discuss the construction of approximation~(\ref{eq:lra}),
let us view it as a matrix decomposition problem
\begin{equation}
  \label{eq:lramatrix}
  \mathbf{W} \approx \mathbf{W}_1 \, \mathbf{A} \, \mathbf{W}_2
\end{equation} 
where $\mathbf{W}$ is the $N_x\times N_x$ matrix with entries
$W(\x,\k)$, $\mathbf{W}_1$ is the submatrix of $\mathbf{W}$ that
consists of the columns associated with $\{\k_m\}$, $\mathbf{W}_2$ is
the submatrix that consists of the rows associated with $\{\x_n\}$,
and $\mathbf{A} = \{a_{mn}\}$. In practice, we find that the
matrix $\mathbf{W}$ has a low rank separated representation provided
that $\Delta t$ is sufficiently small, which, in the case of
smooth models, can be partially explained by the separation of terms
in the Taylor series~\ref{eq:taylor}. Let $\eps$ be a prescribed
accuracy of this separated representation, and $r_\eps$ be the
numerical rank of $\mathbf{W}$. The construction of the separated
representation in equation~(\ref{eq:lramatrix}) follows the method of
%\cite{engquistying2007,engquistying2009} 
Engquist and Ying (2007, 2009) and is detailed in the appendix. The
main observation is that the columns of $\mathbf{W}_1$ and the rows of
$\mathbf{W}_2$ should span the column space and row space of
$\mathbf{W}$, respectively, as well as possible. The algorithm for
computing (\ref{eq:lramatrix}) takes the following steps:
\begin{enumerate}
\item Pick a {\em uniformly random} set $S$ of $\beta \cdot r_\eps$
  columns of $\mathbf{W}$ where $\beta$ is chosen to be 3 or 4 in
  practice. Perform the pivoted QR factorization to
  $(\mathbf{W}(:,S))^*$ \cite[]{golub}. The first $r_\eps$ pivoted
  columns correspond to $r_\eps$ rows of the matrix
  $\mathbf{W}(:,S)$. Define $\mathbf{W}_1$ to be the submatrix of
  $\mathbf{W}$ that consists of these rows and set $\x_1,\ldots,\x_N$
  with $n=r_\eps$ to be the corresponding $x$ values of these rows.
\item Pick a {\em uniformly random} set $T$ of $\beta \cdot r_\eps$
  rows of $\mathbf{W}$ and perform the pivoted QR factorization to
  $\mathbf{W}(T,:)$. Define $\mathbf{W}_2$ to be the submatrix of $\mathbf{W}$ that consists of
  these columns and set $\k_1,\ldots,\k_M$ with $m=r_\eps$ to be the
  corresponding $k$ values of these columns.
\item Set the middle matrix $\mathbf{A} = \mathbf{W}^{\dagger}(\x_n,\k_m)_{1\le n \le N, 1\le
      m \le M}$ where $\dagger$ stands for the pseudoinverse.
\item Combining the result of the previous three steps gives the
  required separated representation $\mathbf{W} \approx \mathbf{W}_1 \, \mathbf{A} \, \mathbf{W}_2$.
\end{enumerate}
The algorithm does not require, at any step, access to the full matrix
$\mathbf{W}$, only to its selected rows and columns. Once the
decomposition is complete, it can be used at every time step during
the wave extrapolation process. In multiple-core implementations, the
matrix operations in equation~(\ref{eq:lra}) are easy to
parallelize. The algorithm details are outlined in the appendix.

The cost of the algorithm is $O(M\,N_x\,\log{N_x})$ operations per time
 step, where $N_x\,\log{N_x}$ refers to the cost of the Fourier
 transform. In comparison, the cost of finite-difference wave
 extrapolation is $O(L\,N_x)$, where $L$ is the size of the
 finite-difference stencil. \cite{lfd} present an application of the proposed
 lowrank approximation algorithm for devising accurate
 finite-different schemes. There is a natural trade-off in the
 selection of $M$: larger values lead to a more accurate wave
 representation but require a longer computational time. In the
 examples of the next section, we select these parameters based on an
 estimate of the approximation accuracy and generally aiming for the
 relative accuracy of $10^{-4}$. The resulting $M$ is typically
 smaller than the number of Fourier transforms required for
 pseudo-spectral algorithms such as pseudo-spectral implementations of
 the rapid expansion method \cite[]{rem}. 

\section{EXAMPLES}

We start with a simple 1-D example. The 1-D velocity model contains a
linear increase in velocity, from 1~km/s to 2.275~km/s. The
extrapolation matrix, $2\,(\cos\left[V(x)\,|k|\,\Delta t\right]-1)$,
or pseudo-Laplacian in the terminology of \cite{etgen}, for the time
step $\Delta t=0.001\,\mbox{s}$ is plotted in
Figure~\ref{fig:prop}. Its lowrank approximation is shown in
Figure~\ref{fig:prod} and corresponds to $N=M=2$. The $x$ locations
selected by the algorithm correspond to velocities of 1.59 and
2.275~km/s. The wavenumbers selected by the algorithm correspond to
the Nyq\-uist frequency and 0.7 of the Nyquist frequency. The
approximation error is shown in Figure~\ref{fig:proderr}. The relative
error does not exceed 0.34\%. Such a small approximation error results
in accurate wave extrapolation, which is illustrated in
Figure~\ref{fig:wave2,awave2,waverr}. The extrapolated wavefield shows
a negligible error in wave amplitudes, as demonstrated in
Figure~\ref{fig:waverr}.

\inputdir{fio} 

\multiplot{3}{prop,prod,proderr}{height=0.25\textheight}{Wave 
  extrapolation matrix for 1-D wave propagation with linearly
  increasing velocity (a), its lowrank approximation (b), and
  Approximation error (c).}
\multiplot{3}{wave2,awave2,waverr}{height=0.225\textheight}{(a) 1-D wave
  extrapolation using the exact extrapolation symbol. (b) 1-D wave
  extrapolation using lowrank approximation. (c) Difference between
  (a) and (b), with the scale amplified 10 times compared to (a) and (b).}

\inputdir{impres} 

\multiplot{2}{wavefd,wave}{width=0.45\textwidth}{Wavefield snapshot in a smooth
  velocity model computed using (a) fourth-order finite-difference method
  and (b) lowrank approximation. The velocity model is 
  $v(x,z)\,=\,550+0.00015\,(x-800)^2+0.001\,(z-500)^2$. 
 The wave source is a point-source Ricker wavelet, located in the
  middle of the model. The finite-difference result exhibits
  dispersion artifacts while the result of the lowrank approximation,
  similarly to that of the FFD method, is dispersion-free.}

\multiplot{2}{slicefd,slice}{width=0.45\textwidth}{Horizontal slices through wavefield snapshots in Figure~\ref{fig:wavefd,wave}}

Our next example (Figures~\ref{fig:wavefd,wave} and
\ref{fig:slicefd,slice}) corresponds to wave extrapolation in a 2-D
smoothly variable isotropic velocity field. As shown by
\cite{song2010b}, the classic finite-difference method (second-order
in time, fourth-order in space) tends to exhibit dispersion artifacts
with the chosen model size and extrapolation step, while spectral
methods exhibit high accuracy. As yet another spectral method, the
lowrank approximation is highly accurate. The wavefield snapshot,
shown in Figures~\ref{fig:wave} and~\ref{fig:slice}, is free from
dispersion artifacts and demonstrates high accuracy. The approximation
rank decomposition in this case is $N=M=2$, with the expected error of
less than $10^{-4}$. In our implementation, the CPU time for
  finding the lowrank approximation was 2.45~s, the single-processor
  CPU time for extrapolation for 2500 time steps was 101.88~s or 2.2
  times slower than the corresponding time for the finite-difference
  extrapolation (46.11~s).

\inputdir{fowler}

\multiplot{2}{fwavefd,fwave}{width=0.45\textwidth}{Wavefield snapshot in a simple two-layer 
  velocity model using (a) fourth-order finite-difference method and
  (b) lowrank approximation. The upper-layer velocity is 1500~m/s, and
  the bottom-layer velocity is 4500~m/s.  The finite-difference result
  exhibits clearly visible dispersion artifacts while the result of
  the lowrank approximation is dispersion-free.}

To show that the same effect takes place in case of rough velocity
model, we use first a simple two-layer velocity model, similar to the
one used by \cite{fowler}. The difference between a
dispersion-infested result of the classic finite-difference method
(second-order in time, fourth-order in space) and a dispersion-free
result of the lowrank approximation is clearly visible in Figure~\ref{fig:fwavefd,fwave}. The time step
was 2~ms, which corresponded to the approximation rank of 3. In
  our implementation, the CPU time for finding the lowrank
  approximation was 2.69~s, the single-processor CPU time for
  extrapolation for 601 time steps was 19.76~s or 2.48 times slower
  than the corresponding time for the finite-difference extrapolation
  (7.97~s).  At larger time steps, the finite-difference method in
this model becomes unstable, while the lowrank method remains stable
but requires a higher rank.

\inputdir{bp}

\plot{sub}{width=\columnwidth}{Portion of BP-2004 synthetic isotropic velocity
  model.}

\plot{snap}{width=\columnwidth}{Wavefield snapshot for the velocity
  model shown in Figure~\ref{fig:sub}.}

Next, we move to isotropic wave extrapolation in a complex 2-D velocity
field. Figure~\ref{fig:sub} shows a portion of the BP velocity model
\cite[]{bp}, containing a salt body. The wavefield snapshot (shown in
Figure~\ref{fig:snap}) confirms the ability of our method to handle
complex models and sharp velocity variations. The lowrank
decomposition in this case corresponds to $N=M=3$, with the expected
error of less than $10^{-7}$. Increasing the time step size $\Delta t$
does not break the algorithm but increases the rank of the
approximation and correspondingly the number of the required Fourier
transforms. For example, increasing $\Delta t$ from 1~ms to 5~ms leads
to $N=M=5$.

\inputdir{threed}

\plot{salt}{width=\textwidth}{SEG/EAGE 3-D salt model.}
\plot{wave3}{width=\textwidth}{Snapshot of a point-source wavefield propagating in the SEG/EAGE 3-D salt model.}

Our next example is isotropic wave extrapolation in a 3-D complex
velocity field: the SEG/EAGE salt model \cite[]{seg-eage} shown in
Figure~\ref{fig:salt}. A dispersion-free wavefield snapshot is shown
in Figure~\ref{fig:wave3}. The lowrank decomposition used $N=M=2$, with
the expected error of $10^{-5}$.

\inputdir{bptti}

\multiplot{4}{vpend2,vxend2,etaend2,thetaend2}{width=0.45\textwidth}{Portion of BP-2007 anisotropic benchmark model. (a) Velocity along the axis of symmetry. (b) Velocity perpendicular to the axis of symmetry. (c) Anellipticity parameter $\eta$. (d) Tilt of the symmetry axis.}

\plot{snap4299}{width=\textwidth}{Wavefield snapshot for the velocity
  model shown in Figure~\ref{fig:vpend2,vxend2,etaend2,thetaend2}.}

Finally, we illustrate wave propagation in a complex anisotropic
model. The model is a 2007 anisotropic benchmark dataset from
BP\footnote{The dataset was created by Hemang Shah and is
    provided at \url{http://software.seg.org/} courtesy of BP Exploration Operation Company
    Limited.}. It exhibits a strong TTI (tilted transverse isotropy)
with a variable tilt of the symmetry axis
(Figure~\ref{fig:vpend2,vxend2,etaend2,thetaend2}).  A wavefield
snapshot is shown in Figure~\ref{fig:ttisnapshot}. Because of the
complexity of the wave propagation patterns, the lowrank decomposition
took $N=M=10$ in this case and required 10 FFTs per time step. In a
TTI medium, the phase velocity $V(\x,\k)$ from
equation~(\ref{eq:cosf}) can be expressed with the help of the
acoustic approximation
\cite[]{GEO63-02-06230631,GEO65-04-12391250,GPR52-03-02470259}
\begin{equation}
\label{eq:tti}
V(\x,\k) =
\sqrt{\frac{1}{2}(v_x^2\,\hat{k}_x^2+v_z^2\,\hat{k}_z^2)+\frac{1}{2}\sqrt{(v_x^2\,\hat{k}_x^2+v_z^2\,\hat{k}_z^2)^2-\frac{8\eta}{1+2\eta}v_x^2v_z^2\,\hat{k}_x^2\,\hat{k_z^2}}}\;,
\end{equation}
where $v_x$ is the P-wave phase velocity in the symmetry plane, $v_z$
is the P-wave phase velocity in the direction normal to the symmetry
plane, $\eta$ is the anellipticity parameter
\cite[]{GEO60-05-15501566}, and $\hat{k}_x$ and $\hat{k}_z$ stand for
the wavenumbers evaluated in a rotated coordinate system aligned with
the symmetry axis:
\begin{equation}
\label{eq:wavenumber}
\begin{array}{*{20}c}
\hat{k}_x=k_x\cos{\theta}+k_z\sin{\theta}\;\\ 
\hat{k}_z=k_z\cos{\theta}-k_x\sin{\theta}\;,\\ 
 \end{array}
\end{equation}
where $\theta$ is the tilt angle measured with respect to horizontal.

\section{CONCLUSIONS}

We have presented a novel algorithm for wave extrapolation in
heterogeneous and anisotropic media. The algorithm is based on a
lowrank approximation of the extrapolation symbol. It reduces the cost
of extrapolation to that of a small number of FFT operations per time
step, which correspond to the approximation rank. The algorithm
has a high, spectral accuracy. In that sense, it is comparable with a
number of other recently proposed FFT-based methods. Its advantage is
a direct control on the accuracy-efficiency trade-off by controlling
the rank of the approximation and the corresponding approximation
error. We propose to incorporate the lowrank extrapolation algorithm
in seismic imaging by reverse-time migration. 

\section{Acknowledgments}

We thank KAUST for partial financial support; Tariq Alkhalifah,
Bj\"{o}rn Engquist, Laurent Demanet, and Paul Fowler for useful
discussions; and BP for releasing benchmark synthetic models.

The reproducible computational examples in this paper use the
Madagascar open-source software package \url{http://www.ahay.org/}.

This publication is authorized by the Director, Bureau of Economic
Geology, The University of Texas at Austin.

\appendix
\section{Linear-time algorithm for a lowrank matrix approximation}
\label{sec:lowrank}

In this appendix, we outline the lowrank matrix approximation
algorithm in more details.

Let $N_x$ be the number of samples both in space and wavenumber. Let
us denote the samples in the spatial domain by
$\x=\{x_1,\ldots,x_{N_x}\}$ and the ones in the Fourier domain by
$\k=\{k_1,\ldots,k_{N_x}\}$. The elements of the interaction matrix
$\mathbf{W}$ from equation~(\ref{eq:mat}) are then defined as
\begin{equation}
W_{ij} = e^{\imath [\phi(x_i,k_j,\Delta t] - x_i \cdot k_j]},\quad
1\le i,j\le N_x.
\label{eq:wij}
\end{equation}
Here we describe an algorithm by \cite{engquistying2009} that
generates, in a time linear with respect to $N_x$, an approximate
factorization of $\mathbf{W}$ of rank $r$ in the following form
\begin{equation}
\mathbf{W} \approx \mathbf{U M V^*}\;,
\label{eq:umv}
\end{equation}
where $\mathbf{U}$ consists of $r$ selected columns from $\mathbf{W}$,
$\mathbf{M}$ is a matrix of size $r\times r$ and $\mathbf{V^*}$ consists of $r$
selected rows from $\mathbf{W}$.

The first question is: which columns of $\mathbf{W}$ shall one pick
for the matrix $\mathbf{U}$? It has been shown by \cite{Goreinov:97}
and \cite{GuEisenstat:96} that the $r$-dimensional volume spanned by
these columns should be the maximum or close to the maximum among all
possible choices of $r$ columns from $\mathbf{W}$. More precisely, suppose
$\mathbf{W} = [w_1,\ldots,w_{N_x}]$ is a column partitioning of $\mathbf{W}$. Then one aims to find
$\{j_1,\ldots,j_r\}$ such that
\begin{equation}
\{j_1,\ldots,j_r\} = \text{argmin}_{\{j_1,\ldots,j_r\}} \vol_r(w_{j_1},\ldots,w_{j_r}).
\label{eq:argmin}
\end{equation}
However, finding a set of $r$ columns with almost the maximum
$r$-dimensional volume is a computationally difficult problem due to
the following two reasons. First, the length of the vectors $N$ is
typically very large for three dimensional problems, hence
manipulating these vectors can be costly. Second, the number of the
vectors $N_x$ is also large. A exhaustive search over all possible
choices of $r$ vectors to find the one with the maximum volume is
prohibitive expensive, so one needs to find a more practical approach.

In order to overcome the problem associated with long vectors, the
first idea is to project to a lower dimensional space and search for
the set of vectors with maximum volume among the projected vectors.
However, one needs to ensure that the volume is roughly preserved
after the projection so that the set of vectors with the maximum
projected volume also has a near-maximum volume in the original space.
One of the most celebrated theorems in high dimensional geometry and
probability is the following Johnson-Lindenstrauss lemma
\cite[]{JohnsonLindenstrauss:84}.
\begin{theorem}
  Let $v_1,\ldots,v_N$ be a set of $N$ vectors in $R^d$. Let $T$ be a
  randomly generated subspace of dimension $t=O(\log N /\varepsilon^2)$ and
  use $P_T$ to denote the orthogonal projection onto $T$. Then with
  high probability,
  \[
  (1-\varepsilon) \lVert v_i-v_j \rVert \le 
  \sqrt{\frac{d}{t}}  \lVert P_T v_i-P_T v_j \rVert \le
  (1+\varepsilon) \lVert v_i-v_j \rVert
  \]
  for $1\le i,j \le N$.
\end{theorem}

This theorem essentially says that projecting to a subspace of
dimension $O(\log N)$ preserves the pairwise distance between $N$
arbitrary vectors. There is an immediate generalization of this
theorem due to \cite{Magen:2002}, formulated 
slightly differently for our purpose.

\begin{theorem}
  Let $v_1,\ldots,v_N$ be a set of $N$ vectors in $R^d$. Let $T$ be a
  randomly generated subspace of dimension $t=O(r^3 \log N /\varepsilon^2)$
  and use $P_T$ to denote the orthogonal projection onto $T$. Then
  with high probability,
  \[
  (1-\varepsilon) \cdot \vol_r(v_{i_1},\ldots,v_{i_r}) \le
  \left(\frac{d}{t}\right)^{r/2}  \vol_r(P_T v_{i_1},\ldots,P_T v_{i_r}) \le
  (1+\varepsilon) \cdot \vol_r(v_{i_1},\ldots,v_{i_r})
  \]
  for any $\{i_1,\ldots,i_r\} \subset \{1,\ldots,N\}$.
\end{theorem}

The main step of the proof is to bound the singular values of a random
matrix between $(1-\varepsilon)^{1/r}$ and $(1+\varepsilon)^{1/r}$
(after a uniform scaling) and this ensures that the $r$-dimensional
volume is preserved within a factor of $(1-\varepsilon)$ and
$(1+\varepsilon)$. In order to obtain this bound on the singular
values, we need $t$ to be $O(r^3 \log N)$. However, bounding the
singular values is only one way to bound the volume, hence it is
possible to improve the dependence of $t$ on $r$. In fact, in
practice, we observe that $t$ only needs to scale like $O(r\log N)$.

Given a generic subspace $T$ of dimension $t$, computing the
projections $P_T w_1,\ldots, P_T w_N$ takes $O(tN^2)$ steps. Recall
that our goal is to find an algorithm with linear complexity, hence
this is still too costly. In order to reduce the cost of the random
projection, the second idea of our approach is to randomly choose $t$
coordinates and then project (or restrict) each vector only to these
coordinates. This is a projection with much less randomness but one
that is much more efficient to apply. Computationally, this is
equivalent to restricting $\mathbf{W}$ to $t$ randomly selected
rows. We do not yet have a theorem regarding the volume for this
projection. However, it preserves the $r$-dimensional volume very well
for the matrix $\mathbf{W}$ and this is in fact due to the oscillatory
nature of the columns of $\mathbf{W}$. We denote the resulting vectors by
$\{\mathbf{\wt{w}}_1,\ldots,\mathbf{\wt{w}}_{N_x}\}$.

The next task is to find a set of columns $\{j_1,\ldots,j_r\}$ so
that the volume $\vol_r(\wt{w}_{j_1},\ldots,\wt{w}_{j_r})$ is
nearly maximum. As we mentioned earlier, exhaustive search is too
costly. To overcome this, the third idea is to use the following
pivoted QR algorithm (or pivoted Gram-Schmidt process) to find the $r$
columns.
\begin{algorithmic}[1]
  \FOR{$s=1,\ldots, r$}

  \STATE Find $j_s$ among $\{1,\ldots,N\} \setminus {j_1,\ldots,j_{s-1}}$ such that
  $\wt{w}_{j_s}$ has the largest norm
  \STATE Orthogonalize the vectors $\wt{w}_j$ for $j\in \{1,\ldots,N\}
  \setminus {j_1,\ldots,j_s}$ with $\wt{w}_{j_s}$ and update them
  \ENDFOR

  \STATE $\{j_1,\ldots,j_r\}$ is the column set required
\end{algorithmic}
Once the column set is found, we set $\mathbf{U} = \left[ \mathbf{w}_{j_1},\ldots,\mathbf{w}_{j_r}\right]$.

In order to identify $\mathbf{V^*}$, one needs to find a set of $r$ rows of $\mathbf{W}$
that has an almost maximum volume. To do that, we repeat the same
steps now to $\mathbf{W^*}$. More precisely, let
\begin{equation}
\mathbf{W} = 
\begin{bmatrix}
  \mathbf{m}_1 \\
  \vdots\\
  \mathbf{m}_{N_x}
\end{bmatrix}
\label{eq:w}
\end{equation}
be the row partitioning of the matrix $\mathbf{W}$. The algorithm takes the
following steps:
\begin{algorithmic}[1]
  \STATE Select uniform randomly a set of $t$ columns and obtain an
  $N_x \times t$ tall matrix

  \STATE Perform pivoted QR algorithm on the rows of this tall matrix
  and denote the first $r$ rows selected by $\{i_1,\ldots,i_r\}$
  
  \STATE The matrix $\mathbf{V^*}$ is
  \begin{equation}
  \mathbf{V^*} = 
  \begin{bmatrix}
    \mathbf{m}_{i_1} \\
    \vdots\\
    \mathbf{m}_{i_r}
  \end{bmatrix}\;.
  \label{eq:v}
  \end{equation}
\end{algorithmic}

Once both $\mathbf{U}$ and $\mathbf{V^*}$ are identified, the last task is to compute
the $r\times r$ matrix $\mathbf{M}$ for $ \mathbf{W} \approx \mathbf{U M V^*}$. Minimizing
\begin{equation}
\min_{\mathbf{M}} \lVert \mathbf{W - U M V^*} \rVert_F
\label{eq:min}
\end{equation} 
yeilds $ \mathbf{M = (U)^{\dagger} W (V^*)}^{\dagger}$ where $\dagger$ stands for the
pseudo-inverse. However, this formula requires taking matrix product
with $\mathbf{W}$, which takes $O(t\,N_x^2)$ steps. In order to achieve linear
scaling, the fourth idea of our approach is to select randomly a set
of $t$ rows $A$ and a set of $t$ columns $B$ and minimize
\begin{equation}
\min_{\mathbf{M}} \lVert \mathbf{W}(A,B) - \mathbf{U}(A,:) \, \mathbf{M} \, \mathbf{V}(B,:)^* \rVert_F\;.
\label{eq:min2}
\end{equation}
The solution for this problem is 
\begin{equation}
\mathbf{M} = (\mathbf{U}(A,:))^{\dagger} \,\mathbf{W}(A,B) \, \left(\mathbf{V}(B,:)^*\right)^{\dagger}\;.
\label{eq:m}
\end{equation}

Let us now discuss the overall cost of this algorithm. Random sampling
of $t$ rows and $t$ columns of the matrix $\mathbf{W}$ clearly takes $O(t\,N_x)$
steps. Pivoted QR factorization on the projected columns
$\{\wt{w}_1,\ldots,\wt{w}_{N_x}\}$ takes $O(t^2\,N_x)$ steps and the cost for
for the pivoted QR factorization on the projected rows. Finally,
performing pseudo-inverses takes $O(t^3)$ steps. Therefore, the
overall cost of the algorithm is $O(t\,N_x) + O(t^2\,N_x) + O(t^3) = O(t^2\,N_x)$.
As we mentioned earlier, in practice $t=O(r \log N_x)$. Hence, the
overall cost is linear in $N_x$.

\bibliographystyle{seg}
\bibliography{SEG,SEP2,wave}
