\published{Geophysics, 79, no. 3, V81-V91, (2014)}

\title{Random noise attenuation by $f$-$x$ empirical mode decomposition predictive filtering }

\renewcommand{\thefootnote}{\fnsymbol{footnote}}

\author{Yangkang Chen\footnotemark[1] and Jitao Ma\footnotemark[2]}

%\ms{GEO-2013}

\address{
\footnotemark[1]Bureau of Economic Geology \\
John A. and Katherine G. Jackson School of Geosciences \\
The University of Texas at Austin \\
University Station, Box X \\
Austin, TX 78713-8924 \\
\footnotemark[2] State Key Laboratory of Petroleum Resources and Prospecting \\
China University of Petroleum \\
Fuxue Road 18th\\
Beijing, China, 102200
}


\lefthead{Chen \& Ma}
\righthead{EMD predictive filtering}
\footer{TCCS-7}

% for multi-confusing-revise
\DeclareRobustCommand{\dlo}[1]{}
\DeclareRobustCommand{\wen}[1]{#1}%

%\dlo means remove removing
%\wen means preserve

\maketitle
\begin{abstract}

Random noise attenuation always plays an important role in seismic data processing. One of the most widely used methods \wen{for} suppressing random noise is $f-x$ predictive filtering. When the subsurface structure becomes complex, this method suffers from higher prediction errors owing to the large number of different dip components that need to be predicted\dlo{ out}. In this paper, we \old{present}\new{propose} a novel denoising method \dlo{called}\wen{termed} $f-x$ empirical mode decomposition predictive filtering (EMDPF). \dlo{The}\wen{This} new scheme solves the problem that \wen{makes} $f-x$ empirical mode decomposition (EMD) \dlo{can't deal}\wen{ineffective} with complex seismic data. \dlo{Since it makes}\wen{Also, by making} the prediction more precise, \dlo{it}\wen{the new scheme} \dlo{also }removes the limitation of conventional $f-x$ predictive filtering when dealing with multi-dip seismic profiles. In \dlo{the}\wen{this} new method, we first apply EMD to each frequency slice in the $f-x$ domain and obtain several intrinsic mode functions (IMF). Then an auto\wen{-}regressive (AR) model is applied to the sum of \wen{the} first \dlo{several}\wen{few} IMFs\wen{,} which \dlo{stand for}\wen{contain} the high-dip-angle components\wen{,} in order to predict the useful steeper events. Finally, the predicted events are added to the sum of the remaining IMFs.  This process \dlo{equals to improving}\wen{improves} the \dlo{predictive pricision}\wen{prediction precision} by utilizing an EMD based dip filter to reduce the dip components before $f-x$ predictive filtering. Both synthetic and real data sets demonstrate the performance of our proposed method in preserving more useful energy.

\end{abstract}
\section{Introduction}

Further development of exploration and production of reservoirs increases the demand
for random noise suppression.
Current random noise attenuation methods are realized in either the $t-x$ or \new{a }\dlo{some other }transformed domain \cite[]{zhipeng}. In the
 $t-x$ domain, denoising methods include stacking \cite[]{mayne,yilmaz,guochang}, polynomial fitting \cite[]{zhong,guochang1}, and median filtering \cite[]{Liu}. All these methods fully utilize the differences of both travel time and apparent velocity between signal and noise in the $t-x$ domain. In transformed domains, denoising methods include $f-x$ predictive filtering \cite[]{canales}, \wen{the} wavelet transform \cite[]{Zhang,gao}, \wen{the} curvelet transform \cite[]{neelamani}, \wen{and the} seislet transform \cite[]{yang,seislet}. \dlo{The latter}\wen{These} methods \dlo{first apply a }transform\dlo{ation of} the seismic data from \wen{the} $t-x$ domain to some other domain, where the signal and random noise can be separated. The noise is removed in the transformed domain prior to transformation back to \new{the} $t-x$ \wen{domain}.

\cite{canales} first used $f-x$ predictive filtering to attenuate random noise. Since then, continuous efforts have been \dlo{paid}\wen{made} to improve \dlo{its}\wen{the} predictive precision or \wen{to} modify the conventional version to  \dlo{meet} \old{better }\wen{meet} \new{better} the requirements set by various applications. \cite{guo} proposed $f-xy$ predictive filtering in order to  improve the adaptation to both 2D and 3D post-stack seismic data processing. \cite{su} suggested $f-xyz$ predictive filtering, which is mainly used for random noise attenuation in \wen{a} pre-stack data set. 
\cite{kang} proposed a $f-x$ quasi-linear transform method  adapted to the non-linear events of seismic data in complex regions. Unfortunately, when the subsurface is extremely complex, $f-x$ predictive filtering does not yield good  results because of the large \dlo{amount}\wen{number} of dip components that need to be predicted\dlo{ out}.

\cite{emd} proposed a new signal processing method, which uses empirical mode decomposition (EMD) to prepare stable input for \wen{the} Hilbert Transform. 
The essence of EMD is to stabilize a \dlo{non-stational}\wen{non-stationary} signal. That is, to decompose a signal into a series of intrinsic mode functions (IMF). Each IMF has a relatively local-constant frequency. The frequency of each IMF decreases according to the separation sequence of each IMF.  EMD is a breakthrough in the analysis of linear and stable spectra. It \dlo{can }adaptively separate\wen{s} non-linear and \dlo{non-stational}\wen{non-stationary} signals\wen{,} which are features of seismic data, into different frequency ranges. 
\cite{bekara} applied $f-x$ EMD to attenuation of random and coherent noise\wen{,} with good results. \cite{cai} suggested using $t-f-x$ EMD to denoise seismic data \dlo{based on}\wen{on the basis of} a mixed time-frequency analysis. \dlo{Nevertheless, the $f-x$ and $t-f-x$ domain EMD methods can only be used for random noise attenuation for NMO corrected or post-stack seismic data. With dipping events, they will suppress some of the useful energy.}\wen{Nevertheless, for the
purpose of random noise attenuation, the \dlo{f- x}\wen{$f-x$} and \dlo{t - f - x}\wen{$t-f-x$} domain EMD methods can
only be applied on NMO corrected or post-stack seismic data. With profiles containing
dipping events, \dlo{they}\wen{these methods} will suppress some of the useful energy.}

In this paper, we propose a new approach \dlo{called}\wen{, termed} \dlo{"}\emph{$f-x$ empirical mode decomposition predictive filtering\dlo{"}} (EMDPF), 
which combines both $f-x$ EMD and $f-x$ predictive filtering. \dlo{The}\wen{This} new \dlo{denoising}\wen{noise attenuation} \dlo{methology}\wen{methodology} can adapt to \wen{more} complex seismic profiles than \old{can }$f-x$ EMD\old{,} and \old{can }preserve more useful energy \wen{than} \old{can} $f-x$ predictive filtering. $f-x$ EMDPF uses an EMD based dip filter to reduce the dip components for the \dlo{following}\wen{subsequent} predictive filtering in order to improve the predictive precision.

We start this paper by reviewing the conventional $f-x$ predictive filtering theory and point\dlo{ed} out its high-prediction-error problem when the number of dip components become\wen{s} large\dlo{,}\wen{.} \dlo{t}\wen{T}hen \wen{we} \dlo{introduce}\wen{review} basic EMD theory and its application\wen{,} both in data processing and \wen{the} exploration geophysical fields\dlo{,}\wen{.} \dlo{f}\wen{F}inally\wen{,} we suggest a way to combine the properties of both $f-x$ predictive filtering and $f-x$ EMD \wen{in order} to form the new denoising algorithm, $f-x$ EMDPF. Three synthetic data sets and one real data set\dlo{s} demonstrate that $f-x$ EMDPF can preserve much more useful energy while remov\dlo{e}\wen{ing} slightly less random noise than \old{can }$f-x$ EMD and $f-x$ predictive filtering.

\section{\emph{f-x} predictive filtering}
\inputdir{./}
Let $s(t,h)(h=1,2,\cdots,H)$ be the \dlo{seismic }signal of trace $h$ and $H$ be the number of traces. If the slope of a linear event with constant amplitude in a seismic section is $\psi$, then:
\begin{equation}
\label{eq:first}
s(t,h+1)=s(t-h\psi\Delta x,1),
\end{equation}
where $\Delta x$ denotes the trace interval. Equation \ref{eq:first} can be transformed into the frequency domain \wen{in order} to give:
\begin{equation}
\label{eq:second}
S(f,h+1)=S(f,1)e^{-i2\pi fh\psi\Delta x}.
\end{equation}
For a specific frequency $f_0$,  from equation \ref{eq:second} we can  obtain a linear recursion\wen{,} which is given by:
\begin{equation}
\label{eq:third}
S(f_0,h+1)=a(f_0,1)S(f_0,h),
\end{equation}
where  $a(f_0,1)=e^{-i2\pi f_0\psi\Delta x}$. 
This recursion is a first\wen{-}order difference equation\wen{,} also known as an auto\wen{-}regressive (AR) model of order 1. Similarly, superposition of $p$ linear events in the $t-x$ domain can be represented by an AR model of order $p$ \cite[]{tufts,har} as the following equation:
\begin{equation}
\label{eq:fifth}
S(f_0,h+1)=a(f_0,1)S(f_0,h)+a(f_0,2)S(f_0,h-1)+\cdots+a(f_0,p)S(f_0,h+1-p),
\end{equation}
\dlo{\dlo{By}\wen{Using} equation \ref{eq:fifth} and \dlo{w}\wen{W}iener filtering, the predictive error filter $a(f_0,h)(h=1,2,\cdots,p)$, whose length is $p$, can be derived.}\wen{where $a(f_0,h)(h=1,2,\cdots,p)$ denotes the predictive error filter, with a length of $p$.} The prediction error energy $E(f_0)$ is given by the following equation:
\begin{equation}
\label{eq:sixth}
E(f_0)={\Arrowvert a(f_0,h) \ast S(f_0,h)-S(f_0,h+1) \Arrowvert}_2^2,
\end{equation}
where symbol $*$ denotes convolution, and $\Arrowvert\cdot\Arrowvert_2^2$ denotes the least-squares energy. By minimizing the prediction error energy $E(f_0)$, we can get the filtering operator $a(f_0,m)$. Applying this operator to the spatial trace yields the denoised results \dlo{corresponding to}\wen{for the} frequency slice $f_0$.

$f-x$ predictive filtering works perfectly on a single event. Figures  \ref{fig:syn01-flat}-\ref{fig:syn01-flat-fxdecon-noise} show and compare the denoised results for a single flat synthetic event. The denoised result (Figure \ref{fig:syn01-flat-fxdecon}) is quite \dlo{pleasant}\wen{good}\wen{,} with the random noise in Figure \ref{fig:syn01-flat} largely removed \dlo{with}\wen{and}  only \wen{a} small amount of \wen{the} useful component in the noise section, Figure \ref{fig:syn01-flat-fxdecon-noise}.  For a single dipping event, Figures \ref{fig:syn01-dip}-\ref{fig:syn01-dip-fxdecon-noise}, the results are similar. However, when the number of different dips is increased, the seismic section becomes more complex and predictive filtering is not as effective. Figure \ref{fig:syn01-complex} shows a synthetic section containing four events with differing dips. In the removed noise section, Figure \ref{fig:syn01-complex-fxdecon-noise}, there \dlo{exists}\wen{remains} a significant amount of residual useful energy. 

The synthetic data shown in Figures \ref{fig:syn01-flat}, \ref{fig:syn01-dip}\wen{,} and \ref{fig:syn01-complex} \dlo{are}\wen{were} all generated by SeismicLab \wen{\cite[]{seismiclab}}, \dlo{and the}\wen{with a} signal-to-noise ratio (SNR) of \dlo{them are all} 2.0 \wen{for all of them}. Here we define the SNR as the ratio of maximum amplitude of useful energy and the maximum amplitude of \dlo{g}\wen{G}aussian white noise. \dlo{It should be note}\wen{Note} that the same parameters were used for the predictive filters in each case shown \dlo{on}\wen{in} Figure \ref{fig:syn01-flat,syn01-flat-fxdecon,syn01-flat-fxdecon-noise,syn01-dip,syn01-dip-fxdecon,syn01-dip-fxdecon-noise,syn01-complex,syn01-complex-fxdecon,syn01-complex-fxdecon-noise,syn01-complex,syn01-complex-fxemdpf,syn01-complex-fxemdpf-noise}.

We now conclude \wen{that} the effectiveness of $f-x$ predictive filtering deteriorates as the number of different dips increases, mainly because \dlo{of the sum of}\wen{the total of} leaked useful energy increases at the same time. In particular, when the number of dips is \dlo{exetremely}\wen{extremely} large\wen{,} as \wen{occurs} with hyperbolic events, $f-x$ predictive filtering fails to achieve acceptable results. \dlo{It's}\wen{It is} natural\dlo{ly} to infer that if we can first reduce the number of dips\wen{,} or in other words pick the very steep events and total random noise out\dlo{ }, then \wen{by} \dlo{apllying}\wen{applying} the same $f-x$ predictive filtering, the predictive precision will \dlo{be improved}\wen{improve}. That is \dlo{what we will talk about in}\wen{the subject of} the \dlo{chapter of}\wen{section on} \dlo{\texttt{$f-x$ emprical mode decomposition predictive filtering}}\wen{$f-x$ empirical mode decomposition predictive filtering}.

\inputdir{fxdecon}
\multiplot{12}{syn01-flat,syn01-flat-fxdecon,syn01-flat-fxdecon-noise,syn01-dip,syn01-dip-fxdecon,syn01-dip-fxdecon-noise,syn01-complex,syn01-complex-fxdecon,syn01-complex-fxdecon-noise,syn01-complex,syn01-complex-fxemdpf,syn01-complex-fxemdpf-noise}{width=0.25\textwidth}{Demonstration of $f-x$ predictive filtering (a-i) and $f-x$ EMDPF (j-l) on synthetic section with different number of dip components. (a) Single flat event. (b) Denoised single flat event. (c) Removed noise section corresponding to (a) and (b). (d) Single dipping event. (e) Denoised single dipping event. (f) Removed noise section corresponding to (d) and (e). (g) Complex events section. (h) Denoised complex events section. (i) Removed noise section corresponding to (g) and (h). (j) Same as (g). (k) Denoised result by $f-x$ EMDPF. (l) Removed noise section corresponding to (j) and (k).}

\section{Empirical mode decomposition}
\subsection{1D EMD}
The \dlo{target}\wen{aim} of empirical mode decomposition (EMD) is to empirically decompose a \dlo{non-stational}\wen{non-stationary} signal into a finite set of subsignals, which are \dlo{called}\wen{termed} intri\wen{n}sic mode functions (IMF) and \wen{are} considered to be stable. The IMFs satisfy two conditions: (1) in the whole data set, the number of extrema and the number of zero crossings must either equal or differ at most by one; and (2) at any point, the mean value of the envelope defined by the local maxima and the envelope defined by the local minima is zero \cite[]{emd}.

Provided that $s(t)$, $c_n(t)$, $r(t)$\wen{,} and $N$ denote the original \dlo{non-stational}\wen{non-stationary} signal, the separated \dlo{Intrisic Mode Function (IMF)}\wen{IMFs}, the residual\wen{,} and the number of IMFs\wen{,} respectively\dlo{.}\wen{,} \dlo{T}\wen{t}he mathematical principle of EMD can be expressed as:
\begin{equation}
\label{eq:emd}
s(t)=\sum_{n=1}^{N}c_n(t)+r(t).
\end{equation}
For a \dlo{non-stational}\wen{non-stationary} signal $s(t)$, using equation \ref{eq:emd}, we get a finite set of subsignals $c_n(t)$,($n=1,2,\cdots,N$). 

A special property of \dlo{IMF}\wen{EMD} is that the IMFs represent\dlo{s} different osci\dlo{a}ll\wen{a}tions embed\wen{d}ed in the data, \dlo{and}\wen{where} the oscillating frequency for each subsignal $c_n(t)$ decreases as the sequence number of \wen{the} IMF become\wen{s} larger (\dlo{I}\wen{we} call it \wen{a} frequency decreasing property in the following context). This property results from the sifting algorithm \wen{used} to implement the decomposition. Appendix A gives a detailed instruction about the sifting process\dlo{. The sifting process}\wen{, which} can be summarized as a process in which low-frequency component\wen{s} \dlo{is}\wen{are} gradually removed \dlo{out }to generate a more local-constant-frequency mode, which is followed by \wen{the generation of}\dlo{generating} the next mode.

Figure \ref{fig:sigimf} gives a demonstration for a synthetic signal. The original synthetic signal is generated through $d(t)=\sin(0.2\pi t)+\sin(0.4\pi t)+\sin(0.8\pi t)$\dlo{,}\wen{;} in other words, it is constructed from three individual frequency components corresponding \wen{to} 0.1\wen{ }Hz, 0.2\wen{ }Hz and 0.4 Hz\wen{,} respectively. From \dlo{the demonstration}\wen{Figure} \ref{fig:sigimf}, we can see \wen{that,} except for small edge imprecision and \dlo{neglibible}\wen{negligible} residual, EMD successfully decompose this signal into three components\old{, which comprise three signals having an approximate frequency ratio: 4:2:1} \new{with a frequency ratio of approximately 4:2:1}.

Because of the frequency decreasing property, EMD has been used \dlo{to separate signal components in \dlo{the }field\wen{s} of signal data processing, such as noise attenuation}\wen{outside geophysics for noise attenuation} \cite[]{mao2007,kopsinis2009}. \old{Because}\new{Since} random noise \dlo{mainly }represents \wen{mainly} the high-frequency components, by removing \dlo{out }the IMFs with \wen{the} highest \wen{frequency,}\dlo{oscillation} we can \dlo{knock \dlo{known}\wen{down}}\wen{attenuate} \dlo{these}\wen{this type of} noise. However, \dlo{in the field of exploration geophysics, directly applying EMD to each time trace hasn't been utilized because of the serious mode mixing problem of EMD. Mode mixing has been defined as any IMF consisting of oscillation frequencies of dramatically disparate scales .}\wen{in exploration geophysics, applying EMD to time traces is not effective because of the mode mixing problem. \cite{kopecky2010} defined mode mixing as any IMF consisting of frequencies of dramatically disparate scales.} When mode mixing exists, the first one or two IMFs contain\dlo{s} \dlo{much}\wen{a lot of} useful \dlo{refelction}\wen{reflection} energy. \dlo{Although enhanced e}\wen{E}xtensions \wen{to EMD}\wen{,} such as ensemble empirical mode decomposition (EEMD) \cite[]{wu2009} \dlo{or}\wen{and} complete ensemble \dlo{enpirical}\wen{empirical} mode decomposition (CEEMD) \cite[]{torres2011} have been proposed to solve the mode\wen{-}mixing problem in signal processing \dlo{field }and have been used in geophysics\dlo{ field} to \dlo{analysize}\wen{analyze} time-frequency properties, \wen{but have not been used for t-x domain seismic noise attenuation.}\dlo{, a method to suppress the mode\wen{-}mixing problem in removing the seismic noise in \dlo{the} t-x domain has \dlo{been barely}\wen{not been} proposed.}
\subsection{\emph{f-x} EMD}
Instead of $t-x$ domain EMD, a $f-x$ domain EMD method to \dlo{remove out}\wen{attenuate} random noise in seismic data has been proposed by \cite{bekara}. They apply EMD on each frequency slice in the $f-x$ domain, and \dlo{remove out}\wen{suppress} the \dlo{large}\wen{higher} wavenumber components, which mainly represent\dlo{s the high-wavenumber} random noise. \wen{However,} \dlo{A}\wen{a} problem occurs when applying $f-x$ EMD, because the high-wavenumber dipping events will also be removed\dlo{ out}. \dlo{It's easy to understand that those dipping events represent the high-wavenumber energy when considering the following equation: $p=\frac{t}{v} = \frac{k}{f}$.}
\dlo{When frequency $f$ in equation \ref{eq:dipk} becomes constant, wavenumber $k$ will increase as the dip $p$ increase.} 
\wen{This problem occurs because, for many data sets, the random noise and any steeply dipping coherent energy make
a significantly larger contribution to the high-wavenumber energy in
the f-x domain than any desired signal \cite[]{bekara}.}

\cite{bekara} cleverly utilize this\dlo{ this} by-product of $f-x$ EMD to attenuate coherent noise such as ground roll\dlo{s}.

\wen{
The detailed algorithmic steps of $f-x$ EMD are given by \cite{bekara} as:
\begin{enumerate}
\item 
Select a time window and transform the data to the $f-x$ domain.
\item 
For every frequency, 
\begin{enumerate}
\item
separate real and imaginary parts in the spatial sequence,
\item
compute IMF1, for the real signal and subtract it to obtain the filtered real signal,
\item
repeat for the imaginary part,
\item
combine to create the filtered complex signal.
\end{enumerate}
\item
Transform data back to the $t-x$ domain.
\item
Repeat for the next time window.
\end{enumerate}}

$f-x$ EMD can be used as an adaptive $f-k$ filter. The cutoff wavenumber is adaptively defined \wen{and does not}\dlo{, which means it doesn't} need any \dlo{pre-knowledge}\wen{apriori knowledge} about the seismic data \wen{in order }to define the filter param\wen{e}ters. This \dlo{adaptivity}\wen{adaptability} makes $f-x$ \wen{EMD} very convenient to utilize in real application\wen{s}. The frequency-slice-dependent \dlo{adaptivity}\wen{adaptability} also makes $f-x$ EMD more precise than $f-x$ predictive filtering, because \wen{all} the filter parameters in $f-x$ predictive filtering for each frequenc\dlo{e}\wen{y} slice are \dlo{all }the same. Another advantage of $f-x$ EMD over $f-x$ predictive filtering is that \wen{the} trace spacing \dlo{need}\wen{does} not \wen{need} to be perfectly regular \dlo{becasue}\wen{because} \dlo{of }no convolutional operator is used, \dlo{which at this point is similear}\wen{a characteristic similar} to local median and \dlo{local }SVD filtering \cite[]{bekara2007,bekara}.


\inputdir{synsig}
\plot{sigimf}{width=\textwidth}{Demonstration of empirical mode decomposition on a synthetic signal. (a) The original signal, (b) first IMF, (c) second IMF, (d) third IMF, (e) residual.}

\section{\emph{f-x} empirical mode decomposition predictive filtering}

$f-x$ EMDPF utilizes the property that the first \dlo{several}\wen{few} (generally $1\sim3$) IMFs for each frequency slice in the $f-x$ domain contain\dlo{s} the high-dip-angle components and random noise\dlo{,}\wen{.} \dlo{so}\wen{Thus} the leaked dipping events can be obtained by applying a predictive filter to \dlo{the first several }\wen{these} IMFs. Adding the predicted signal to the sum of \wen{the} remaining IMFs will suppress random noise without harming the effective signals. 

$F-x$ EMDPF is a new seismic \dlo{denoising}\wen{noise attenuation} method which combines \dlo{those respective}\wen{the advantages of both} \dlo{advantages of }$f-x$ predictive filtering and $f-x$ EMD. The detailed algorithm\wen{ic} steps of $f-x$ EMDPF are \dlo{much like}\wen{similar to} $f-x$ EMD \cite[]{bekara} and are shown \old{bellow}\new{below}:
\begin{enumerate}
\item 
Select a time window and transform the data to the $f-x$ domain.
\item 
For every frequency, 
\begin{enumerate}
\item
separate real and imaginary parts in the spatial sequence,
\item
compute IMF1, for the real signal and subtract it to obtain the filtered real signal,
\item
apply an AR model to IMF1 and add the result to the sum of the remaining IMFs,
\item
repeat for the imaginary part,
\item
combine to create the filtered complex signal.
\end{enumerate}
\item
Transform data back to the $t-x$ domain.
\item
Repeat for the next time window.
\end{enumerate}

It should be emphasized that the number of the filtered IMFs is not limited to one\wen{,} but \dlo{to be decided by}\wen{is selected according to} both \wen{the} noise level and the distribution of the dip components \dlo{of a }\wen{within the} specific seismic data\wen{ }set. If the noise level is \dlo{strong}\wen{high}, \wen{then a larger} \dlo{the }number of IMFs should be chosen\dlo{ larger}, because the noise \wen{remains} not only \dlo{stay }in the first IMF but also \wen{in} the second or the third, \wen{albeit} with decreasing energy. If the dip components \dlo{is}\wen{are} mainly distributed in the \dlo{high-dip-angle}\wen{high-angle} range, then the number of IMFs could be relatively smaller, but when the dip components \dlo{is}\wen{are} distributed in the low- or \dlo{mid-dip-angle}\wen{mid-angle} range, we should choose more IMFs \wen{in order} to ensure \wen{that} \dlo{more }noise is \dlo{to be }removed \dlo{but at the same time}\wen{whilst} still preserv\dlo{e}\wen{ing} these dipping components.

Generally the number of IMFs for filtering is \dlo{proper in}\wen{within} the range of $1\sim3$. \dlo{Besides, i}\wen{I}n conventional EMD\wen{,} the signal is completely decomposed into all the IMFs\wen{,} along with the remainder. In our proposed algorithm, EMD decomposes a signal into only $1\sim3$ components, which correspond to the number of IMFs to be filtered. Compared with the conventional EMD, this uncompleted decomposition algorithm can improve the computation efficiency \dlo{at}\wen{by} about 5 times.

\section{EMD based dip filter}
In this \dlo{part}\wen{section}, we seek to connect $f-x$ EMDPF with $f-x$ predictive filtering and $f-x$ EMD. We would like to first introduce the so-called EMD based dip filter. 


\dlo{Now that the oscillating}\wen{Since the} frequency of each IMF\dlo{s} \dlo{is decreasing}\wen{decreases} according to the \dlo{sequence they are separated out}\wen{order in which it is separated out}, by subtracting the first \dlo{several}\wen{few} IMFs of each frequency slice in \wen{the} $f-x$ domain, we \dlo{can separate out }\wen{extract} the higher wavenumber components, which represent\dlo{s} the energ\dlo{e}y of random noise and high-dip-angle events \dlo{for}\wen{in} seismic sections. 

If we divide the set of IMFs into more\wen{-}detailed zones, we can separate the section into several dip bands. Thus, we reach the definition of the EMD based dip filter:
\begin{equation}
\label{eq:dip}
\tilde{u}_i(f,h)=\left\{\begin{array}{ll}
\epsilon_1u_i(f,h)   &   i\in D_1\\
\epsilon_2u_i(f,h)   &   i\in D_2\\
\vdots	           &   \vdots  \\
\epsilon_mu_i(f,h)   &   i\in D_m
\end{array}\right.,
\end{equation}
\begin{equation}
\label{eq:dip1}
\Lambda(f,h)=\sum_{i=1}^{N}\tilde{u}_i(f,h),
\end{equation}

where $\Lambda(f,h)$ is the filtered data for frequency slice $f$ in the $f-x$ domain. $u_i(f,h)(i=1,2,\cdots,N)$ is the $i$th separated IMF such that $S(f,h)=\sum_{i=1}^{N}u_i(f,h)$, where $S(f,h)$ is the transformed $f-x$ domain seismic data. $D_i(i=1,2,\cdots,m)$ is the $i$th \wen{of $m$, the number of dip bands}\dlo{ dip band, where $m$ is the number of dip bands}, and $\epsilon_i$ is the corresponding weighting coefficient. For a simple high-pass dip filter, we choose $m=2$, $\epsilon_1=1$, $\epsilon_2=0$ and $D_1=\{1,2\}$, $D_2=\{3,4,\cdots,N\}$.

Figure \ref{fig:plane,planeemd1,planeemd2,planeemd3} demonstrates how a\wen{n} EMD based dip filter works on a synthetic plane-wave seismic profile \dlo{. The original synthetic profile contains}\wen{containing} three events corresponding to three dips. After filtering with high-pass, mid-pass\wen{,} and low-pass dip filter\wen{s,} respectively\dlo{. T}\wen{, the t}hree plane waves are successfully separated \dlo{out}. The parameters we choose in designing these three filters are shown in Table \ref{tbl:filterpara}.


\tabl{filterpara}{Parameters in designing high-pass, mid-pass and low-pass dip filters corresponding to Figure \ref{fig:plane,planeemd1,planeemd2,planeemd3}.}
 {
    \begin{center}
     \begin{tabular}{|c|c|c|c|c|}
      \hline Type      & $N$  & $m$  & $\epsilon$   			        & $D$    		 \\ 
      \hline high-pass & 10   &2    & $\epsilon_1=1, \epsilon_2=0$ 		& $D_1=\{1\}, D_2=\{2,3,\cdots,10\}$ \\
      \hline mid-pass  & 10   &3    & $\epsilon_1=0, \epsilon_2=1, \epsilon_3=0$& $D_1=\{1\}, D_2=\{2\}, D_3=\{3,4,\cdots,10\}$ \\
      \hline low-pass  & 10   &2    & $\epsilon_1=0, \epsilon_2=1$ 		& $D_1=\{1\}, D_2=\{2,3,\cdots,10\}$ \\
      \hline
    \end{tabular} 
   \end{center}
} 

The EMD based dip filter is \dlo{adatively }defined \wen{adaptively since}\dlo{becasue} the filtering process is data driven. We only need to define the number of IMFs \wen{contained in} each dip band \dlo{contains} at the \dlo{beginning}\wen{start}, \dlo{which}\wen{a step that} is convenient to implement. 

If we consider random noise as \dlo{"}high-dip-angle\dlo{"} components, then $f-x$ EMD denoising \cite[]{bekara} is \dlo{equal}\wen{equivalent} to applying a high-cut dip filter with the form of equation \ref{eq:dip} to the seismic data \dlo{to remove out }\wen{in order to remove} both random noise and grou\wen{n}d roll\dlo{s}. We can also understand $f-x$ EMDPF \dlo{in the thought of}\wen{from} EMD based dip filter.\dlo{ The} \dlo{\wen{, because the} basic workflow of}$f-x$ EMDPF first uses an EMD based dip filter to separate\dlo{ out} the high-dip-angle and \wen{low-dip-angle components}\dlo{ and random noise}\wen{, where the high-dip-angle components are composed of steeply dipping useful events and noise, and low-dip-angle components are all useful signals}. The useful signal \dlo{components} \wen{in the high-dip-angle components} are predicted and \wen{subsequently} restored\dlo{ subsequently}. Due to the effect\wen{s} of the EMD based dip filter, \wen{a decrease occurs in} the \dlo{amount}\wen{number} of \dlo{the }useful signal components that needs to be predicted\dlo{ decreases}. This \wen{decrease} results in more accurate overall performance when compared with conventional predictive filtering.

\inputdir{dipfilter}
\multiplot{4}{plane,planeemd1,planeemd2,planeemd3}{width=0.45\textwidth}{Demonstration of EMD based dip filter. (a) Original synthetic profile, (b) with high-pass dip filter, (c) with mid-pass dip filter, (d) with low-pass dip filter.}


\section{Examples}

In this section, we first reuse the previously discussed synthetic data (Figure \ref{fig:syn01-complex}), then \dlo{use}\wen{show} two \dlo{an}other synthetic \wen{data} \dlo{examples }and one field data example to demonstrate the performance of ${f-x}$ EMDPF.

Figure \ref{fig:syn01-complex-fxemdpf} shows the denoised result after $f-x$ EMDPF. The removed noise section is shown in Figure \ref{fig:syn01-complex-fxemdpf-noise}. 
Comparing Figure \ref{fig:syn01-complex-fxdecon-noise} and Figure \ref{fig:syn01-complex-fxemdpf-noise}, we see that the useful events leaked into Figure \ref{fig:syn01-complex-fxdecon-noise} have been \dlo{move back}\wen{returned} to the denoised result (Figure \ref{fig:syn01-complex-fxemdpf}), while the noise level stays nearly unchanged.

The second synthetic example is composed of one linear dipping event and two flat events (Figure \ref{fig:syn02-clean,syn02-noise,syn02,syn02-fxdecon,syn02-fxemd,syn02-fxemdpf,syn02-fxdecon-noise,syn02-fxemd-noise,syn02-fxemdpf-noise}). A 40\wen{ }Hz Ricker wavelet has been used with a time sample interval of 4\wen{ }ms. The number of samples for each trace is 501, and the number of traces is 120. Figures \ref{fig:syn02-fxdecon}, \ref{fig:syn02-fxemd}\wen{,} and \ref{fig:syn02-fxemdpf} illustrate the comparison of denoised results of the synthetic data using $f-x$ predictive filtering, $f-x$ EMD filtering\wen{,} and $f-x$ EMDPF\wen{,} respectively. Figure \ref{fig:syn02-clean} \dlo{is the original clean data, and Figure \ref{fig:syn02} show\wen{s} the noisy data after adding Gaussian white noise (Figure \ref{fig:syn02-noise}) to Figure \ref{fig:syn02-clean}.}\wen{is the noise free data, Figure \ref{fig:syn02-noise} is Gaussian white noise, and Figure \ref{fig:syn02} is the noisy data.} The SNR of the noisy data is 2.0 (\dlo{the definition of SNR was given before}\wen{using the previous definition of SNR}). Figures \ref{fig:syn02-fxdecon-noise}, \ref{fig:syn02-fxemd-noise}\wen{,} and \ref{fig:syn02-fxemdpf-noise} show the removed noise sections corresponding to $f-x$ predictive filtering, $f-x$ EMD\wen{,} and $f-x$ EMDPF\wen{,} respectively. From Figure \ref{fig:syn02-fxdecon-noise}, we can see that $f-x$ predictive filtering harms both flat and dipping events to some extent. Although by increasing the length of \wen{the} predictive step we can decrease the damage \dlo{of}\wen{done to the} signals, the noise suppression is less effective \dlo{at the same time }because of the stronger prediction of noise. Also\wen{,} as seen in Figure \ref{fig:syn02-fxdecon}, the $f-x$ predictive filtering introduces some \dlo{artifacts}\wen{artefacts}. In Figure \ref{fig:syn02-fxemd-noise}, we see that $f-x$ EMD tends to harm much of the dip energy but preserve\wen{s} \dlo{the }entire\wen{ly} \wen{the} flat events. Using the same predictive filtering parameters\wen{,} we see from Figure \ref{fig:syn02-fxemdpf-noise} that both flat and dipping signals are hardly affected when $f-x$ EMDPF is applied. \wen{In this example, the first IMF is removed for prediction in the pr\wen{o}cess of $f-x$ EMDPF.} Figure \ref{fig:syn02-fxemdpf-1imf,syn02-fxemdpf-1imf-noise,syn02-fxemdpf-2imf,syn02-fxemdpf-2imf-noise,syn02-fxemdpf-3imf,syn02-fxemdpf-3imf-noise} demonstrates the sensitivity of the $f-x$ EMDPF for increasing \wen{the number of} filtered IMFs. \dlo{As w}\wen{W}e can see \wen{that}, as the number of filtered IMFs increases, the denoising result becomes more similar to $f-x$ predictive filtering\dlo{,}\wen{;} that is, more noise \wen{is} removed and more obvious \dlo{artifacts showing}\wen{artefacts appear} in the denoised section. However, for $f-x$ EMDPF, the horizontal events are always totally preserved, which suppose\wen{s} a generally better denoising result than $f-x$ predictive filtering.

\inputdir{linear}
\multiplot{9}{syn02-clean,syn02-noise,syn02,syn02-fxdecon,syn02-fxemd,syn02-fxemdpf,syn02-fxdecon-noise,syn02-fxemd-noise,syn02-fxemdpf-noise}
{width=0.29\textwidth}{Comparison of denoising effects. (a) Clean data. (b) Gaussian white noise. (c) Noisy data. (d) Denoised result by $f-x$ predictive filtering. (e) Denoised result by $f-x$ EMD. (f) Denoised result by $f-x$ EMDPF. (g) Removed noise section corresponding to (d). (h) Removed noise section corresponding to (e). (i) Removed noise section corresponding to (f).}
\multiplot{6}{syn02-fxemdpf-1imf,syn02-fxemdpf-1imf-noise,syn02-fxemdpf-2imf,syn02-fxemdpf-2imf-noise,syn02-fxemdpf-3imf,syn02-fxemdpf-3imf-noise}
{width=0.4\textwidth}{Comparison of denoising effects. (a) $f-x$ EMDPF denoised result with prediction on 1 IMF. (b) Noise section corresponding to (a). (c) $f-x$ EMDPF denoised result with prediction on 2 IMFs. (d) Noise section corresponding to (c). (e) $f-x$ EMDPF denoised result with prediction on 3 IMFs. (f) Noise section corresponding to (e).}

\inputdir{hyper}
\multiplot{9}{syn03-clean,syn03-noise,syn03,syn03-fxdecon,syn03-fxemd,syn03-fxemdpf,syn03-fxdecon-noise,syn03-fxemd-noise,syn03-fxemdpf-noise}
{width=0.29\textwidth}{Comparison of denoising effects. (a) Clean data. (b) Gaussian white noise. (c) Noisy data. (d) Denoised result by $f-x$ predictive filtering. (e) Denoised result by $f-x$ EMD. (f) Denoised result by $f-x$ EMDPF. (g) Removed noise section corresponding to (d). (h) Removed noise section corresponding to (e). (i) Removed noise section corresponding to (f).}

The third synthetic example is a benchmark data set from SeismicLab. The central frequency of \wen{the} \dlo{ricker}\wen{Ricker} wavelet is 40 Hz and the temporal sampling is 2\wen{ }ms. The number of \dlo{temporal}\wen{time} samples is 750 and the number of spatial samples is 50. Figures \ref{fig:syn03-clean}, \ref{fig:syn03-noise}\wen{,} and \ref{fig:syn03} denote the clean data, noise section\wen{,} and noisy section\wen{, respectively}. The SNR in Figure \ref{fig:syn03} is \dlo{also} 2.0. Figures \ref{fig:syn03-fxdecon}, \ref{fig:syn03-fxemd}\wen{,} and \ref{fig:syn03-fxemdpf} are \wen{the} denoised results using $f-x$ predictive filtering, $f-x$ EMD\wen{,} and $f-x$ EMDPF\wen{,} respectively. From removed noise sections Figures \ref{fig:syn03-fxdecon-noise}, \ref{fig:syn03-fxemd-noise}\wen{,} and \ref{fig:syn03-fxemdpf-noise} we can \dlo{get a more convincing conclusion}\wen{conclude} that $f-x$ predictive filtering harms much useful energy when the number of dip components \dlo{become large}\wen{increases,} \dlo{while}\wen{whereas} $f-x$ EMD \dlo{harms}\wen{affects} most of the dipping events\wen{,} and $f-x$ EMDPF preserve\wen{s} the useful energy to the \dlo{largest}\wen{greatest} extent while removing the slightly weaker level of noise. \wen{In this example, the first IMF is removed for prediction in the process of $f-x$ EMDPF.}

The field data is shown in Figure \ref{fig:southsea,southsea-zoom}. It is a \dlo{post-stack but pre-migration profile}\wen{stacked section without migration} from \dlo{South Sea, China}\wen{the South China Sea}. Figure \ref{fig:southsea-zoom} is a \dlo{temporal} zoomed \dlo{part of}\wen{portion} of Figure \ref{fig:southsea} \wen{from 1.5s to 3.0s.} \dlo{The temporal range is from 1.5s to 3.0s.} The denoised profiles\wen{, shown in Figures \ref{fig:southsea-fxdecon}, \ref{fig:southsea-fxemd}\wen{,} and \ref{fig:southsea-fxemdpf}, demonstrate}  \dlo{using }$f-x$ predictive filtering, $f-x$ EMD\wen{,} and $f-x$ EMDPF\dlo{ are shown in Figures \ref{fig:southsea-fxdecon}, \ref{fig:southsea-fxemd} and \ref{fig:southsea-fxemdpf} }\wen{,} respectively. The corr\wen{e}sponding noise sections are shown in Figures \ref{fig:southsea-fxdecon-noise}, \ref{fig:southsea-fxemd-noise}\wen{,} and \ref{fig:southsea-fxemdpf-noise}\dlo{ respectively}. From \dlo{the}\wen{these} noise sections\wen{,} \dlo{Figures \ref{fig:southsea-fxdecon-noise}, \ref{fig:southsea-fxemd-noise} and \ref{fig:southsea-fxemdpf-noise},} we see clearly that $f-x$ EMD remove\wen{s} \dlo{much}\wen{many} dipping events. Even though we can't see clearly the improvement after applying $f-x$ EMDPF\wen{ at the scale of} \dlo{from }Figure \ref{fig:southsea-fxdecon,southsea-fxdecon-noise,southsea-fxemd,southsea-fxemd-noise,southsea-fxemdpf,southsea-fxemdpf-noise}, \dlo{we find it obvious to see the difference from the zoomed part of the}\wen{the improvement can be identified on the zoomed} noise sections shown in Figure \ref{fig:southsea-fxdecon-zoom,southsea-fxdecon-noise-zoom1,southsea-fxemd-zoom,southsea-fxemd-noise-zoom,southsea-fxemdpf-zoom,southsea-fxemdpf-noise-zoom1}. The useful energy shown around 1.75s, 2.6s in Figure \ref{fig:southsea-fxdecon-noise-zoom1} \dlo{are not existing}\wen{does not exist} in the same part of Figure \ref{fig:southsea-fxemdpf-noise-zoom1}. \dlo{Also, the dip components shown around 2.9s (trace 500) lost in Figure \ref{fig:southsea-fxdecon-noise-zoom1} is not shown in the same part of Figure \ref{fig:southsea-fxemdpf-noise-zoom1}.} From these differences, we \dlo{find}\wen{conclude} that $f-x$ EMDPF is more satisfa\wen{c}tory than $f-x$ predictive filtering and $f-x$ EMD\wen{,} in that it leaves less useful \dlo{events}\wen{energy} in the noise section. In this real data example, we apply the AR model on the first three IMFs for $f-x$ EMDPF\dlo{ so that we can remove as much amount of noise as $f-x$ predictive filtering as possible}. \wen{For display reasons, the noise sections have been amplified by 3 times.}
\inputdir{southsea}
\multiplot{2}{southsea,southsea-zoom}{width=0.45\textwidth}{Field data from \dlo{South Sea, China}\wen{the South China Sea}. (a) Original post-stack pre-migration profile. (b) Temporal zoomed part from 1.5s to 3.0s. }

\multiplot{6}{southsea-fxdecon,southsea-fxdecon-noise,southsea-fxemd,southsea-fxemd-noise,southsea-fxemdpf,southsea-fxemdpf-noise}
{width=0.4\textwidth}{Comparisons between denoised results and corresponding noise sections. (a) Denoised result by $f-x$ predictive filtering. (b) Removed noise section by $f-x$ predictive filtering \wen{($\times 3$)}. (c) Denoised result by $f-x$ EMD. (d) Removed noise section by $f-x$ EMD \wen{($\times 3$)}. (e) Denoised result by $f-x$ EMDPF. (d) Removed noise section by $f-x$ EMDPF \wen{($\times 3$)}.}

\multiplot{6}{southsea-fxdecon-zoom,southsea-fxdecon-noise-zoom1,southsea-fxemd-zoom,southsea-fxemd-noise-zoom,southsea-fxemdpf-zoom,southsea-fxemdpf-noise-zoom1}
{width=0.4\textwidth}{Zoomed part of denoised results and corresponding noise sections. (a) Denoised result by $f-x$ predictive filtering. (b) Removed noise section by $f-x$ predictive filtering \wen{($\times 3$)}. (c) Denoised result by $f-x$ EMD. (d) Removed noise section by $f-x$ EMD \wen{($\times 3$)}. (e) Denoised result by $f-x$ EMDPF. (d) Removed noise section by $f-x$ EMDPF \wen{($\times 3$)}.}

\section{Conclusions}

We have proposed a new denoising method suitable for complex subsurface structures. We demonstrate that the number of dipping events will affect the denoising performance of $f-x$ predictive filtering\dlo{ and}\dlo{\wen{, because} a smaller number of dips \wen{has less leakage of useful energy,} \dlo{leads}\wen{leading} to a more precise prediction\wen{.} \dlo{becasue of less leakage of usful energy.}}. We also give the definition of \wen{an} EMD based dip filter and ascribe the \dlo{effectivity}\wen{effectiveness} of $f-x$ EMD to applying a high-cut EMD based dip filter to seismic profiles.

By using \wen{the} AR model to predict the steep\wen{ly} dipping event, $f-x$ EMDPF can deal with complex seismic profiles \dlo{which}\wen{that} conventional $f-x$ EMD can't \dlo{deal with}\wen{handle}. By applying an EMD based adaptive dip filter in advance, $f-x$ EMDPF can preserve more useful \dlo{events}\wen{energy as} compared with conventional $f-x$ predictive filtering. $f-x$ EMDPF is actually a modification to both $f-x$ predictive filtering and $f-x$ EMD, so it maintains the benefits of \dlo{both} being convenient, data driven\dlo{ and combines}\wen{, whilst combining} the dip-selection property of EMD with the power of the AR model used in $f-x$ predictive filtering.

Although the incomplete EMD described in this paper can improve computational efficiency, \dlo{it still takes much time}\wen{a great deal of time is still required} to process the data. Currently\wen{,} this \wen{time requirement} is the major drawback of \dlo{this}\wen{the} approach. In addition, \wen{continued research is required in order to} \dlo{how to }find an efficient thresholding method in the $f-x$ domain in order to \dlo{further preserve the }\wen{improve the preservation of} useful signal\dlo{ guides the direction of continued research}.







\section{Acknowledgements}
We thank Josef Paffenholz, Mauricio Sacchi, Sergey Fomel, and Karl Schleicher for helpful discussions and all the developers of Madagascar and SeismicLab software packages for providing the codes. We also thank the associate editor Danilo Velis and three anonymous reviewers for their cons\wen{t}ructive suggesions, which help\wen{ed}\dlo{s} to improve the paper. 

%\newpage
%\onecolumn
\bibliographystyle{seg}
\bibliography{emdpf}

%\newpage

\appendix
\section{Appendix A: Sifting algorithm for empirical mode decomposition}

In this appendix, we review the sifting algorithm of empirical mode decomposition (equation \ref{eq:emd} in the main \dlo{context}\wen{paper}). 
For the original signal, we first find the local maxima and minima of the signal\dlo{respectively}. Once \dlo{the extrema are }identified, fit the\wen{se} local maxima and minima by cubic spline interpolation \wen{in turn in order} \dlo{respectively} to generate the upper and lower envelopes\dlo{,}\wen{.} \dlo{t}\wen{T}hen compute the mean of the upper and lower envelopes $m_{11}$, the difference between the data and first mean $h_{11}$.
\begin{equation}
\label{eq:a1}
m_{11}=\frac{h^{+}_{10}+h^{-}_{10}}{2},
\end{equation}
\begin{equation}
\label{eq:a2}
h_{11}=h_{10}-m_{11},
\end{equation}
where $h_{ij}$ denotes the remaining signal after $j$th sifting for generating the $i$th IMF, $h^+_{ij}$ and $h^-_{ij}$ are corresponding upper and lower envelopes\wen{,} respectively, and $m_{ij}$ is the mean of upper and lower envelopes after $j$th sifting for generating the $i$th IMF.
Repeating the sifting procedure (\ref{eq:a2}) $k$ times, until $h_{1k}$ reach the prerequisites of IMF, \dlo{that is}\wen{these are}:
\begin{equation}
\label{eq:a3}
h_{1(k-1)}-m_{1k}=h_{1k}.
\end{equation}
The criterion for \wen{the} sifting process to stop is given by \cite{emd} as:
\begin{equation}
\label{eq:a4}
0.2\le SD=\sum_{t=0}^{T}\left[\frac{|h_{1(k-1)}(t)-h_{1k}(t)|^2}{h^2_{1(k-1)}}\right]\le 0.3,
\end{equation}
where $SD$ denotes the standard deviation.
When $h_{1k}$ is considered as an IMF, let $c_1=h_{1k}$, we separate the first IMF from the original data:
\begin{equation}
\label{eq:a5}
d-c_1=r_1,
\end{equation}
where $d$ is the original signal, $c_n$ denotes the $n$th IMF\wen{,} and $r_n$ is the residual after \wen{the} $n$th IMF based sifting.
Repeat\wen{ing} the sifting process from equation \ref{eq:a1} to \ref{eq:a5}\wen{,} \dlo{except for }changing $h_{1j}$ to $h_{ij}$, \dlo{we can}\wen{in order to} get the following IMFs: $c_2, c_3, \cdots, c_N$.
The sifting process can be stopped when the residual $r_n$, becomes so small that it is less than \dlo{the}\wen{a} predetermined value of substantial consequence, or when $r_n$  becomes a monotonic function from which no more IMF can be extracted.

Finally, we achieved a decomposition of the original data into N modes, and one residual, as shown in equation \ref{eq:emd} in the main context.




