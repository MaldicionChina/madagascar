\author{Vladimir Bashkardin, Sergey Fomel, Parvaneh Karimi, Siwei Li, and Xiaolei Song}
\title{Marmousi model}

\maketitle

\begin{abstract}
The Marmousi model was created in 1988 by the Institut Fran\c{c}ais du
P\'{e}trole (IFP) in 1988.  The geometry of this model is based on a
profile through the North Quenguela trough in the Cuanza basin. The
geometry and velocity model were created to produce complex seismic
data which require advanced processing techniques to obtain a correct
earth image \cite[]{TLE13-09-09270936}. The Marmousi dataset was used
for the workshop on practical aspects of seismic data inversion at the
52nd EAEG meeting in 1990.
\end{abstract}

\section{Model}
\inputdir{model}

\plot{vel}{width=\textwidth}{Velocity model.} \clearpage
\plot{shots}{width=\textwidth}{Shot gathers from finite-diffence modeling.} \clearpage

\plot{shots2}{width=\textwidth}{Shot gathers from Fourier finite-diffence modeling, shot-receiver coordinates.} \clearpage

\plot{shots3}{width=\textwidth}{Shot gathers from Fourier finite-diffence modeling, shot-offset coordinates.} \clearpage

\plot{refl}{width=\textwidth}{Exloding reflector.} \clearpage
\plot{marmexp}{width=\textwidth}{High-resolution zero-offset data from exploding-reflector modeling.} \clearpage

\section{Lowrank RTM}
\inputdir{lowrank}

\plot{zomig}{width=\textwidth}{Zero-offset  RTM image by the lowrank
  method.} \clearpage
\plot{impres}{width=\textwidth}{Impulse response from  RTM by the lowrank method.} \clearpage

\section{FFD RTM}
\inputdir{ffd}

\plot{zomig}{width=\textwidth}{Zero-offset RTM image by Fourier finite-differences.} \clearpage

\plot{rtm}{width=\textwidth}{Prestack RTM image by Fourier finite-differences.} \clearpage

\section{One-way wave extrapolation}
\inputdir{oway}

\plot{mig}{width=\textwidth}{Zero-offset image by one-way wave-equation migration using extended split-step approximation.} \clearpage

\section{First-arrival Kirchhoff}
\inputdir{recurkir}

\plot{dmig0}{width=\textwidth}{First-arrival Kirchhoff migration.} \clearpage
\plot{dmig1}{width=\textwidth}{First-arrival semirecursive Kirchhoff migration, two steps.} \clearpage
\plot{dmig2}{width=\textwidth}{First-arrival semirecursive Kirchhoff migration, three steps.} \clearpage

\section{Common Reflection Angle Migration}
\inputdir{cram}

\plot{marmdcrstk}{width=\textwidth}{Common reflection angle migration.}
\multiplot{2}{marmocig0,marmdcig0}{width=0.45\textwidth}
{Opening (a) and dip (b) angle gathers at 2 km.}
\multiplot{2}{marmocig1,marmdcig1}{width=0.45\textwidth}
{Opening (a) and dip (b) angle gathers at 4 km.}
\multiplot{2}{marmocig2,marmdcig2}{width=0.45\textwidth}
{Opening (a) and dip (b) angle gathers at 6 km.}
\multiplot{2}{marmocig3,marmdcig3}{width=0.45\textwidth}
{Opening (a) and dip (b) angle gathers at 8 km.} \clearpage


\bibliographystyle{seg}
\bibliography{SEG}
