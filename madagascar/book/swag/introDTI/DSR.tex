\section{Downward Continuation}

The angle decomposition discussed in the preceding section allows us
to produce angle gathers after downward continuation in DTI media.
Wavefield reconstruction for multi-offset migration based on the
one-way wave-equation under the survey-sinking framework
\cite[]{Claerbout.iei} is implemented by recursive phase-shift of
prestack wavefields
%
\beq \label{eqn:PHS-SS}
\UU_{z+\dz}\ofmh = \PS{-}\UU_z\ofmh \;,
\eeq
%
followed by extraction of the image at time $t=0$.  Here, $\mm$ and
$\ho$ represent the midpoint and half-offset coordinates, which are
equivalent with the space and space-lag variables discussed earlier,
but restricted to the horizontal plane. In \req{PHS-SS}, $\UU_z\ofmh$
represents the acoustic wavefield for a given frequency $\ww$ at all
midpoint positions $\mm$ and half-offsets $\ho$ at depth $z$, and
$\UU_{z+\dz}\ofmh$ represents the same wavefield extrapolated to depth
$z+\dz$. The phase shift operation uses the depth wavenumber $k_z$
which is defined in 2D by the DSR \req{3p7} as follows:
%
\beq \label{eqn:dsr22}
k_z = 
\sqrt{ {\omega^2 s^2(\theta) } - \lp k_m - k_h \rp^2} + 
\sqrt{ {\omega^2 s^2(\theta) } - \lp k_m + k_h \rp^2 } \;,
\eeq
\geouline{where $k_h$ is equivalent to $\klx$}.

\rFg{kz} shows $k_z$ as a function of the midpoint wavenumber and the
reflection angle for a DTI model characterized by $\eta=0.3$ (left).
As expected, the range of angles reduces with increasing dip angle (or
$k_m$). The phase shift per depth is maximum for horizontal reflector
($k_m=0$) and zero offset (equivalent with $\theta=0$). The right
plot in \rFg{kz} shows the difference between the $k_z$ for this DTI
model and that for an isotropic model with velocity equal to
$v=1.8$~km/s. As expected, for zero reflection angle, the
DTI phase shift is given by the isotropic operator as we discussed
earlier. For the non-zero-offset case, the difference increases with
the reflection angle.

To use $k_z$ in this form we need to evaluate the reflection angle,
$\theta$, in the downward continuation process as the angle gather defines the phase angle needed for \req{dsr22}.
 \rEq{quaddd36} provides a one-to-one relation between
angle gathers and the offset wavenumber. However, to insure an
explicit evaluation we formulate the problem as a mapping process to
find the wavefield for a given offset wavenumber $k_h$ that
corresponds to a particular reflection angle. As a result, we can
devise an algorithm for downward continuation for a wavefield with
sources and receivers at depth $z$ as follows:
\begin{itemize}
\item For a given reflection angle, use \req{quaddd36} to find the
  corresponding $k_h$ \geouline{($=\klx$)}.
\item Using $k_h(\theta)$, map $\UU(k_m,k_h,\ww,z)$ to $\UU(k_m,
  \theta,\ww,z)$ (the angle decomposition).
\item Apply the imaging condition by summing over frequencies $\ww$ to
  obtain imaged angle gathers.
\item Apply phase shift to the wavefield $\UU(k_m,k_h,\ww, z)$ \geouline{to}
  obtain $\UU(k_m,k_h,\ww, z+\dz)$ by \req{PHS-SS} using the depth
  wavenumber given by \req{dsr22}.
\item \geosout{We} Repeat the steps for depth $z=z+\dz$.
\end{itemize}
The process provides imaged angle gathers in DTI media.  This approach
also allows us to better treat illumination as we downward
continue while keeping the sampling in reflection angle uniform.

\inputdir{Math} 

\plot{kz}{width=1.0\textwidth}{A plot of the vertical wavenumber,
  $k_z$, as a function of midpoint wavenumber and angle gather for a \geoulin2{dip-constrained transversely isotropic
  (DTI)} \geosou2{DTI} model with \geouline2{NMO velocity,} $v$=2.0 km/s, \geoulin2{titled direction velocity,} $v_T$=1.8 km/s, and $\eta=0.3$ (left)
  and the difference in $k_z$ between the DTI model and an isotropic
  model with $v$=1.8 km/s (right). 
\geouline{The wave numbers are given in units of $km^{-1}$.}}

