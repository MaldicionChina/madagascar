\footer{SEP--89}
\published{Journal of Seismic Exploration, v. 10, 293-310 (2002)}
\title{Antialiasing of Kirchhoff operators by reciprocal
  parameterization}

\author{Sergey Fomel}
\maketitle

\newcommand{\maindir}{\$RSFSRC/book/tutorial}
\newcommand{\exampledir}{\maindir/authors}

\section{Getting started}

Before you can get started writing reproducible documents, you need to ensure that your system is properly setup.  This section of the tutorial will walk you through the setup process, which can be somewhat difficult and laborious depending on which operating system you are on, as you will need a full installation of \LaTeX and additional \LaTeX class files that Madagascar makes available to you. The good news is that this configuration only happens once.

\subsection{Downloading \LaTeX}

To begin, you need to download a full installation of \LaTeX for your operating system.  To do so, you may need to contact your system administrator.  If you have administrative rights, then you can download a full install for your platform from the following locations:

\begin{itemize}
\item Linux - use your package management software to install a full texlive (you may need additional packages depending on your distribution).
\item Mac - install MacTex \url{http://www.tug.org/mactex/2011/}.
\item Windows - install MikTex \url{http://miktex.org/}.
\end{itemize}

The respective downloads typically are quite large and take a substantial amount of time to complete, so start the download and come back later.  Once your download is done, test your installation by executing \textbf{pdflatex} at the command line.  If everything works fine then continue onwards.

\subsection{Downloading SEGTeX}

The next step is to download the SEG\TeX class files, which tells \LaTeX how to build certain documents that we commonly use.  To download SEG\TeX first \textbf{cd} to a directory where \LaTeX can find additional class files.  This directory is typically operating system dependent, and may be distribution dependent for Linux.  Typically, you can place these files in \textbf{\$HOME/texmf}.  Otherwise, you will need to place them in the root installation for Latex which can be found by searching for the basic class files, such as article.cls.  On a Mac the typical place to put these files is \textbf{\$HOME/Library/texmf}.  In any case, once you are in the proper location, then use the following command to download the class files (using subversion):
\begin{verbatim}
svn co https://segtex.svn.sourceforge.net/svnroot/segtex/trunk texmf 
\end{verbatim}
or download a stable release from \url{http://sourceforge.net/projects/segtex/files/} and unpack it into the \texttt{texmf} directory.

\subsection{Updating your \LaTeX install}

Once the class files are successfully downloaded, you will need to run \textbf{texhash} or \textbf{texconfig rehash} to update \LaTeX about the new class files.  For reference, a successful run of \textbf{texhash} produces the following output:

\begin{verbatim}
jgodwin$ texhash
texhash: Updating /usr/local/texlive/2010/../texmf-local/ls-R... 
texhash: Updating /usr/local/texlive/2010/texmf/ls-R... 
texhash: Updating /usr/local/texlive/2010/texmf-config/ls-R... 
texhash: Updating /usr/local/texlive/2010/texmf-dist/ls-R... 
texhash: Updating /usr/local/texlive/2010/texmf-var/ls-R... 
texhash: Done.
\end{verbatim}

To determine if these files are found successfully, go into \$RSFROOT/book/rsf/manual and run \textbf{scons manual.read}.  If \LaTeX complains about being unable to find the class files then you should re-run texhash, or make sure that your class files are in the proper location.  If you are still having difficulties, then contact the \href{http://www.ahay.org/wiki/SEGTeX}{SEG\TeX webpage} or the user mailing list for further information.

\section{Papers}

With \LaTeX installed, we can now create reproducible documents using Madagascar.  First, we will demonstrate how to build shorter, less complicated documents using Madagascar, such as SEG/EAGE abstracts, Geophysics articles, and handouts.  All of these papers have similar build styles, so the rules for building each respective paper have only slight differences from one another.  Instead of talking in detail about each of these documents, we illustrate the basic idea for each of the documents, and provide examples that demonstrate the particulars for each type of document.  

\subsection{Paper organization}

All Madagascar papers expect a specific organization to your directories.  In particular, you are expected to have a paper-level directory where your \textbf{tex} files and main SConstruct will exist.  These files will tell Madagascar how to build your documents for a particular project.  You can have multiple documents built from the same location, using the same SConstruct as we will demonstrate later.  

Below the paper directory, are the individual ``chapters" that contain the processing flows used to generate the plots or process the data that you wish to use in your reproducible documents.  Ideally, each ``chapter" directory correlates to the processing flows or examples in each chapter or section for your paper.  Additionally, each ``chapter" contains its own SConstruct that operates independently of the paper SConstruct one level above it.  Furthermore, inside the ``chapter" folder, Madagascar needs to have a \textbf{Fig} folder that contains all of the VPLOT files that were created using Result commands during processing.    This folder is automatically created during processing using SCons, so you don't need to manually create it.  It is important to note that Madagascar can only locate VPLOT files that are in this file hierarchy for use in your papers.  Figure~\ref{fig:paperhierarchy} illustrates the folder hierarchy as well.

Note: ``chapter" folders may be symbolic links that point to folders elsewhere on the file system.  This trick can be useful to reuse figures without copying files and folders around to various folders.  If you use symlinks, make sure to avoid editing files that are symbolically linked, as your changes may propagate in unintended ways to other projects and papers.

\setlength{\unitlength}{1in}
\begin{figure}
\begin{picture}(5,4)(0,1)
\put(0,4.5){\makebox(1.0,0.5)[c]{Folder hierarchy}}   \put(3.0,4.5){\makebox(1.0,0.5)[c]{Contents}}

\put(0,4){\framebox(0.75,0.5)[c]{Book}} 
\put(2,4){\framebox(3.0,0.5)[c]{Book SConstruct}}
\put(0.35,4){\vector(0,-1){0.5}}
\put(0.75,4.25){\vector(1,0){1.25}}

\put(0,3){\framebox(0.75,0.5)[c]{Chapter}} 
\put(2,3){\framebox(3.0,0.5)[c]{Paper SConstruct, TeX files}}
\put(0.35,3){\vector(0,-1){0.5}}
\put(0.75,3.25){\vector(1,0){1.25}}

\put(0,2){\framebox(0.75,0.5)[c]{Project}} 
\put(2,2){\framebox(3.0,0.5)[c]{Processing SConstruct, data, RSF files}}
\put(0.35,2){\vector(0,-1){0.5}}
\put(0.75,2.25){\vector(1,0){1.25}}

\put(0,1){\framebox(0.75,0.5)[c]{Fig}} 
\put(2,1){\framebox(3.0,0.5)[c]{VPLOT files, Results ONLY}}
\put(0.75,1.25){\vector(1,0){1.25}}
\end{picture}
\caption{The organizational hierarchy for Madagascar paper directories.}
\label{fig:paperhierarchy}
\end{figure}

\subsection{Locking figures}

Once you have created the necessary folder hierarchy with your ``chapters" and processing flows, then go ahead and run your processing SConstructs.  After those are finished, you need to lock your figures using \textbf{scons lock}.  \textbf{scons lock} tells Madagascar that the figures you have generated are ready to go into a publication, and will store them in a subfolder of the \textbf{\$RSFFIGS} directory for safe keeping.  Locked figures are used for document figures instead of the figures in your local directory, because they are locked and not still changing.  If you change your plots but do not lock your figures, you will not see your figures change.  Always make sure to lock your figures before building a document.

\subsection{Tex files}

Now that your figures are locked, you can create your first reproducible document in Madagascar.  To do so, you will need to:
\begin{itemize} 
\item make your tex files, and 
\item make a paper SConstruct, 
\end{itemize}

Before making a document, you need to create your TeX files in the paper level directory.   For example, to create an EAGE abstract, you would create a main TeX file called: \textbf{eageabs.tex} which contains the content and TeX commands to build your abstract.  Your TeX file can use all of the standard and expanded \LaTeX commands provided by any available packages on your system. It's important to remember that you should try and break apart your TeX files into manageable chunks, so that you can modify them independently, or reuse the content in other documents.  For example, instead of having a single TeX file for your EAGE abstract, you could have a separate TeX file that contains: \textbf{\\input{...}} statements that include additional TeX files for each section, such as the abstract, theory, discussion, conclusions, etc.

Additionally, Madagascar provides some convenience commands for often used \LaTeX functions.  Here is a short description of some of those convenience commands that you may run across.  Here's a brief list of these convenience functions:
\begin{itemize}
\item \verb#\plot#,
\item \verb#\multiplot#,
\item \verb#\sideplot#,
\item and more.
\end{itemize}

These convenience functions are not available for every type of document, but are demonstrated in documents where they are available.  The definition for the convenience functions may be found in the \LaTeX class definitions listed at the end of this tutorial.

\subsection{Paper SConstructs}

One of Madagascar's aims is to make TeX files as layout-agnostic as possible.  To do so, Madagascar automatically adds the TeX document preamble (including the \LaTeX document class information), the \LaTeX package inclusions, and end of document information at runtime.  This allows you to generate multiple documents from a single TeX file by simply changing the SConstruct, instead of the TeX file.  

Note: the paper SConstruct is only used to build papers.  It contains no other information, and cannot be used to process data in the same SConstruct.  This is why the paper SConstruct must exist in a separate directory from any processing SConstructs.

The paper SConstruct is very simple compared to most processing SConstructs, in that it contains only a few lines as shown below (in an example for an EAGE abstract):
\lstset{language=python,showstringspaces=false}
\begin{lstlisting}
from rsf.tex import *

Paper('eageabs',
      lclass='eageabs',
      options='11pt',
      use='times,natbib,color,amssymb,amsmath,amsbsy,graphicx,fancyhdr')
\end{lstlisting}

The first section, \textbf{from rsf.tex import *} tells Madagascar to import Python packages for processing TeX files instead of the usual processing packages.  Next, we call a \textbf{Paper} object, which takes the following parameters:
\begin{verbatim}
Paper(name,lclass,options,use)
name - name of the root tex file to build.
lclass - name of the LaTeX class file to use.
options - document options for LaTeX class file.
use - names of LaTeX packages to import during compilation.
\end{verbatim}
All of the parameters are passed as strings to the Paper object.  Parameters with more that one possible value (e.g. options and use) accept comma delimited strings as shown above.  

To generate different types of documents, you simply change the \textbf{lclass} and options sent to the Paper object in the SConstruct for the respective document type.  Since the documents that we are creating use custom \LaTeX document classes that require additional TeX commands to function properly, it is easier for us to provide you with a template instead of discussing the details of each document class.  The templates for the documents can be found in the following directory: \textbf{\exampledir}.  

\subsection{Templates}

To run the templates, you first need to generate the data used for them in the \textbf{data} directory inside of the \textbf{\exampledir}.  To do so, run \textbf{scons lock} which will produce and lock the figures necessary.  Then go into the template directory that you are interested in, and make a symbolic link to the data directory: \textbf{ln -s ../data} and a symbolic link to the BibTeX file: \textbf{ln -s ../demobib.bib} in the template directory.  After those steps are made you can build and view the paper using \textbf{scons} or \textbf{scons paper.read} where paper is the name of the root tex file.  Of course, if you want to remove all the generated files, then you can use \textbf{scons -c} to clean the directory.

\subsection{Handouts}

Handouts are informal documents that are loosely formatted, and very flexible.  The handout example is located in: \textbf{\exampledir/handout}.  Handouts do not require many additional arguments and are the most flexible of the documents discussed here.

\subsection{EAGE abstracts}

EAGE abstracts are short documents, with a few particular formatting tricks. In particular, EAGE requires the logo of the current year's convention to appear in the abstract.  A template is available in: \textbf{\exampledir/eage}.  

\subsection{SEG abstracts}

SEG abstracts are different from EAGE abstracts in that they require two-column formatting and are strictly limited to four pages not including references.  To build an SEG abstract, we first build the abstract, and then build another document using the segcut.tex file that removes the references from the final pdf.  An example is found in: \textbf{\exampledir/seg}.  

\subsection{Geophysics manuscripts}

Geophysics manuscripts come in two flavors: the first is the manuscript prepared for peer review, and the second is the final document that would appear in a print version of Geophysics.  The example shows how to build both from the same TeX files, which makes it painless to transition the formatting between the two documents.  An example is located in: \textbf{\exampledir/geophys}.  Make sure to use the template as there are quite a few additional TeX commands that have to be used to get the correct formatting.

\subsection{CWP reports}

CWP reports are slightly more complicated then most documents in that they require substantial modification to get the proper formatting.  The CWP template is available in \textbf{\exampledir/cwp}.  

\section{Slides}

Additionally, one can create presentation slides using \LaTeX  and Madagascar together.  To create slides, we use the Beamer class files that have been customized for the CWP.  Slides have distinctly different commands then regular documents, so be sure to examine the template before diving in.  The template is in: \textbf{\exampledir/slides}.  

\section{Theses}

One can also create very complex documents using Madagascar in a reproducible way.  To iillustrate this point we provide a template for building a thesis for the Colorado School of Mines.  This template is quite heavily modified, and requires substantial modification due to all the formatting requirements.  If you want to include a thesis template for another institution then you can do so by examining this template along with the CSM class files.  The template is located in: \textbf{\exampledir/thesis}.

\section{Books}

You can make whole books using Madagascar.  The advantage to doing so, is that you can make individual chapters with examples of processing or workflows that can be run independently of one another.  Then Madagascar will stitch the chapters together into a cohesive package seamlessly.  The example for a book is this document itself, which is located in: \textbf{\maindir}.  Note: creating a book is significantly different from creating a paper.

\section{Adding/modifying \LaTeX class files}

The \LaTeX class files made available from SEGtex are found in \textbf{texmf/tex/latex}.  You can modify these files locally by changing the files inside this location.  

To add your own \LaTeX class files, place them in this same directory, and then request SEG\TeX access to commit them to the main repository.

\section{Using the default \LaTeX class files}

Lastly, you can use any of the default \LaTeX class files just by changing the options to the Paper object to the appropriate lclass and options. For example:
\begin{lstlisting}
Paper('article',
      lclass='article',
      options='11pt',
      use='times,natbib,color,amssymb,amsmath,amsbsy,graphicx,fancyhdr')
\end{lstlisting}


\section{Overview of existing methods}

I start with reviewing the existing approaches to operator
antialiasing and discussing their main principles and limitations. The
two reviewed approaches are temporal filtering, as suggested by
\cite{GPR40-05-05650572} and \cite{SEG-1994-1282}, and Hale's spatial
filtering technique, developed originally for an integral
implementation of the dip moveout operator \cite[]{GEO56-06-07950805}.

\subsection{Temporal filtering}
%%%%%%%%%%%%%%%%%%%%%%%%
The temporal filtering idea follows
from the well-known Nyquist sampling criterion. With application to
integral operators, the Nyquist criterion takes the form
\begin{equation}
\Delta x \leq {{\Delta t} \over {|\partial t / \partial x|}}\;,
\label{eqn:Nyquist}
\end{equation}
where $t(x)$ is the traveltime of the operator impulse response,
$\Delta x$ is the space sampling interval and $\Delta t$ is the time
sampling interval. In the steep parts of the traveltime curve, the
sampling criterion (\ref{eqn:Nyquist}) is not satisfied, which causes
aliasing artifacts in the output data. To overcome this problem, the
method of local triangle filtering
\cite[]{Claerbout.sep.73.371,SEG-1994-1282} suggests convolving the
traces of the generated impulse response with a triangle-shaped filter
of the length
\begin{equation}
\delta t = \Delta x\,|\partial t / \partial x|\;.
\label{eqn:dt}
\end{equation}

\inputdir{XFig}

Cascading operators of causal and anticausal numerical integration is
an efficient way to construct the desired filter shape.

Triangle filters approximate the ideal (sinc) low-pass time filters.
The idea of using low-pass filtering for antialiasing
\cite[]{GPR40-05-05650572} is illustrated in Figure \ref{fig:amolow}.
When a steeply dipping event is included in the operator, its
counterpart in the frequency domain wraps around to produce the
aliasing artifacts. Those are removed by a dip-dependent low-pass
filtering.

\plot{amolow}{width=4.5in,height=1.5in}{Schematic
illustration of low-pass antialiasing (triangle filters). The aliased
events are removed by low-pass filtration on the temporal frequency axis.
The width of the low-pass filter depends on dips of the aliased events.}

\inputdir{flt}

The method of low-pass filtering is less evident in the case of a
three-dimensional integral operator. We can take the length of a
triangle filter proportional to the absolute value of the time
gradient \cite[]{SEG-1994-1282}, the maximum of the gradient
components in the two directions of the operator space
\cite[]{GEO64-06-17831792}, or the sum of these components. The latter
follows from considering the 3-D operator as a double integration in
space. Decoupling the 3-D integral into a cascade of two 2-D operators
suggests convolving two triangle filters designed with respect to two
coordinates of the operator. In this case, the length of the resultant
filter is approximately equal to
\begin{equation}
\delta t =      \Delta x\,|\partial t / \partial x| + 
                \Delta y\,|\partial t / \partial y|\;,
\label{eqn:dt3}
\end{equation}
and its shape is smoother than that of a triangle filter (Figure
\ref{fig:amoflt}).

\plot{amoflt}{height=2.5in}{Building the smoothed filter for
3-D antialiasing by successive integration of a five-point wavelet. C
denotes the operator of causal integration, C' denotes its adjoint (the
anticausal integration). The result is equivalent to the convolution
of two equal triangle filters.}

The temporal filtering method was proven to be an efficient tool in
the design of stacking operators of different types. However, when the
operator introduces rapid changes in the length and direction of the
traveltime gradient, it leads to an inexact estimation of the filter
cutoff (triangle length for the method of triangle filtering) at the
curved parts of the operator. Consequently, the high-frequency part of
the output can be distorted, causing a loss in the image resolution.

\subsection{Hale's method}
%%%%%%%%%%%%%%%%%%%%

Considering the case of integral dip moveout, \cite{GEO56-06-07950805}
points out that the steep parts of the operator, while aliased in the
space (midpoint) coordinate, are not aliased with respect to the time
coordinate. He suggests replacing the conventional $t(x)$
parameterization of the DMO impulse response by $x(t)$
parameterization. Conventionally, the integral operators are
implemented by shifting the input traces in space and transforming
them in time.  According to Hale's method, the traces are shifted in
time and transformed along the $x(t)$ trajectories in space.
Interpolation in time, required in the conventional approach, is
replaced by interpolation in space. The idea of Hale's method is
related to the idea of the ``pixel-precise velocity transform''
\cite[]{Claerbout.blackwell.92}.

The steep parts of the operator satisfy the criterion
\begin{equation}
\Delta t \leq {{\Delta x} \over {|\partial x / \partial t|}}\;,
\label{eqn:Nyquist2}
\end{equation}
which is the the obvious reverse of inequality (\ref{eqn:Nyquist}).
Therefore, they are not aliased if defined on the time grid. In these
parts, one can implement the operator by constant time shifts equal to
the time sampling interval $\Delta t$. In the parts where the
criterion (\ref{eqn:Nyquist2}) is not valid (the flat part of the DMO
operator), Hale suggests reducing the length of the time shifts
according to equality (\ref{eqn:dt}), where $\delta t$ becomes less
than $\Delta t$. He formulates the following principle of operator
antialiasing:
\begin{quote}
  To eliminate spatial aliasing, simply never allow successive time
shifts applied to the input trace to differ by more than one time
sampling interval. Further restrict the difference between time shifts
so that the spacing between the corresponding output trajectories
never exceeds the CMP sampling interval.
\end{quote}

\inputdir{XFig}

The idea of Hale's method is illustrated in Figure \ref{fig:amosft}.
Increasing the density of spatial sampling by small successive time
shifts implies increasing the Nyquist boundaries of the spatial
wavenumber. Further interpolation is a low-pass spatial filtering that
removes the parts of the spectrum beyond the Nyquist frequency of the
output. If the dip of the operator does not vary between neighboring
traces (the operator is a straight line as in the slant stack case),
Hale's approach will produce essentially the same result as that of
temporal filtering. Triangle filters in this case approximately
correspond to linear interpolation in space between adjacent traces
\cite[]{Nichols.sep.77.283}. The difference between the two approaches
occurs if the local dip varies in space as in the case of a curved
operator, such as DMO. In this case, Hale's approach provides a more
accurate space interpolation of the operator and preserves the
high-frequency part of its spectrum from distortion.

\plot{amosft}{width=4.125in,height=1.5in}{Schematic
illustration of Hale's antialiasing. The aliased events are removed by
spatial interpolation. In the frequency domain, the interpolation
consists of widening and low-passing on the wavenumber axis. The
low-pass spatial filtering does not depend on dip.} 

Hale's method has proven to preserve the amplitude of flat reflectors
from aliasing distortions, which is the simplest antialiasing test on
a DMO operator. The most valuable advantage of this method in the fact
that the implied low-pass spatial filtering (interpolation) does not
depend on the operator dip and is controlled by the Nyquist boundary
of the spectrum only (compare Figures \ref{fig:amolow} and
\ref{fig:amosft}). This is especially important, when the local dip of
the operator changes rapidly and therefore cannot be estimated
precisely by finite-difference approximation at spatially separated
traces. Such a situation is common in dip moveout and azimuth moveout
integral operators, as well as in prestack Kirchhoff migration.

A weakness of the method is the necessity to switch from
interpolation in space to two-dimensional interpolation in both the
time and the space variables, when trying to construct the flat part
of the operator. 
%In the case of AMO, the 2-D spatial interpolation
%arises as a result of building the operator in the transformed
%coordinate system. 
In the next section, I show how to avoid the expense of the additional
time interpolation required by Hale's method of antialiasing.

\section{Proposed technique}
%%%%%%%%%%%%%%%%%%%%%%%%%%
We can use the reciprocity of the time parameterization and the space
parameterization of integral operators, discovered by Hale, to arrive
at the following antialiasing technique.

For simplicity, let us consider the two-dimensional case first.  The
linearity of a two-dimensional integral operator allows us to
decompose this operator into two parts. The first part corresponds to
the steep part of the travel-time function, which satisfies the
time-sampling criterion (\ref{eqn:Nyquist2}). The second term
corresponds to the flat part of the traveltime, which satisfies the
space-sampling criterion (\ref{eqn:Nyquist}). The first part is not
aliased with respect to the time sampling interval, while the second
one is not aliased with respect to the space sampling. We can apply
interpolation in time to construct the flat part.  Reciprocally,
interpolation in space is applied to construct the steep part of the
operator in the fashion of Hale's time-shifting method. 
%Linear
%interpolation in this case is a cheap substitution for the accurate
%but computationally expensive sinc interpolation. 
The amplitude
difference between the two integrals is simply the Jacobian term

\begin{equation}
{\mbox{amp}_t \over \mbox{amp}_x}=
\left|{\partial x \over \partial t}\right|\,{\Delta t \over \Delta x}=
{\Delta t \over \delta t} \leq 1\;.
\label{eqn:Jacobian}
\end{equation}

According to the proposed modification, Hale's antialiasing principle
is reformulated, as follows:
\begin{quote}
{\em In the steep part of an integral operator, never allow successive
time shifts applied to the input trace to differ by more than one time
sampling interval. In the flat part of the operator, never allow
successive space shifts to differ by more than one space sampling
interval.}
\end{quote}

\inputdir{flt}

Figure \ref{fig:amotra}, borrowed from \cite{Claerbout.bei.95},
illustrates the basic idea of the proposed technique. It clearly shows
the difference between the flat and steep parts of migration
hyperbolas. To observe the reciprocity, rotate the figure by 90
degrees.

\plot{amotra}{height=2in}{Figure borrowed from
  \cite{Claerbout.bei.95} to illustrate the reciprocity
  antialiasing. The flat parts of the hyperbolas require interpolation
  in time. The steep parts of the hyperbolas require interpolation in
  space.}

\inputdir{mod}

To compare the proposed antialiasing method with the temporal
filtering method, I test the antialiased migration program on simple
2-D synthetic tests. Figure \ref{fig:amomod} shows a simple model and
the modeling results from modeling without antialiasing, with temporal
filtering, and with the proposed reciprocity method.  The modeling
results were migrated with the corresponding migration operators to
obtain the image of the model in Figure \ref{fig:amomig}.  Both the
temporal filtering and the proposed method succeed in removing the
major aliasing artifacts. However, the reciprocity method demonstrates
a higher resolution and a better preservation of the frequency
content.

\plot{amomod}{width=4.5in}{Top left is a synthetic model. Top
right is modeling without antialiasing. Bottom left is modeling with
reciprocity antialiasing (the proposed method). Bottom right is modeling
with antialiasing by temporal filtering.}

\plot{amomig}{width=4.5in}{Top left plot is the synthetic
model. The other plots are migrations of the corresponding data shown
in the previous figure . Top right is a migration without
antialiasing. Bottom left is a migration with reciprocity antialiasing
(the proposed method). Bottom right is a migration with triangle
filter antialiasing.}

\inputdir{sig}

These properties are examined more closely in the next synthetic
example. Figure \ref{fig:amosmo} shows a more sophisticated synthetic
model that contains a fault, an unconformity and layered structures
\cite[]{Claerbout.bei.95}. For better displaying, I extract the central
part of the model and compare it with the migration results of
different methods in Figure \ref{fig:amosmi}. Comparing the plots
shows that the reciprocity method successfully removes the aliasing
artifacts (round-off errors) of the aliased (nearest neighbor
interpolation) migration.  At the same time, it is less harmful to the
high-frequency components of the data than triangle filtering. This
conclusion finds an additional support in Figure \ref{fig:amospe} that
displays the average spectrum of the image traces for different
methods. Both of the antialiasing methods remove the high-frequency
artifacts of the nearest neighbor modeling and migration. The
reciprocity method performs it in a gentler way, preserving the
high-frequency components of the model.

\plot{amosmo}{height=2.5in}{Synthetic model used to test
the antialiased migration program.}

\plot{amosmi}{width=6in}{Top left plot is a
zoomed portion of the synthetic model. The other plots are migrated
images. Top right is a migration without antialiasing. Bottom left is
a migration with reciprocity antialiasing (the proposed method). Bottom
right is a migration with triangle filter antialiasing.}

\plot{amospe}{height=2.5in}{Top is the spectrum of the
model. The other plots are the spectra of the migrated images. The
second plot corresponds to the modeling/migration without account for
antialiasing. The third plot is modeling/migration with the
reciprocity antialiasing. The bottom plot is modeling/migration with
triangle antialiasing.}  

The algorithm sequence of the antialiased migration is illustrated in
Figures \ref{fig:amormo} and \ref{fig:amormi}. The two plots in Figure
\ref{fig:amormo} show the steep-dip and flat-dip modeling respectively. The
superposition of these two terms is the resultant antialiased data
shown in the left plot of Figure \ref{fig:amormm}. The right plot of
Figure \ref{fig:amormm} shows the migrated image obtained by adding the
flat-dip (left of Figure \ref{fig:amormi}) and steep-dip (right of Figure
\ref{fig:amormi}) migrations.

\plot{amormo}{width=6in,height=2.25in}{Antialiased modeling. Left
corresponds to the flat-dip term. Right corresponds to the steep-dip
term.}
\plot{amormi}{width=6in,height=2.25in}{Antialiased
migration. Left corresponds to the flat-dip term. Right corresponds to
the steep-dip term.}
\plot{amormm}{width=6in,height=2.25in}{Antialiased
modeling and migration. Left is the superposition of the flat-dip and
steep-dip modeling. Right is superposition of the flat-dip and
steep-dip migration.}

\begin{comment}
The efficiency of the antialiased migration is compared in the CPU
time chart in Figure \ref{fig:amochp}.  The test data set included 500
by 250 data points with $\Delta t=0.004$ sec, and $\Delta x = 25$ m.
The figure shows that the performance of the reciprocity antialiasing
increases with increase of the migration velocity. This behavior can
be explained by the fact that high-velocity migration hyperbolas
require a smaller number of expensive computations in the steep
(aliased) parts. It allows us to expect a high performance of the
method in application to the curvilinear operators with limited
aperture (dip moveout, azimuth moveout, shot continuation).

%\plot{amochp}{height=1.5in}{CPU time of migration programs
%on HP 9000-735 versus the constant migration velocity used in the
%experiment.}
\end{comment}

\subsection{3-D antialiasing}
\inputdir{imp}

The proposed method of antialiasing is easily generalized to the case
of a three-dimensional integral operator. In this case, we need to
consider three different parameterizations: $t(x,y)$, $x(t,y)$, and
$y(t,x)$ and switch from one of them to another according to the rule:
\begin{itemize}
\item if $\Delta t \geq {{\Delta x} \, {|\partial t / \partial x|}}$
and      $\Delta t \geq {{\Delta y} \, {|\partial t / \partial y|}}$,
use $t(x,y)$,
\item if $\Delta x \geq {{\Delta t} \, {|\partial x / \partial t|}}$
and      $\Delta x \geq {{\Delta y} \, {|\partial x / \partial y|}}$,
use $x(t,y)$,
\item if $\Delta y \geq {{\Delta t} \, {|\partial y / \partial t|}}$
and      $\Delta y \geq {{\Delta x} \, {|\partial y / \partial x|}}$,
use $y(t,x)$.
\end{itemize}

Following \cite{GEO66-02-06540666}, I illustrate 3-D
antialiasing by applying prestack time migration on a single input
trace. The results are shown in Figures~\ref{fig:imp-noa},
\ref{fig:imp-aal} and \ref{fig:imp-all}. The result without any
antialiasing protection (Figure~\ref{fig:imp-noa}) contains clearly
visible aliasing artifacts caused by the steeply dipping parts of the
operator. Antialiasing by temporal filtering
(Figure~\ref{fig:imp-aal}) removes the artifacts but also attenuates
the steeply dipping events. Antialiasing by the proposed reciprocal
parameterization (Figure~\ref{fig:imp-all}) removes the aliasing
artifacts while preserving the steeply dipping events and the image
resolution.

\plot{imp-noa}{width=6in}{Prestack 3-D time migration of a single
  input trace. Top: time slice at 1 s. Bottom: vertical slice. No
  antialiasing protection has been applied. As a result, aliasing
  artifacts are clearly visible in the image.}

\plot{imp-aal}{width=6in}{Prestack 3-D time migration of a single
  input trace. Top: time slice at 1 s. Bottom: vertical slice.
  Antialiasing by temporal filtering has been applied. Aliasing artifacts
  are removed, steeply dipping events are attenuated.}

\plot{imp-all}{width=6in}{Prestack 3-D time migration of a single
  input trace. Top: time slice at 1 s. Bottom: vertical slice.  The
  proposed reciprocal antialiasing has been applied. Aliasing
  artifacts are removed, steeply dipping events are preserved.}

\section{Conclusions}
%%%%
I have introduced a new method of antialiasing integral operators,
modified from Hale's approach to antialiased dip moveout. The method
compares favorably with the popular temporal filtering technique. The
main advantages are:
\begin{enumerate}
\item Accurate handling of variable operator dips.
\item Consequent preservation of the high-frequency part of the data
spectrum, leading to a higher resolution.
\item Easy control of operator amplitudes.
\item Easy generalization to 3-D.
\end{enumerate}
The method possesses a sufficient numerical efficiency in practical
implementations. Its most appropriate usage is for antialiasing
operators with analytically computed summation paths, such as prestack
time migration, dip moveout, azimuth moveout, and shot continuation.

\section{Acknowledgments}

I thank Biondo Biondi for many helpful discussions. The financial
support for this work was provided by the sponsors of the Stanford
Exploration Project.

\bibliographystyle{seg}
\bibliography{antial,SEG,SEP2}

