\section{ASYMPTOTIC INVERSION: RECONSTRUCTING THE DISCONTINUITIES}
%%%%%%%%%%%%%%%%%%%%%%%%%%%%%%%%%%%%%%%%%%%%%%%%%%%%%%%%%%%%%%%%

Mathematical analysis of the inverse problem for operator
(\ref{eqn:operator}) shows that only in rare cases can we obtain an
analytically exact inversion. A well-known example is the Radon
transform, which has acquired a lot of different aliases in
geophysical literature: slant stack, tau-p transform, plane wave
decomposition, and controlled directional reception (CDR) transform
\cite[]{slant}. In this case,
\begin{eqnarray}
\theta(x;t,y) & = & t+x\,y\;,
\label{eqn:radont}\\
 w(x;t,y) & = & 1\;.
\label{eqn:radonw}
\end{eqnarray} 
\par
Radon obtained a result similar to the theoretical inversion of
operator (\ref{eqn:operator}) with the summation path
(\ref{eqn:radont}) and the weighting function (\ref{eqn:radonw}) in
1917, but his result was not widely known until the development of
computer tomography. According to \cite{radon}, the inverse operator
has the form
\begin{equation}
M(z,x) ={\bf A^{-1}}[S(t,y)]= |{\bf D}|^m\;\int
\widehat{w}\,S(\widehat{\theta}(y;z,x),y)\,dy\;,
\label{eqn:radon}
\end{equation}
where 
\begin{eqnarray}
\widehat{\theta}(y;z,x) & = & z-x\,y\;,
\\
\widehat{w}  & = & {1\over {\left(2\,\pi\right)^m}}\;,
\end{eqnarray} 
$|{\bf D}|$ is a one-dimensional convolution operator with the
spectrum $|\omega|$:
\begin{equation}
  \label{eq:rho}
  |{\bf D}| \left[U(z,x)\right] = \frac{1}{2\,\pi}\,
  \int U(\xi,x) \int |\omega|\,e^{i \omega (z-\xi)} d \omega \, d \xi\;,
\end{equation}
 and $m$ is the dimensionality of $x$ and $y$
(usually 1 or 2).  In Russian geophysical literature, a similar result
for the inversion of the CDR transform was published by \cite{nakham}.
\par
Extension of Radon's result to the general form of integral operator
(\ref{eqn:operator}) ({\em generalized Radon transform}) is possible
via asymptotic analysis of the inverse problem. In the general case,
\cite{beylkin} and \cite{goldin} have shown that asymptotic 
inversion can
reconstruct discontinuous parts of the model. These are the parts
responsible for the asymptotic behavior of the model at high
frequencies. Since the discontinuities are associated with wavefronts
and reflection events at seismic sections, there is a certain
correspondence between asymptotic inversion and such standard goals of
seismic data processing as kinematic equivalence and amplitude
preservation.
\par
The main theorem of asymptotic inversion can be formulated as follows
\cite[]{goldin}. The leading-order discontinuities in $M$ are
reconstructed by an integral operator of the form
\begin{equation}
\widehat{M}(z,x)={\bf \widehat{A}}[S(t,y)]=
|{\bf D}|^m\;\int \widehat{w}(y;z,x)\,S(\widehat{\theta}(y;z,x),y)\;dy\;,
\label{eqn:inverse}
\end{equation}
where the summation path $\widehat{\theta}$ is obtained simply by
solving the equation
\begin{equation}
z=\theta(x;t,y)
\label{eqn:summ}
\end{equation}
for $t$ (if such an explicit solution is possible). The correctly
chosen summation path reconstructs the geometry of the
discontinuities. To recover the amplitude, we must choose the correct
weighting function, which is constrained by the equation
\cite[]{beylkin,goldin}
\begin{equation}
w\,\widehat{w}={1\over{\left(2\,\pi\right)^m}} \, 
{\sqrt{\left|F\,\widehat{F}\right|\,
\left|\partial \widehat{\theta} \over \partial z\right|^m}}\;,
\label{eqn:whatw}
\end{equation} 
where
\begin{eqnarray}
F & = & {\partial \theta \over \partial t}\,
{\partial^2 \theta \over \partial x\, \partial y} -
{\partial \theta \over \partial y}\,
{\partial^2 \theta \over \partial x\, \partial t}\;, 
 \\
\widehat{F} & = & {\partial \widehat{\theta} \over \partial z}\,
{\partial^2 \widehat{\theta} \over \partial x\, \partial y} -
{\partial \widehat{\theta} \over \partial x}\,
{\partial^2 \widehat{\theta} \over \partial y\, \partial z}\;. 
\end{eqnarray} 
The solution assumes that differential forms $F$ and $\widehat{F}$
exist and are bounded and non-vanishing\footnote{This requirement is
related to the requirement for the normal $\mathbf{A}^{T}\,\mathbf{A}$
operator, inroduced in the next section, to be a pseudo-differential
operator \cite[]{wong}.  Situations where this condition is violated
require a special consideration \cite[]{SEG-1996-0359,stolk}.}.  In
the multi-dimensional case $(m \geq 2)$, they are replaced by the
determinants of the corresponding matrices. To ensure the asymptotic
inversion, equation (\ref{eqn:whatw}) must be satisfied at least in
the vicinity of the {\em stationary points} of integral
(\ref{eqn:operator}).  Those are the points where the summation path
of the form (\ref{eqn:summ}) is tangent to the traveltimes of the
actual events on the transformed model.
%A
%brief review of the asymptotic inverse theory is included in Appendix A.
%\par
In the case of the Radon transform,
$\left|F\,\widehat{F}\right|=\left|\partial \widehat{\theta} \over
\partial z\right|=1$, and the asymptotic inverse coincides with the
exact inversion.

%%% Local Variables: 
%%% mode: latex
%%% TeX-master: t
%%% TeX-master: t
%%% End: 
