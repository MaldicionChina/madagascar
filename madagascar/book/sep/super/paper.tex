\published{SEP report, 93, 133-136, (1996)}


\footer{SEP--93}
\righthead{Super resolution}
\lefthead{Claerbout}

\title{
	A prospect for super resolution
        }
%\keywords{velocity, modeling, NMO }
\email{jon@sep.stanford.edu}
\author{Jon Claerbout}
\def\eq{\quad =\quad}
\maketitle

\section{INTRODUCTION}

Wouldn't it be great if I could take signals of
10-30 Hz bandwidth
from 100 different offsets and construct a zero-offset trace with 5-100 Hz
bandwidth?
This would not violate Shannon's sampling theorem
which theoretically allows us to have a transform
from 100 signals of 20 Hz bandwidth to one signal
at 2000 Hz bandwidth.
The trouble is that simple NMO is not such a transformation.
Never-the-less,
if the different offsets really did give us any extra information,
we should be able to put the information into extra bandwidth.
Let us consider noise free synthetic data and see
if we can come up with a model where this could happen.

\section{FITTING FRAMEWORK}

\par
The operator of interest is the one that creates many offsets
of seismic data from a zero-offset model space.
\begin{description}
\item [$\mathbf z$]
	is a white seismic trace (model) at zero-offset
\item [$\mathbf d_j$]
	is a red seismic trace (data) at nonzero-offset $x_j$
\item [$\mathbf L$]	is a seismic band pass filter
\item [$\mathbf H_j$]
			sprays along hyperbola
				using a known, rough  $v(z)$ 
\item [$\overline{\mathbf H}_j$]
			sprays 
				using a known, smooth $\overline{v}(z)$
\end{description}
The operator of interest is the one that transforms
$\mathbf z$ to all the data
$\mathbf d_j$
at all of the offsets $x_j$.

\begin{equation}
\mathbf d_j \eq  \mathbf L \mathbf H_j \mathbf z
\end{equation}
\par
Here is a trivial idea:
Estimates $\hat{\mathbf z}_j$
of $\mathbf z$ from data
$\mathbf d_j$
at different offsets $x_j$
have different spectral bands because of NMO stretch.
Wide offsets create low frequency.
Trouble is, these low frequencies add little spectral bandwidth.
We want extra high frequencies too.

\par
We know a simple two-step process where
one offset can be obtained from another:
First moveout for one offset.  Then inverse moveout for the other offset.
Whenever such offset continuation works,
extra offsets cannot bring us extra information.
Extra traces give only redundancy.

\par
Inversion theory says if the transformation has no null space
we should be able to solve for everything.
Since in practice we cannot seem to obtain that extra bandwidth,
it seems that
the operator $\mathbf L \mathbf H_j $
has a large null space,
about equal in size to the trace length times
(the number of offsets minus one).



\section{ROUGH VELOCITY(z)}

Taking velocity to be a rough (bumpy) function of depth,
different offset traces might be fundamentally different
thus providing different information
(i.e.~more information hence potentially more bandwidth).
The bumpy velocity model
seems artificial since it requires
to be known a rough velocity as a function depth.
Never-the-less, the idea could be helpful
because
we sometimes have well logs,
or we might later
learn how to bootstrap our velocity estimate
from a smooth velocity to a rougher one.

\par
Many people think about rough {\it impedance}.
Here we consider a rough {\it interval velocity}.


\section{ROUGH V(z) MAKES TAU(t) MULTIVALUED}
\inputdir{tau}

According to the Dix approximation,
travel time $t(\tau)$ is a unique function of
vertical travel time $\tau$ because
\begin{equation}
t^2 \eq \tau^2 + x^2 / v(\tau)^2
\end{equation}
The reverse is not true, however,
$\tau(t)$ can be a multivalued function of $t$,
and is especially likely to be so
where $v(\tau)$ is a rough function of $\tau$.
When $\tau(t)$ is a multivalued function of $t$
the process of offset continuation breaks down.
Then extra offsets are providing extra information.
We don't yet know if the extra information is a small amount
or a large amount or whether that extra information
is uniformly or locally distributed.
Figure \ref{fig:superres} shows an example.

\plot{superres}{width=6in,height=8.4in}{
	Right shows $t(\tau)$ for many offsets.
	}
%\activeplot{moveout}{width=6in}{.} { From BEI }

\par
Figure \ref{fig:superres} shows two kinds of multivaluedness
in the transformation.
First is the familiar kind that arises whenever $dv/dz > 0$
where travel times of shallow waves cross those of deep waves.
Let us place a line through the broad maxima in $t(\tau)$
at about $t=2.5\tau$ for all $x$.
In a constant velocity earth, the ratio $t/\tau=2.5$
corresponds to a propagation angle $\cos\theta = \tau/t$ or
about $\theta = 66^\circ$.
Thus, a wave with average angle greater than
$\theta = 66^\circ$
generally arrives at the same time and offset
as another wave with an average angle less than
$\theta = 66^\circ$.

\par
The second
way of being multivalued
is less familiar
and hence more interesting,
the roughness in the $t(\tau)$ transformation.
We see this roughness does give rise to multivaluedness.
Disappointingly, the multivaluedness
is not found everywhere but mainly along the
$\theta = 66^\circ$
trend.
We have not yet answered how much extra information we can obtain from this.
Clearly though, if multivaluedness is what makes different offsets
give us different information,
it is along this ``mute-line''
$\theta = 66^\circ$
trend where we must look.

\par
Let us find the high frequency.
Where does an observable (low) frequency on the $t$ axis
map to a high frequency on the $\tau$ axis?
It happens where a long region on the $t$ axis
maps to a short region on the $\tau$ axis,
in other words, where the slope $dt/d\tau$ is greatest.
This is the opposite of usual NMO
in the neighborhood of the diagonal asymptote
in Figure \ref{fig:superres}
where $dt/d\tau<1$.
From the figure,
we see the possibility for frequency boosting
does not arise from the roughness in velocity
but just beneath the water bottom at any offset,
i.e., at the greatest angles.
Since $dt/d\tau$ is negative there,
it gives a kind of upside-down image.
To understand this image, think of head waves where
the deepest layer is fastest and hence has the earliest arrival
with
{\it shallower}
layer arrivals coming
{\it later}.
\par
It is possible the Dix approximation is breaking down here,
a concern that requires further study.
Accurate
{\it reflection} seismograms in this region
are easy to make with the phase shift method.
Getting correct head waves is more complicated.

\section{FURTHER STEPS}
Each offset $x_j$ allows us to make a different estimate of
the earth model $\mathbf z_j$.
There are two possiblities:

i.e. $\mathbf z_j \ne \mathbf z_{j+1}$.
\begin{equation}
\hat {\mathbf z}_j \eq \mathbf H_j' \mathbf L' \mathbf d_j
\end{equation}
\begin{equation}
\hat {\overline{\mathbf z}}_j \eq \overline{\mathbf H}_j' \mathbf L' \mathbf d_j
\end{equation}
We should plot $\hat {\mathbf z}_j$ as a function of $j$.
We should also plot
$\hat {\mathbf z}_j - \hat {\mathbf z}_{j-1}$
as a function of $j$
and see if we can find any higher temporal frequencies.

