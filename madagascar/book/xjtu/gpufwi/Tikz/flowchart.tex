


% Define block styles
\tikzstyle{decision} = [diamond, draw, fill=blue!20, 
    text width=4.5em, text badly centered, node distance=2.8cm, inner sep=0pt]
\tikzstyle{block} = [rectangle, draw, fill=blue!20, 
    text width=14em, text centered, rounded corners, minimum height=2em]
\tikzstyle{blocksmall} = [rectangle, draw, fill=blue!20, node distance=4cm,
    text width=6em, text centered, rounded corners, minimum height=2em]
\tikzstyle{line} = [draw, -latex']

\begin{figure}
  \centering 
\begin{tikzpicture}[node distance = 3.5cm, auto]
    % Place nodes
    \node [blocksmall] (init) {initialize with starting model};
    \node [decision, below of=init] (iter) {k$<$niter?};
    \node [blocksmall, left of=iter](output) {output FWI result};   
    \node [block, below of=iter] (forw) {1)generate synthetic seismogram via modeling, 2) save the effective boundaries, and 3) compute the residual wavefield};
    \node [block, below of=forw] (back) {1)reconstruct source wavefield with saved boundaries, 2)back propagate residual wavefield, and 3) calculate the gradient}; 
    \node [decision, below of=back] (is) {is$<$ns?};
    \node [blocksmall,left of=is] (loop) {loop over shots: is++};
    \node [blocksmall, right of=is] (cg) {calculate $\beta_k$ and the conjugate gradient};
    \node [blocksmall, right of=cg] (eps) {estimate trial stepsize $\epsilon$ and a test velocity};
    \node [blocksmall, above of=eps] (est) {estimate stepsize $\alpha_k$: redo forward modeling (ns shots)};
    \node [blocksmall, above of=est] (update) {update velocity model, k++};
    % Draw edges
    \path [line] (init) -- (iter);
    \path [line] (iter) --node [near start] {Yes} (forw);
    \path [line] (forw) -- (back);
    \path [line] (back) -- (is);
    \path [line] (is) --node [near start] {Yes} (loop);
    \path [line] (loop) |- (forw);
    \path [line] (is)--node[near start]{No} (cg);
    \path [line] (cg)--(eps);
    \path [line] (eps)--(est);
    \path [line] (est)--(update);
    \path [line] (update)|-(iter);
    \path [line] (iter)--node[near start]{No}(output);
\end{tikzpicture}
\end{figure}
