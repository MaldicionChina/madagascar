\section{Introduction}

Seismic waves are described by the elastic wave equation with P- and S-waves intrinsically coupled. An elastic migration or inversion
program \old{must}\new{should} be able to handle both wave modes. Normally the \old{two}\new{P and S} modes are separated\old{ from one another,} and each is treated independently.
\new{Otherwise, the two modes are mixed on all wavefield components and cause crosstalk and image artifacts \cite[]{yan.sava:2009}.}
In isotropic media, far-field P- and S-wave modes can be separated by taking the divergence and curl \old{of}\new{in} the extrapolated elastic wavefield.
It is well known that a shear wave passing through an anisotropic medium \old{splits}\new{can split} into two mutually orthogonal waves \cite[]{crampin:1984}.
Generally the P-wave and the two S-waves \new{in anisotropic materials} are not polarized parallel and perpendicular to the wave vectors
 and \old{do not}\new{can not be} fully separate\new{d} with divergence and curl operations.

To account for seismic anisotropy, wave mode separation concept and approach have been extended\old{ to anisotropic media during}\new{in} the past two decades. 
\cite {dellinger.etgen:1990} generalize divergence and curl to anisotropic media by constructing the separators in the
wavenumber domain, and independently solving the Christoffel equation in each wave propagation direction.
For heterogeneous media, these divergence-like and curl-like separators are rewritten by \cite{yan.sava:2009} as nonstationary spatial filters
 determined by the local polarization vectors. 
\cite{zhang.mcmechan:2010} develop a wavefield decomposition method to separate elastic wavefields into vector P- and S-wave fields
for vertically transverse isotropic (VTI) media.
Alternatively, we may \old{simulate propagation of separated wave modes using the so-called pseudo-pure-mode wave equations,
which can implicitly achieve partial mode separation during wavefield extrapolation, plus correction of polarization deviation}
\new{implicitly achieve partial mode separation during wavefield extrapolation using the so-called
pseudo-pure-mode wave equations, and then obtain completely separated wave modes by correcting the polarization projection deviation of the
pseudo-pure-mode wavefields}
from the isotropic reference\old{ for complete mode separation} \cite[]{cheng.kang:2012,cheng.kang:2013}.
Although these studies provide significant insights into wave mode separation in anisotropic media,
\old{great challenges remains for computational implementation}\new{many challenges remain, especially in the computational implementation}
if the proposed approaches are directly used in practice.
For example, mode separation using nonstationary filtering is computationally expensive, especially in 3D.
To improve efficiency, \cite {yan.sava:2011} present a mixed-domain algorithm that resembles the phase-shift plus interpolation (PSPI) scheme
 from one-way wave equation migration \cite[]{gazdag:1984}.
The compromise between accuracy and cost requires to determine the minimal reference models
 that best represent the true model space, and the choice of the models is case dependent. On the other hand, \cite{zhang.mcmechan:2010}'s
 wavenumber-domain vector decomposition approach is effective when the model can be separated into distinct geologic units.
 \new{In addition, spectral methods were proposed to provide solutions which can completely avoid the crosstalk between the qP and qS modes
in wavefield modeling and reverse-time migration (RTM) \cite[]{etgen:2009,liu:2009,chu:2011,pestana:2011,fomel:2013,song:2013a}. However, these pure-mode solutions fail to provide accurate amplitudes for qP- and qS-waves.
For true-amplitude ERTM in anisotropic media,
effective mode separation and decomposition are highly required before applying
 the imaging condition to the extrapolated elastic wavefields \cite[]{zhang.mcmechan:2011}.} 

 In this paper, we \new{respectively} propose fast algorithms for \new{elastic} wave mode separation and vector decomposition in \new{3D} heterogeneous transverse isotropic (TI) media.
 First, we give a brief review of the underlying principles. Then we present space-wavenumber-domain operations for mode separation
 and vector decomposition in the form of Fourier integral operators, and discuss how to construct efficient algorithms using low-rank approximation
\cite[]{engquist.ying:2008}\old{ of them}. 
 At the end, we test efficiency and accuracy of the proposed method using synthetic models of increasing complexity.
