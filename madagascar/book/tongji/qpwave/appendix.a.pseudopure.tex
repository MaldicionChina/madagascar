\append{Pseudo-pure-mode qP-wave equation for vertical TI and orthorhombic media}
For vertical TI and orthorhombic media, the stiffness tensors have the same null components and can
be represented in a two-index notation \cite[]{musgrave} often called the “Voigt notation” as
\begin{equation}
\mathbf{C} =
\begin{pmatrix}C_{11} &C_{12} &C_{13} &0 &0 &0 \cr
         C_{12} &C_{22} &C_{23} &0 &0 &0 \cr
         C_{13} &C_{23} &C_{33} &0 &0 &0 \cr
         0& 0&  0 & C_{44} &0 &0 \cr
         0& 0&  0 &0 & C_{55} &0 \cr
         0& 0&  0 &0 &0 &C_{66}\end{pmatrix}.
\end{equation}
For vertical orthorhombic tensor, the nine coefficients are indepedent, but the VTI tensor has only 
five independent coefficients with $C_{12}=C_{11}-2C_{66}$, $C_{22}=C_{11}$, $C_{23}=C_{13}$ and $C_{55}=C_{44}$.
The stability condition requires these parameters to satisfy the corresponding constraints \cite[]{helbig:1994, tsvankin:2001}.
According to equations~\ref{eq:elastic1} and \ref{eq:gama}, the full elastic wave equation without the source terms is expressed as:
\begin{equation}
\begin{split}
\rho\frac{\partial^2{u_x}}{\partial t^2} &= C_{11}\frac{\partial^2{u_x}}{\partial x^2}
                                         + C_{66}\frac{\partial^2{u_x}}{\partial y^2}
                                         + C_{55}\frac{\partial^2{u_x}}{\partial z^2}
                                         +(C_{12}+C_{66})\frac{\partial^2{u_y}}{{\partial x}{\partial y}}
                                         +(C_{13}+C_{55})\frac{\partial^2{u_z}}{{\partial x}{\partial z}}, \\
\rho\frac{\partial^2{u_y}}{\partial t^2} &= C_{66}\frac{\partial^2{u_y}}{\partial x^2}
                                         + C_{22}\frac{\partial^2{u_y}}{\partial y^2}
                                         + C_{44}\frac{\partial^2{u_y}}{\partial z^2}
                                         +(C_{12}+C_{66})\frac{\partial^2{u_x}}{{\partial x}{\partial y}}
                                         +(C_{23}+C_{44})\frac{\partial^2{u_z}}{{\partial y}{\partial z}}, \\
\rho\frac{\partial^2{u_z}}{\partial t^2} &= C_{55}\frac{\partial^2{u_z}}{\partial x^2}
                                         + C_{44}\frac{\partial^2{u_z}}{\partial y^2}
                                         + C_{33}\frac{\partial^2{u_z}}{\partial z^2} 
                                         +(C_{13}+C_{55})\frac{\partial^2{u_x}}{{\partial x}{\partial z}}
                                         +(C_{23}+C_{44})\frac{\partial^2{u_y}}{{\partial y}{\partial z}}.
\end{split}
\end{equation}
Thus the corresponding Christoffel matrix in wavenumber domain satisfies
\begin{equation}
\widetilde{\mathbf{\Gamma}}=
 \begin{pmatrix} C_{11}{k_x}^2 + C_{66}{k_y}^2 + C_{55}{k_z}^2  & (C_{12}+C_{66}){k_x}{k_y} &(C_{13}+C_{55}){k_x}{k_z} \cr
         (C_{12}+C_{66}){k_x}{k_y} & C_{66}{k_x}^2+C_{22}{k_y}^2+C_{44}{k_z}^2 &(C_{23}+C_{44}){k_y}{k_z} \cr
         (C_{13}+C_{55}){k_x}{k_z} & (C_{23}+C_{44}){k_y}{k_z} & C_{55}{k_x}^2+C_{44}{k_y}^2+C_{33}{k_z}^2\end{pmatrix}.
\end{equation}
According to equation~\ref{eq:tansChrisM}, the Christoffel matrix after the similarity transformation is given as, 
\begin{equation}
\label{eq:transGama}
\widetilde{\overline{\mathbf{\Gamma}}}=
\begin{pmatrix} C_{11}{k_x}^2+C_{66}{k_y}^2+C_{55}{k_z}^2 &(C_{12}+C_{66}){k_x}^2 &(C_{13}+C_{55}){k_x}^2 \cr
         (C_{12}+C_{66}){k_y}^2 & C_{66}{k_x}^2+C_{22}{k_y}^2+C_{44}{k_z}^2 &(C_{23}+C_{44}){k_y}^2 \cr
         (C_{13}+C_{55}){k_z}^2 & (C_{23}+C_{44}){k_z}^2 & C_{55}{k_x}^2+C_{44}{k_y}^2+C_{33}{k_z}^2\end{pmatrix}.
\end{equation}
Finally, we obtain the pseudo-pure-mode qP-wave equation (i.e., equation~\ref{eq:pseudo})
by inserting equation~\ref{eq:transGama} into equation~\ref{eq:tansChris} and applying an
 inverse Fourier transform.
