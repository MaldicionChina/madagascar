\section{Zero finder (fzero.c)}




\subsection{{sf\_zero}}
Returns the zero (root) of the input function, $f(x)$ in a specified interval $[a,b]$.

\subsubsection*{Call}
\begin{verbatim}b = sf_zero ((*func)(float), a, b, fa, fb, toler, verb);\end{verbatim}

\subsubsection*{Definition}
\begin{verbatim}
float sf_zero (float (*func)(float) /* function f(x) */, 
               float a, float b     /* interval */, 
               float fa, float fb   /* f(a) and f(b) */,
               float toler          /* tolerance */, 
               bool verb            /* verbosity flag */)
/*< Return c such that f(c)=0 (within tolerance). 
  fa and fb should have different signs. >*/
{
    float c, fc, m, s, p, q, r, e, d;
    char method[256];

   ...
    return b;
}
\end{verbatim}

\subsubsection*{Input parameters}
\begin{desclist}{\tt }{\quad}[\tt (*func)(float)]
   \setlength\itemsep{0pt}
   \item[(*func)(float)] function, the root of which is required. Must be of type \texttt{sf\_double\_complex}.  
   \item[a]     lower limit of the interval (\texttt{float}).
   \item[b]     upper limit of the interval (\texttt{float}).
   \item[fa]    function value at the lower limit (\texttt{float}).
   \item[fb]    function value at the upper limit (\texttt{float}).
   \item[toler] error tolerance (\texttt{float}).
   \item[verb]  verbosity flag (\texttt{bool}).
\end{desclist}

\subsubsection*{Output}
\begin{desclist}{\tt }{\quad}[\tt ]
   \setlength\itemsep{0pt}  
   \item[b] root of the input function. It is of type \texttt{float}.
\end{desclist}


