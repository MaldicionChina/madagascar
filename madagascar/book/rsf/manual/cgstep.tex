\section{Claerbout's conjugate-gradient iteration (cgstep.c)}




\subsection{{sf\_cgstep}}
Evaluates one step of the conjugate gradient method iteration.

\subsubsection*{Call}
\begin{verbatim}sf_cgstep(forget, nx, ny, x, g, rr, gg);\end{verbatim}

\subsubsection*{Definition}
\begin{verbatim}
void sf_cgstep( bool forget     /* restart flag */, 
                int nx, int ny  /* model size, data size */, 
                float* x        /* current model [nx] */, 
                const float* g  /* gradient [nx] */, 
                float* rr       /* data residual [ny] */, 
                const float* gg /* conjugate gradient [ny] */) 
/*< Step of conjugate-gradient iteration. >*/
{
   ...
}
\end{verbatim}

\subsubsection*{Input parameters}
\begin{desclist}{\tt }{\quad}[\tt forget]
   \setlength\itemsep{0pt}
   \item[forget] restart flag (\texttt{bool}). 
   \item[nx]     size of the model (\texttt{int}). 
   \item[ny]     size of the data (\texttt{int}). 
   \item[g]      the gradient (\texttt{const float*}).  
   \item[rr]     the data residual (\texttt{float*}).  
   \item[gg]     the conjugate gradient (\texttt{const float*}).  
\end{desclist}

\subsubsection*{Output}
\begin{desclist}{\tt }{\quad}[\tt ]
   \setlength\itemsep{0pt}
   \item[c.r] real part of the complex number. It is of type \texttt{double}.
\end{desclist}




\subsection{{sf\_cgstep\_close}}
Frees the space allocated for the conjugate gradient step calculation.

\subsubsection*{Call}
\begin{verbatim}sf_cgstep_close ();\end{verbatim}

\subsubsection*{Definition}
\begin{verbatim}
void sf_cgstep_close (void) 
/*< Free allocated space. >*/ 
{
   ...
}
\end{verbatim}





