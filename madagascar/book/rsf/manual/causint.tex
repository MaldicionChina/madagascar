\section{Causal integration (causint.c)}
This causal integration operator is defined by
\begin{gather*}
	y = Lx,  \qquad\textrm{with}\quad y_t \leftarrow \sum_{\tau=0}^tx_\tau.
\intertext{Its adjoint is}
	x = L^*y,\qquad\textrm{with}\quad x_t \leftarrow \sum_{\tau=t}^{T-1}y_\tau,
\end{gather*}
where $T$ is the total number of samples of $x$.




\subsection{{sf\_causint\_lop}}

\subsubsection*{Call}
\begin{verbatim}sf_causint_lop (adj, add, nx, ny, x, y);\end{verbatim}

\subsubsection*{Defintion}
\begin{verbatim}
sf_causint_lop (adj, add, nx, ny, x, y)
/*< linear operator >*/
{
   ...
}
\end{verbatim}

\subsubsection*{Input parameters}
\begin{desclist}{\tt }{\quad}[\tt add]
   \setlength\itemsep{0pt}
   \item[adj] adjoint flag (\texttt{bool}). If \texttt{true}, then the adjoint is computed, i.e.~$x\leftarrow L^*y$ or $x\leftarrow x+L^*y$. 
   \item[add] addition flag (\texttt{bool}). If \texttt{true}, then $y\leftarrow y+Lx$ or $x\leftarrow x+L^*y$.  
   \item[nx]  size of \texttt{x} (\texttt{int}). 
   \item[ny]  size of \texttt{y} (\texttt{int}). 
   \item[x]   input data or output (\texttt{float*}).
   \item[y]   output or input data (\texttt{float*}).
\end{desclist}


