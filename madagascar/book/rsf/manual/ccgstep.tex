\section[CG iteration (complex data) (cgstep.c)]{Claerbout's conjugate-gradient iteration for complex numbers (cgstep.c)}




\subsection{{sf\_ccgstep}}\label{sec:sf_ccgstep}
Evaluates one step of the Claerbout's conjugate-gradient iteration for complex numbers.

\subsubsection*{Call}
\begin{verbatim}sf_ccgstep(forget, nx, ny, x, g, rr, gg);\end{verbatim}

\subsubsection*{Definition}
\begin{verbatim}
void sf_ccgstep( bool forget          /* restart flag */, 
                 int nx               /* model size */, 
                 int ny               /* data size */, 
                 sf_complex* x        /* current model [nx] */,  
                 const sf_complex* g  /* gradient [nx] */, 
                 sf_complex* rr       /* data residual [ny] */,
                 const sf_complex* gg /* conjugate gradient [ny] */) 
/*< Step of Claerbout's conjugate-gradient iteration for complex operators. 
    The data residual is rr = A x - dat
>*/
{
   ...
}
\end{verbatim}

\subsubsection*{Input parameters}
\begin{desclist}{\tt }{\quad}[\tt forget]
   \setlength\itemsep{0pt}
   \item[forget] restart flag (\texttt{bool}).  
   \item[nx]     size of the model (\texttt{int}).  
   \item[ny]     size of the data (\texttt{int}).  
   \item[x]      current model (\texttt{sf\_complex*}).  
   \item[g]      the gradient. Must be of type \texttt{const sf\_complex*}.  
   \item[rr]     the data residual (\texttt{sf\_complex*}).  
   \item[gg]     the conjugate gradient. Must be of type \texttt{const sf\_complex*}.  
\end{desclist}




\subsection{{sf\_ccgstep\_close}}
Frees the space allocated for \hyperref[sec:sf_ccgstep]{\texttt{sf\_ccgstep}}.

\subsubsection*{Call}
\begin{verbatim}sf_ccgstep_close();\end{verbatim}

\subsubsection*{Definition}
\begin{verbatim}
void sf_ccgstep_close (void) 
/*< Free allocated space. >*/ 
{
   ...
}
\end{verbatim}




\subsection{{dotprod}}
Returns the dot product of two complex numbers or the sum of the dot products if the are two arrays of complex numbers.

\subsubsection*{Call}
\begin{verbatim}prod = dotprod (n, x, y);\end{verbatim}

\subsubsection*{Definition}
\begin{verbatim}
static sf_double_complex dotprod (int n, const sf_complex* x, 
                                         const sf_complex* y)
/* complex dot product */
{
   ...
}
\end{verbatim}

\subsubsection*{Input parameters}
\begin{desclist}{\tt }{\quad}[\tt ]
   \setlength\itemsep{0pt}
   \item[n]  size of the array of complex numbers (\texttt{int}).  
   \item[x]  a complex number (\texttt{sf\_complex*}).  
   \item[y]  a complex number (\texttt{sf\_complex*}).  
\end{desclist}

\subsubsection*{Output}
\begin{desclist}{\tt }{\quad}[\tt ]
   \setlength\itemsep{0pt}  
   \item[prod] dot product of the complex numbers. It is of type \texttt{static sf\_double\_complex}.
\end{desclist}



