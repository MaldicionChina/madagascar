\author{Maurice the Aye-Aye}
%%%%%%%%%%%%%%%%%%%%%%%%%%%%
\title{Madagascar tutorial}

\lefthead{Maurice}
\righthead{Tutorial}
\footer{Madagascar Documentation}

\maketitle

\begin{abstract}
  In this tutorial, you will go through different steps required for writing a research paper with reproducible examples. In particular, you will
  \begin{enumerate}
    \item identify a research problem,
    \item suggest a solution,
    \item test your solution using a synthetic example,
    \item apply your solution to field data,
    \item write  a report about your work.
  \end{enumerate}
\end{abstract}

\section{Prerequisites}

Completing this tutorial requires
\begin{itemize}
\item \textsc{Madagascar} software environment available from \\
\url{http://www.ahay.org}
\item \LaTeX\ environment with \texttt{SEGTeX} available from \\ 
\url{http://www.ahay.org/wiki/SEGTeX}
\end{itemize}
To do the assignment on your personal computer, you need to install
the required environments. An Internet connection is required for
access to the data repository.

The tutorial itself is available from the \textsc{Madagascar} repository
by running
\begin{verbatim}
svn checkout http://svn.code.sf.net/p/rsf/code/trunk/book/rsf/school2015/tutorial
\end{verbatim}

\section{Introduction}

In this tutorial, you will be asked to run commands from the Unix
shell (identified by \texttt{bash\$}) and to edit files in a text
editor. Different editors are available in a typical Unix environment
(\texttt{vi}, \texttt{emacs}, \texttt{nedit}, etc.)

Your first assignment:
\begin{enumerate}
\item Open a Unix shell.
\item Change directory to the tutorial directory
\begin{verbatim}
bash$ cd $RSFSRC/book/rsf/school2015/tutorial
\end{verbatim}
\item Open the \texttt{paper.tex} file in your favorite editor, for example by
running
\begin{verbatim}
bash$ nedit paper.tex & 
\end{verbatim}
\item Look at the first line in the file and change the author name from Maurice the Aye-Aye to your name (first things first). 
\end{enumerate}

\section{Warm-up}
\inputdir{.}
This section familiarizes the reader with some basic commands and
file format of Madagascar. From the command line, the most
straightforward way to search for a functionality is through the use of:
\begin{verbatim}
bash$ sfdoc -k "keyword"
\end{verbatim}
Madagascar command typed without any argument gives the documentation
of that command. For example, try typing the following in the command line
\begin{verbatim}
bash$ sfspike
\end{verbatim}
Madagascar programs can be piped and executed from the command line,
for example:
\begin{verbatim}
bash$ sfspike n1=1000 k1=300 | sfbandpass fhi=2 phase=1 | \
sfwiggle pclip=100 title="\s200 Welcome to \c2 RSF" | sfpen
\end{verbatim}
creates a minimum phase wavelet.
 
To convert the command line arguments into a SCons script, first
create a file named "SConstruct" using your favorite text
editor. Start it with a line
\begin{verbatim}
from rsf.proj import *
\end{verbatim}
and end it with
\begin{verbatim}
End() 
\end{verbatim}
The two lines above are essential components of a SCons script. It can
be executed by running
\begin{verbatim}
bash$ scons
\end{verbatim}
under the current directory, but nothing would happen since we have
not included any commands. To convert the command line arguments we
just used into the SCons script, add the following two lines before the
End() argument:
\begin{verbatim}
Flow('min',None,'spike n1=1000 k1=300 | bandpass fhi=2 phase=1 ')
Result('min','min','wiggle pclip=100 title="\s200 Welcome to \c2 RSF" ')
\end{verbatim}
The result can be viewed by running
\begin{verbatim}
bash$ scons view
\end{verbatim}

\section{The RSF file format}
\inputdir{asc2rsf}
\sideplot{model}{width=\textwidth}{Stratigraphic layers overlayed on a
 $V(z)$ model.}

Next we will try to generate a 2D synthetic model from ASCII files.
\begin{enumerate}
\item Change directory to the project directory
\begin{verbatim}
bash$ cd asc2rsf
\end{verbatim}
\item Run
\begin{verbatim}
bash$ scons view
\end{verbatim}
will generate a synthetic model that looks like Figure~\ref{fig:model}
on the screen.
\item To understand how the model is created, examine the SConstruct
  script. ASCII files inp*.asc are created using a python loop.
\item The ASCII files are converted into the RSF file format and then 
interpolated using 1-D cubic spline interpolation:
\begin{verbatim}
bash$ sfdd form=native <inp0.asc | \
sfspline o1=0 d1=0.05 n1=201 fp=0,0 >lay1.rsf
\end{verbatim}
\item The layer files are gathered into a single RSF file by arranging
  them along the second dimension
\begin{verbatim}
bash$ sfcat axis=2 < lay1.rsf lay2.rsf lay3.rsf lay4.rsf > lays.rsf
\end{verbatim}
\item The layered gradient ($V(z)$) velocity model is created
  independently using sfmath, a very useful command in Madagascar that
  performs common mathematical operations on RSF files.
\item The final figure is created by overlaying the layers onto the
  velocity model using the Overlay plotting option.
\end{enumerate}

\section{Exercise: convert command-line arguments into SConstruct}
\inputdir{adapt}
This command-line example is borrowed from Paul Sava. Let us do it step by step:
\begin{itemize}
 \item Create 2D Gaussian function
  \begin{verbatim}
  sfmath output="exp(-(x1*x1+x2*x2)/(2*1.5*1.5))" n1=200 d1=0.1 \
  o1=-10.  n2=200 d2=0.1 o2=-10. | sfput label1=z \
  unit1=km label2=x unit2=km > gg.rsf
  \end{verbatim}

  \item Plot the 2D Gaussian function 
  \begin{verbatim}
  < gg.rsf sfgrey pclip=100 screenratio=1 scalebar=y | xtpen 
  \end{verbatim}


  \item Extract 1D subset from the 2D Gaussian function
  \begin{verbatim}
  < gg.rsf sfwindow n2=1 f2=100 | sfgraph | xtpen
  \end{verbatim}

  \item Create a velocity model including a Gaussian anomaly
  \begin{verbatim}
  < gg.rsf sfscale rscale=-1. | sfadd add=3 > vel.rsf 
  < vel.rsf sfgrey title="" pclip=100 screenratio=1 \
  bias=3 scalebar=y| xtpen
  \end{verbatim}

  \item Compute traveltimes with an eikonal solver
  \begin{verbatim}
  < vel.rsf sfeikonal zshot=-10 yshot=0 > fme.rsf
  < fme.rsf sfcontour title="" nc=200 screenratio=1 |xtpen
  \end{verbatim}

  \item Compute rays and wavefronts
  \begin{verbatim}
  < vel.rsf sfhwt2d xsou=0 zsou=-10 nt=1000 ot=0 dt=0.01 ng=1801 \
  og=-90  dg=0.1 > hwt.rsf
  
  < hwt.rsf sfwindow j1=20 j2=20 | sfgraph title="" yreverse=y \
  screenratio=1   min1=-10 max1=+10 min2=-10 max2=+10 | xtpen
  \end{verbatim}
\end{itemize}
After running the previous command-lines, the Gaussian function is as shown in Figure~\ref{fig:gg}. 
The velocity is Figure~\ref{fig:vel}. With the eikonal solver, you will obtain the compuated 
traveltimes in Figure~\ref{fig:fme}.The rays and the wavefronts can be overlaid in Figure~\ref{fig:combined}. 
\plot{gg}{width=0.6\textwidth}{2D Gaussian function}
\plot{vel}{width=0.6\textwidth}{The resulting velocity model}
\plot{fme}{width=0.6\textwidth}{Computed traveltimes}
\plot{combined}{width=0.6\textwidth}{Rays and wavefronts can be overlaid.}

Your assignment is to convert the command-line arguments into a SCons script, which can be used for reproducible research. The solution can be viewed in adapt/SConstruct.
%\lstset{language=python,numbers=left,numberstyle=\tiny,showstringspaces=false}
%\lstinputlisting[frame=single]{adapt/SConstruct}


\section{Problem}
\inputdir{channel}

\plot{horizon}{width=\textwidth}{Depth slice from 3-D seismic (left) and output of edge detection (right).}

The left plot in Figure~\ref{fig:horizon} shows a depth slice from a 3-D
seismic volume\footnote{Courtesy of Matt Hall (ConocoPhillips Canada
Ltd.)}. You notice a channel structure and decide to extract it using
and edge detection algorithm from the image processing literature
\cite[]{canny}. In a nutshell, Canny's edge detector picks areas of
high gradient that seem to be aligned along an edge. The extracted
edges are shown in the right plot of Figure~\ref{fig:horizon}. The initial
result is not too clear, because it is affected by random
fluctuations in seismic amplitudes. The goal of your research project
is to achieve a better result in automatic channel extraction.

\begin{enumerate}
\item Change directory to the project directory
\begin{verbatim}
bash$ cd channel
\end{verbatim}
\item Run
\begin{verbatim}
bash$ scons horizon.view
\end{verbatim}
in the Unix shell. A number of commands will appear in the shell followed by Figure~\ref{fig:horizon} appearing on your screen. 
\item To understand the commands, examine the script that generated them by opening the \texttt{SConstruct} file in a text editor. Notice that, instead of Shell commands, the script contains rules. 
\begin{itemize}
\item The first rule, \texttt{Fetch}, allows the script to download the input data file \texttt{horizon.asc} from the data server. 
\item Other rules have the form \texttt{Flow(target,source,command)} for generating data files or \texttt{Plot} and  \texttt{Result} for 
generating picture files. 
\item \texttt{Fetch}, \texttt{Flow}, \texttt{Plot}, and \texttt{Result} are defined in \textsc{Madagascar}'s \texttt{rsf.proj} package, which extends the functionality of \href{http://www.scons.org}{SCons}
\cite[]{icassp}.
\end{itemize}
\item To better understand how rules translate into commands, run 
\begin{verbatim}
bash$ scons -c horizon.rsf
\end{verbatim}
The \texttt{-c} flag tells scons to remove the \texttt{horizon.rsf} file and all its dependencies.
\item Next, run
\begin{verbatim}
bash$ scons -n horizon.rsf
\end{verbatim}
The \texttt{-n} flag tells scons not to run the command but simply to display it on the screen. Identify the lines in the \texttt{SConstruct} file that generate the output you see on the screen.
\item Run
\begin{verbatim}
bash$ scons horizon.rsf
\end{verbatim}
Examine the file \texttt{horizon.rsf} both by opening it in a text editor and by running
\begin{verbatim}
bash$ sfin horizon.rsf
\end{verbatim}
How many different \textsc{Madagascar} modules were used to create this file? What are the file dimensions? Where is the actual data stored?
\item Run
\begin{verbatim}
bash$ scons smoothed.rsf
\end{verbatim}
Notice that the \texttt{horizon.rsf} file is not being rebuilt.
\item What does the \texttt{sfsmooth} module do? Find it out by running
\begin{verbatim}
bash$ sfsmooth
\end{verbatim}
without arguments. Has \texttt{sfsmooth} been used in any other \textsc{Madagascar} examples?
\item What other \textsc{Madagascar} modules perform smoothing? To find out, run
\begin{verbatim}
bash$ sfdoc -k smooth
\end{verbatim}
\item Notice that Figure~\ref{fig:horizon} does not make a very good use of the color scale. 
To improve the scale, find the mean value of the data by running
\begin{verbatim}
bash$ sfattr < horizon.rsf
\end{verbatim}
and insert it as a new value for the \texttt{bias=} parameter in the
\texttt{SConstruct} file. Does smoothing by \texttt{sfsmooth} change
the mean value?
\item Save the \texttt{SConstruct} file and run 
\begin{verbatim}
bash$ scons view
\end{verbatim}
to view improved images. Notice that \texttt{horizon.rsf} and \texttt{smoothed.rsf} files are not being rebuilt. SCons is smart enough to know that only the 
part affected by your changes needs to be updated.
\end{enumerate}

As shown in Figure~\ref{fig:smoothed}, smoothing removes random
amplitude fluctuations but at the same broadens the channel and thus
makes the channel edge detection unreliable. In the next part of this
tutorial, you will try to find a better solution by examining a simple
one-dimensional synthetic example.

\plot{smoothed}{width=\textwidth}{Depth slice from Figure~\ref{fig:horizon} after smoothing (left) and output of edge detection (right).}

\lstset{language=python,numbers=left,numberstyle=\tiny,showstringspaces=false}
\lstinputlisting[frame=single]{channel/SConstruct}

\section{1-D synthetic}
\inputdir{local}

\multiplot{2}{step,smooth}{width=0.45\textwidth}{(a) 1-D synthetic to test edge-preserving smoothing. (b) Output of conventional triangle smoothing.}

To better understand the effect of smoothing, you decide to create a
one-dimensional synthetic example shown in Figure~\ref{fig:step}. The
synthetic contains both sharp edges and random noise.  The output of
conventional triangle smoothing is shown in
Figure~\ref{fig:smooth}. We see an effect similar to the one in the
real data example: random noise gets removed by smoothing at the
expense of blurring the edges. Can you do better?

\multiplot{2}{spray,local}{width=0.45\textwidth}{(a) Input synthetic trace duplicated multiple times. (b) Duplicated traces shifted so that each data 
sample gets surrounded by its neighbors. The original trace is in the middle.}

To better understand what is happening in the process of smoothing,
let us convert 1-D signal into a 2-D signal by first replicating the
trace several times and then shifting the replicated traces with
respect to the original trace (Figure~\ref{fig:spray,local}). This
creates a 2-D dataset, where each sample on the original trace is
surrounded by samples from neighboring traces.

Every local filtering operation can be understood as stacking traces
from Figure~\ref{fig:local} multiplied by weights that correspond to
the filter coefficients.

\begin{enumerate}
\item Change directory to the project directory
\begin{verbatim}
bash$ cd ../local
\end{verbatim}
\item Verify the claim above by running
\begin{verbatim}
bash$ scons smooth.view smooth2.view
\end{verbatim}
Open the \texttt{SConstruct} file in a text editor to verify that the first image is computed by \texttt{sfsmooth} and the second 
image is computed by applying triangle weights and stacking. To compare the two images by flipping between them, run
\begin{verbatim}
bash$ sfpen Fig/smooth.vpl Fig/smooth2.vpl
\end{verbatim}
\item Edit \texttt{SConstruct} to change the weight from triangle
\begin{equation}
\label{eq:triangle}
W_T(x) = 1-\frac{|x|}{x_0}
\end{equation}
to Gaussian
\begin{equation}
\label{eq:gaussian}
W_G(x) = \exp{\left(-\alpha\,\frac{|x|^2}{x_0^2}\right)}
\end{equation}
Repeat the previous computation. Does the result change? What is a good value for $\alpha$? 
\item Thinking about this problem, you invent an idea\footnote{Actually, you reinvent the idea of \emph{bilateral} or \emph{non-local} filters
\cite[]{tomasi,gilboa}.}. Why not apply non-linear filter weights that would discriminate between points not only based on their distance
from the center point but also on the difference in function values
between the points. That is, instead of filtering by
\begin{equation}
\label{eq:local}
g(x) = \int f(y)\,W(x-y)\,dy\;,
\end{equation}
where $f(x)$ is input, $g(y)$ is output, and $W(x)$ is a linear weight, you decide to filter by
\begin{equation}
\label{eq:nonlocal}
g(x) = \int f(y)\,\hat{W}\left(x-y,f(x)-f(y)\right)\,dy\;,
\end{equation}
where and $\hat{W}(x,z)$ is a non-linear weight. Compare the two weights by running
\begin{verbatim}
bash$ scons triangle.view similarity.view
\end{verbatim}
The results should look similar to Figure~\ref{fig:triangle,similarity}.
\item The final output is Figure~\ref{fig:nlsmooth}. By examining \texttt{SConstruct}, find how to reproduce this figure.
\item \textbf{EXTRA CREDIT} If you are familiar with programming in C, add 1-D non-local filtering as a new \textsc{Madagascar} module \texttt{sfnonloc}. Ask the instructor for further instructions. 
\end{enumerate}

\multiplot{2}{triangle,similarity}{width=0.45\textwidth}{(a) Linear and stationary triangle weights. (b) Non-linear and non-stationary weights reflecting both distance
between data points and similarity in data values.}

\sideplot{nlsmooth}{width=\textwidth}{Output of non-local smoothing}

Figure~\ref{fig:nlsmooth} shows that non-linear filtering can eliminate random noise while preserving the edges. The problem is solved! Now let us apply the result to our original problem.
 
\lstset{language=c,numbers=left,numberstyle=\tiny,showstringspaces=false}
\lstinputlisting[frame=single]{Mnonlocal.c}

\section{Solution}
\inputdir{channel2}

\begin{enumerate}
\item Change directory to the project directory
\begin{verbatim}
bash$ cd ../channel2
\end{verbatim}
\item By now, you should know what to do next.
\item Two-dimensional shifts generate a four-dimensional volume. Verify it by running
\begin{verbatim}
bash$ scons local.rsf
\end{verbatim}
and
\begin{verbatim}
bash$ sfin local.rsf
\end{verbatim}
View a movie of different shifts by running 
\begin{verbatim}
bash$ scons local.vpl
\end{verbatim}
\item Modify the filter weights by editing \texttt{SConstruct} in a text editor.
Observe your final result by running
\begin{verbatim}
bash$ scons smoothed2.view
\end{verbatim}
\item The file $\texttt{norm.rsf}$ contains the non-linear weights stacked over different shifts. Add a \texttt{Result} statement to  \texttt{SConstruct} that would display
the contents of $\texttt{norm.rsf}$ in a figure. Do you notice anything interesting?
\item Apply the Canny edge detection to your final result and display it in a figure.
\item \textbf{EXTRA CREDIT} Change directory to \verb#../mona# and apply your method to the image of Mona Lisa. Can you extract her smile?
\end{enumerate}

\lstset{language=python,numbers=left,numberstyle=\tiny,showstringspaces=false}
\lstinputlisting[frame=single]{channel2/SConstruct}

\sideplot{smoothed2}{width=0.75\textwidth}{Your final result.}

\inputdir{mona}
\sideplot{mona}{width=0.75\textwidth}{Can you apply your algorithm to Mona Lisa?}

\lstset{language=python,numbers=left,numberstyle=\tiny,showstringspaces=false}
\lstinputlisting[frame=single]{mona/SConstruct}

\section{Writing a report}

\begin{enumerate}
\item Change directory to the parent directory
\begin{verbatim}
bash$ cd ..
\end{verbatim}
This should be the directory that contains \texttt{paper.tex}.
\item Run
\begin{verbatim}
bash$ sftour scons lock
\end{verbatim}
The \texttt{sftour} command visits all subdirectories and runs \texttt{scons lock}, which copies result files to a different location so that they do not get modified until further notice.
\item You can also run
\begin{verbatim}
bash$ sftour scons -c
\end{verbatim}
to clean intermediate results.
\item Edit the file \texttt{paper.tex} to include your additional results. If you have not used \LaTeX\ before, no worries. It is a descriptive language. Study the file, and it should become evident by example how to include figures.
\item Run
\begin{verbatim}
bash$ scons paper.pdf
\end{verbatim}
and open \texttt{paper.pdf} with a PDF viewing program such as \textbf{Acrobat Reader}. 
\item Want to submit your paper to \emph{Geophysics}? Edit \texttt{SConstruct} in the 
paper directory to add \texttt{option=manuscript} to the \texttt{End} statement. Then run
\begin{verbatim}
bash$ scons paper.pdf
\end{verbatim}
again and display the result.
\item If you have \LaTeX2HTML installed, you can also generate an HTML version of your paper by running
\begin{verbatim}
bash$ scons html
\end{verbatim}
and opening \verb#paper_html/index.html# in a web browser.
\end{enumerate}

%\lstset{language=python,numbers=left,numberstyle=\tiny,showstringspaces=false}
%\lstinputlisting[frame=single]{SConstruct}

\bibliographystyle{seg}
\bibliography{school}



